\documentclass{../../sem_paper}
\newcommand\quest[1]{\subsection*{$\Rightarrow$ #1}}

\begin{document}
\titlepg
{Recherche über einen Beruf im öffentlichen Dienst}
{2010}
{SLM ABK Einführung}
{Seminar 50-500 Berufsfelderkundung}
{WS 2009/2010}
{Christopher Fittschen}

\tableofcontents
\thispagestyle{empty}
\pagenumbering{arabic}

\section{Einleitung}
Dem Bericht liegt ein Interview zu Grunde, welches mit Cordelia Weist, Beauftragte der Senatskanzlei für Osteuropakooperation der Hansestadt Hamburg und die Städtepartnerschaften zu St.\ Petersburg und Prag geführt wurde.

Mein Interesse an dieser Berufsrichtung fing an im Berufserkundungsseminar bei M.A. Christoph Fittschen, das ich immer Montags von 12-14 und das 13-mal innerhalb des Semesters besuchte. Wir hatten dort eine einige interessante Berufe vorgestellt bekommen, aber meine Motivation lag eigentlich nicht innerhalb des Vorgestellten.
Als ich noch zur Schule ging, war mir nicht wirklich klar, was ich später machen sollte. Diplomat zu werden, war für mich ein Traumjob, den ich wohl auch idealisiert habe. Wie auch immer, ein realistischer Beruf fiel mir nicht ein. 

Die Wahl meines Hauptfaches Slawistik schränkte meine Berufsmöglichkeiten nicht gerade ein. In der Uni musste man sich aber in eine bestimmte Richtung orientieren und bei den ABK-Seminaren hatte ich die Qual der Wahl. Schließlich hatte ich Glück und kam bei meinem Wunschseminar unter. Diese Seminare ``Werbung und Eventmanagement, Messe- und Kongressorganisation'', 50-035 ``Berufe mit (Fremd-) Sprachen, Tourismus'', 50-033 ``Öffentlichkeitsarbeit/Public Relations'' rangierten auf den Prioritätsstufen 2-4 -- in der Reihenfolge. 
Gleich zu Beginn wurden wir dazu aufgerufen, uns Interviewpartner zu suchen aus einem der folgenden Gebiete: Öffentlicher Dienst, öffentliche Unternehmen, int. Organisationen, Non-Profit-Organisationen, Stiftungen, Verbände, Wohlfahrtsorganisationen, Personalwesen. 
Die Vorgabe zu den Fragen haben wir in einer der Sitzungen erarbeitet. Dabei drehten  sie sich grob gesagt um den Werdegang, die Firma, die persönliche Arbeit und die Situation speziell für Geisteswissenschaftler.

\quest{Mit welcher Motivation haben sie das Berufsfeld gewählt?}
Gewählt habe ich dieses Seminar, weil ich im Grunde dem  Diplomatenberuf zugeneigt gewesen bin. Da dieser Beruf im öffentlichen Sektor angesiedelt ist, war für mich das Seminar „Öffentlicher Dienst, öffentliche Unternehmen, int. Organisationen, 
Non-Profit-Organisationen, Stiftungen, Verbände, Wohlfahrtsorganisationen, Personalwesen“ natürlich erste Wahl, aber auch die restlichen vom Seminar abgedeckten Berufsfelder wie int. Organisationen, Stiftungen, Verbände oder auch NGOs interessierten mich. Hier waren für mich sicherlich internationale Organisationen und NGOs am interessantesten, aber von denen gab es in Hamburg keine allzu bedeutenden.
Allerdings schwand mein Bedürfnis über einige der Felder etwas zu erfahren, da im Seminar die vom Dozenten eingeladenen Personen sehr ausführlich über diese Gebiete berichtet haben und eigentlich keine Fragen offen ließen. 

\quest{Nach welchen Kriterien haben sie ihren Interviewpartner ausgewählt? Und wie sind sie bei der Kontaktaufnahme vorgegangen?}
Ich habe bereits zeitig angefangen zu suchen. Nur gab es für mich keine klare Richtung, in der ich anfangen konnte zu recherchieren. Was ich mir am ehesten in der Schule vorstellen konnte, war der Diplomatenberuf, aber in Hamburg gab es kein Auswärtiges Amt oder so etwas Ähnliches. Von daher konnte ich mir diesen Wunsch nicht erfüllen. Aber schließlich kam ich auf die Idee, dass nicht nur die Bundesrepublik Kontakt zum Ausland unterhalten kann, sondern eine Stadt wie Hamburg. Dafür habe ich auf der Homepage der Stadt Hamburg unter "`Internationales"' geschaut und wurde fündig bei der internationalen Abteilung der Senatskanzlei. Hier hatte ich die Qual der Wahl, denn Hamburg als eine gut vernetzte Stadt hatte Abteilungen für alle Weltregionen.
Da mich aber der osteuropäische Raum interessierte, notierte ich mir die Nummer der Beauftragten für die Osteuropakooperation und die Städtepartnerschaften zu St.\ Petersburg und Prag.

Ich hatte zuerst Zweifel, ob es dieses Berufsfeld etwas für mich ist. Denn alleine die Berufsbeschreibung klang für mich nach Politik und Wirtschaft -- weit davon entfernt, was ich als Geisteswissenschaftler eigentlich studiere. Ich habe aber auch gehofft, dass zumindest die Kultur und die Sprache nicht zu kleine Aspekte des Berufes darstellen.
Ich war mir zudem ziemlich sicher auf einen Juristen zu stoßen oder einen Politologen. Auf jeden Fall habe ich mir nicht zu träumen gewagt, auf einen Geisteswissenschaftler zu treffen. In der Tat ist die von meiner Interviewpartnerin studierte Romanistik  ein Schwesternfach zu meinem Fach Slawistik. 
Insofern war ich angenehm überrascht, als sie mir ihren Werdegang darlegte. Ich habe nämlich beim Telefonat mit ihr nicht gefragt, was sie eigentlich studierte und ob sie aus sie geisteswissenschaftlichen Bereich „kommt“.

\section{Beschreibung des Berufes}

\quest{Beschreiben Sie den ausgewählten Beruf. Welche Tätigkeiten umfasst er?}
Wie bereits gesagt, ist dieses Berufsfeld recht groß, schließlich müssen alle Regionen der Welt abgedeckt werden. Man würde wohl viele Beschäftigte erwarten, aber tatsächlich arbeiten ca. 15 Leute  in der Abteilung „Internationale Zusammenarbeit“. Die Aufgaben sind aber umso größer. Ich werde im Folgenden die Tätigkeit meiner Interviewpartnerin vorstellen. Man kann aber wohl sagen, dass die unten vorgestellten Aufgaben sich auch bei anderen Mitarbeitern in der Senatskanzlei wieder finden.
Als Beauftragte für internationale Zusammenarbeit ist sie für Vermittlung von Kontakten nach Osteuropa zuständig, aber auch von Osteuropa ins Inland. So wird sie manchmal angerufen von Firmen oder öffentlichen Institutionen, um zum Beispiel Auskunft über Räumlichkeiten zu geben oder möglicherweise die Telefonnummer einer zuständigen Person.

Es passiert auch ab und zu, dass sie Feierlichkeiten zu Jubiläen und Besuchen organisieren muss. Denn nicht nur in Berlin geben sich Diplomaten aus fernen Ländern die Klinke in die Hand, auch Hamburg als zweitwichtigste deutsche Stadt und, was den Export angeht, die wichtigste Stadt, ist international gut bekannt. Die Besuche gilt es, zu planen und zu organisieren, was durch das Büro der Senatskanzlei geht. Hierbei kooperieren die Abteilung für Protokoll und die jeweilige Auslandsabteilung und natürlich ist häufig auch das Konsulat eines Landes beteiligt.

Bei hohen Besuchen wie zum Beispiel dem von Vladimir Putin im Jahr 2004 ist auch die Bundesebene beteiligt. Gerade bei Besuchen von Staatsmännern gerät allerdings  die internationale  Abteilung der Senatskanzlei in den Hintergrund.
Eine weitere zentrale Aufgabe ist das Sammeln von Informationen über die jeweilige Region. Hierbei sollte man zumindest die inländischen Zeitungen studieren und natürlich auch ein Auge für zukünftige Entwicklungen haben. Informationen, die man als relevant empfindet, sollte man dem Bürgermeister zukommen lassen.

Als Organisator für Jubiläen und Feierlichkeiten ist man auch für die teilweise Erstellung des Kulturhaushaltes zuständig. Als Zuständiger hat man deswegen auch gewisse Geldmittel zur Verfügung, um Projekte zu organisieren oder Projekte zu unterstützen.
Auch verläuft ein Großteils des Postverkehrs durch die internationale Abteilung. Es sind übersetzte Briefe oder Mails, die dann an die eigentlichen Adressaten weitergereicht werden. Hier wird des Weiteren die Post vom Bürgermeisterbüro ins Ausland verschickt, was zum Beispiel Einladungen oder Glückwünsche sein können.

In Zusammenarbeit mit der Protokollabteilung wird, verbunden mit der letztgenannten Tätigkeit, an offiziellen Briefen geschrieben.
Zusammenfassend kann man die Tätigkeit in diesem Berufsfeld als internationale Kontakt- und Kulturpflege bezeichnen.
Um ein Beispiel für Kulturpflege zu geben: Meine Interviewpartnerin musste  2007 das 50 Jährige Jubiläum der Partnerschaft zu St.\ Petersburg organisieren. Dabei muss man sich aber nicht zwingend auf die Senatskanzlei und ihre verschiedenen Abteilungen verlassen, sondern auch Stiftungen oder Personen beteiligen, die einem helfen können/wollen. Auf jeden Fall entstand durch das Zusammenspiel zwischen den beteiligten Personen und Stiftungen ein mit Präsentationen, Themenwochen, Diskussionsrunden usw.\  prallgefülltes Heftchen.
Das gleiche arbeitet meine Interviewpartnerin für die in diesem Jahr stattfindenden Feierlichkeiten zum 15-Jährigen Prager-Jubiläum aus.

\quest{Welche Einstiegswege in den Beruf gibt es? Wie sieht der berufliche Werdegang des Interviewpartners aus?}
Einen direkten Weg auf einen solch mit Erfahrung und Verantwortung verbundenen Posten gibt es nicht.
Meine Interviewpartnerin ist geborene Hamburgerin. Sie schloss in den 80ern ihr Studium in Romanistik und Völkerkunde/recht ab. Das Studium bestritt sie ohne konkrete Berufsvorstellungen, auch ein ähnliches Seminar, für welches ich den Bericht schreibe, konnte ihr keine konkreten beruflichen Vorstellungen vermitteln. Höchstens für die Buchbranche konnte sie sich begeistern, wo die Plätze allerdings selbst für sehr gute Absolventen sehr rar waren. Nach dem Studium wusste sie deswegen nicht so recht wohin. Eine tragische Fügung des Schicksals ermöglichte ihr einen Platz an der romanistischen Fakultät der Hamburger Universität. Lange Rede -- kurzer Sinn, an der Fakultät wurde sie nicht glücklich. Ihr öffnete sich aber eine andere Tür während ihrer Arbeitszeit in Uni.
Als sie einmal bei einer Delegation aus Lateinamerika als Übersetzerin anwesend war, machte sie ihren Job so gut, dass man sie dazu ermutigte sich im auf eine vakante Stelle im Europareferat der Stadt Hamburg zu bewerben. Dort wurde sie auch angenommen.
Nach ca. 8 Jahren wurde die jetzige Stelle frei. Diese Stelle wurde innerhalb der Senatskanzlei ausgeschrieben, sodass man sichergehen konnte, erfahrene und verdiente Bewerber vorzufinden.
Im Jahr 2000 wurde sie dann an die neue Stelle versetzt. Erwähnenswert ist, dass sie bei ihrer Anstellung im öffentlichen Dienst im gehobenen Dienst eingestellt wurde und diesen Status dann auch auf die jetzige Stelle übertragen hat, obwohl die Stelle eigentlich für den höheren Dienst gedacht ist.

Um an dieser Stelle ihre wichtigen Projekte zu erwähnen:
2004 konnte sie den deutsch-russischen Schüleraustausch nach Hamburg holen. Dieser nimmt nun als Stiftung in Hamburg Platz. Bereits 2001 von Vladimir Putin und Gerhard Schröder beschlossen, war der Ort des Sitzes der Stiftung nicht klar. Meine Interviewpartnerin hatte entscheidenden Anteil daran, dass sich Hamburg gegen Städte wie Berlin, Köln und München durchsetzte. Welche Bedeutung das Projekt hatte, ermisst man daran, dass Vladimir Putin persönlich nach Hamburg kam, um die nötigen Verträge zu unterschreiben.
Ein großes Projekt war ebenfalls 2007 die 50-Jährige Städtepartnerschaft zu  St.\ Petersburg. Als Hamburgs bedeutendste Partnerschaft wurde dieses Event zu einem Projekt mit einer Vorbereitungszeit von fast 3 Jahren.
Das nächste große Ereignis steht dieses Jahr mit dem Jubiläum der Partnerschaft zu Prag an.


\quest{Welche Qualifikationen sollte man zur Ausübung dieses Berufes mitbringen (fachlich/sozial/selbstbezogen)?}
Die Kompetenzen für einen solch bedeutenden Posten sind weitreichend und anhand der aufgezählten Aufgaben kann man sich auch gewisse Kompetenzen selbst denken.
Obgleich sie in der Senatskanzlei für ihren Bereich alleine verantwortlich ist, spielt Teamwork eine bedeutende Rolle innerhalb der sozialen Kompetenzen, denn viele Projekte werden mit Stiftungen oder Unternehmen auf die Beine gestellt, sodass man die Eigenschaften und Bedürfnisse der Beteiligten schon kennen sollte, was man weiterführend als Menschenkenntnis bezeichnen kann. Wichtig ist, dass man auch über den Tellerrand blickt und vielleicht die Unterschiede zwischen den Philosophien der Stiftungen und Unternehmen im Blick hat, um die Zusammenarbeit zwischen verschiedenen Ebenen der Wirtschaft zu organisieren.
Verbunden damit, sollte man Projekte auch detailbezogen organisieren können. Es soll bedeuten, dass man im Gewirr der Termine und teilnehmenden Personen den Überblick behält. Auch passiert es manchmal, dass man an einigen Projekten parallel arbeiten muss und vielleicht erstreckt sich die Verwirklichung einer Idee über Monate hinweg.
Bestes Beispiel ist das 50 Jährige Jubiläum zu St.\ Petersburg, wo es eine Menge zu organisieren gab, mit vielen verschiedenen Personen, Stiftungen usw.\ und dabei ist man schon 3 Jahre im Voraus mit der Planung beschäftigt und sammelt im Laufe der Zeit bis zur Verwirklichung eine Unmenge Material an, das in regelmäßigen Abständen auch hervorgeholt werden muss, wenn ein Projekt einen Fortschritt verzeichnet.

Da man sehr viel mit Menschen arbeitet und kommuniziert, ist Kommunikationsfähigkeit eine sehr wichtige Kompetenz. Nicht zuletzt führt man eine Reihe von Gesprächen und Telefonaten für das Erreichen eines Ziels. Dabei haben die Beteiligten manchmal andere Philosophien oder Systeme, die einer Zusammenarbeit nicht zuträglich sind. Da braucht man ein gewisses Gespür für Diplomatie und die richtigen Worte, um Kompromisse auszuhandeln und jeden Beteiligten glauben lassen, er habe das größere Stück vom Kuchen abbekommen. Hierbei spielt die Fähigkeit zu überzeugen eine gewichtige Rolle.
Aber wegen der Menge und Komplexität mancher Aufgaben kommt es eben auch manchmal zum  Scheitern und dann ist es wichtig ein dickes Fell und eine gewisse Frustresistenz zu haben. Es kann sich ein von den Eigenschaften her schönes Projekt anbieten, aber vom finanziellen Standpunkt her nicht zu verwirklichendes. Und es passiert nicht gerade selten, dass ein Projekt super aussieht, aber zu teuer ist. Auch kann es sein, dass im Ablauf nicht alles reibungslos abläuft oder sich zufällig etwas Unerwartetes ereignet. Auf jeden Fall kann man wegen der Komplexität nicht alles 100\% übersehen und planen.

Fachlich gesehen sind ebenfalls allerhand Fähigkeiten gefragt:
Da man oft mit Recht und Rechtsfragen zu tun hat, ist ein Gespür für juristische Dinge ganz nützlich. Weil meine Interviewpartnerin Recht bezogen auf Völker studiert hat, hat ihr das besonders bei ihrer Bewerbung geholfen. Es ist zwar nicht so, dass man dauernd nur Paragraphen wälzt, trotzdem  ist Jura wichtig, was auch die hohe Zahl der Mitarbeiter mit einem Juraabschluss beweist.
Da man auch mit Erstellung des Haushaltes und Bewilligung von Mitteln zu tun hat, sind grundlegende Dinge aus dem Rechnungswesen nützlich.

Selbstverständlich  ist ein Sinn für Kultur. Diese interkulturellen Kompetenzen helfen dabei die Aufgabe der Kulturpflege zu schultern. Wie schon beschrieben ist man häufig auch mit kulturellen Projekten wie zum Beispiel Jubiläen beschäftigt. Daher ist es wichtig Kulturfragen nicht abgeneigt zu sein, sondern mit Interesse sich über aktuelle Künstler, Schriftsteller und Sänger auf dem Laufenden zu halten.
Ich schätze auch Englisch als eine fachliche Fähigkeit ein. Zumindest ist es eine wichtige Kompetenz, besonders in einer globalisierten Welt. Wie meine Interviewpartnerin mir sagte, war Englisch bei ihrem Eintritt in den öffentlichen Dienst um 1990 keine Kompetenz, auf die man in ihrem Lebenslauf besonders geachtet hat.

Im Laufe der Zeit ist es aber immer wichtiger geworden. Jetzt bekommen die Mitarbeiter wöchentlich Unterricht in der englischen Sprache. Man mag wohl glauben, dass in außerenglischen Regionen wohl eher die Nationalsprache erforderlich ist. Dem ist aber nicht so. Tatsächlich konnte meine Interviewpartnerin kein Russisch oder eine der osteuropäischen Sprachen. Die Gespräche mit dem Ausland werden meistens auf Englisch geführt. Man sollte aber sagen, dass entgegen der Erwartung nicht so viele Gespräche ins Ausland geführt werden, beziehungsweise nicht mit Ausländern. Dafür sitzen an allen wichtigen Regionen, mit denen Hamburg handelswirtschaftlich verknüpft ist, auch in der Nationalsprache kundige Leute und ein Grossteil der Gespräche wird auch von ihnen übernommen. Übersetzungsarbeiten für Dokumente oder ähnliches werden  von speziellen Büros übernommen. Ein umspannendes Netz aus Kontakten wird ebenfalls von dem vor Ort sitzenden Zweigbüro geschaffen.

Insofern sollte man die sprachlichen Anforderungen nicht überbewerten: Englisch ist schon wichtig, aber man lernt es heutzutage intensiv in der Schule und ist daher gut vorbereitet. Trotzdem kann die Beherrschung von einigen anderen Sprachen die Arbeit erheblich erleichtern, da dadurch Zeit im Hinblick auf die Bearbeitung von Aufgaben gespart wird. Wie mir meine Interviewpartnerin versicherte, würde sich die Beherrschung von außerenglischen Sprachen gut in der Vita bei der Bewerbung machen, obgleich die wichtigen  Kriterien bei der Erfahrung liegen.
Alles in Allem überwiegen wohl die sozialen und selbstbezogen Kompetenzen beim Stellenwert für den Job, fachliche Kompetenzen sind vielleicht besser zu einzuschätzen, spielen aber nicht die gleiche Rolle.

\quest{Welche Kompetenzen werden in einem geisteswissenschaftlichen Studium vermittelt?}
Was sie aus ihrem geisteswissenschaftlichen Studium nehmen konnte, war das Kulturverständnis, besonders für den romanischen Raum, sprich Italien, Frankreich, Spanien, Portugal und Lateinamerika. Dies hat ihr bei ihrer ersten und zweiten Station geholfen, d.h.\ an der romanistischen Fakultät und im Europareferat der Stadt Hamburg. Hier waren auch tatsächlich Sprachen wie Spanisch bzw. Französisch wichtig.

Im Rahmen eines geisteswissenschaftlichen Studiums hat ihr auch der Umgang mit Texten geholfen, Informationen über dieses und jenes zu sortieren und richtig zu bewerten. Auch hat sie der Überzeugung Ausdruck gegeben, dass besonders im geisteswissenschaftlichen Studium die sozialen Kompetenzen besonders zum Tragen kommen und daher besonders entwickelt werden. Im Gegensatz dazu steht ihr Studium in Völkerkunde und Völkerrecht, wo sie viel fachliches Wissen, aber ihrer Meinung nach nur wenig die sozialen Kompetenzen trainierte.

\quest{Wie bewertet der Interviewpartner seinen Beruf?}
Die Bewertung ihres Arbeitsplatzes fällt sehr positiv aus. Obgleich sie sich diesen Job nicht vorstellen konnte zu Studiumszeiten, wurde sie schnell mit dem Job im Europareferat vertraut. Natürlich war für sie der Beruf zu Anfang trotz Berufserfahrung wie ein Sprung ins kalte Wasser. Denn in der Praxis das Gelernte umzusetzen, fällt doch schwer, besonders ohne größere Erfahrung in einem Gebiet.
Mit dem Wechsel zur Beauftragten der Osteuropakoordination waren allerdings keine großen Schwierigkeiten verbunden, denn die Arbeitsmechanik ähnelte der vorangegangenen Beschäftigung.
Die zehn Jahre, die sie nun diesen Beruf ausübt, haben ihr eine Menge interessanter Momente beschert. Zum Beispiel sprach sie mir gegenüber gerne davon, dass sie dem damaligen russischen Präsidenten Vladimir Putin die Hand geschüttelt hat. Außerdem hat sie auch schon einige höherrangige Diplomaten  und auch Bürgermeister von Städten wie Prag oder St.\ Petersburg  getroffen.
Dazu konnte sie viele tolle Projekte organisieren oder finanzieren. Dabei konnte sie interessante Menschen treffen und viel lernen. Folgerichtig hat sie sich eine gewisse Identifikation mit dem Beruf aufgebaut.
Überhaupt entsprechen die Eigenschaften des Berufes ihrem Charakter, dazu zählt zum Beispiel auf Menschen zuzugehen, Neugier zu zeigen, sich mit Kultur befassen usw.

Das macht auch die relativ geringe Bezahlung (auf dem Niveau des gehobenen Dienstes) und die manchmal hohe Arbeitszeit (besonders in den Schlussphasen eines Projektes)  wett. Den damit verbundenen Stress kompensiert sie durch ihre Familie. Ansonsten hat sie eine  normale 40-Stunden-Woche, von 9 Uhr morgens bis ca. 17/18 Uhr abends, wobei das auch variieren kann.
Positiv hat sie des Weiteren angemerkt, dass natürlich der sichere Arbeitsplatz eine wichtige Rolle spielt.
Ihr Wunsch wäre zwar das Lateinamerika und Spanien-Referat, aber, wie gesagt, mit dem derzeitigen Job ist sie absolut glücklich. 
Im Übrigen ist sie mit \_ Euro Netto für den Job unterbezahlt, aber ihrer Aussage nach spielt die Bezahlung mit 50 Jahren keine bedeutende Rolle bei der Beurteilung eines Jobs.
Zuletzt sollte auch gesagt werden, dass die Arbeitsatmosphäre innerhalb der Senatskanzlei eine ausgezeichnete ist und die Kollegialität nichts zu wünschen übrig lässt.

\quest{Wie ist die ausgewählte Branche im Hinblick auf die Beschäftigungschancen für Bachelor- Absolventen einzuschätzen?}
Aufgrund der Verantwortung und der komplexen Aufgaben, die eine Anstellung in der Senatskanzlei mit sich bringt, ist es äußerst schwer für einen BA- Absolventen sich erfolgreich um eine Stelle zu bewerben.
Denn die aufgezählten Aufgaben verlangen Berufserfahrung. Auch meine Interviewpartnerin war zuerst einige Jahre an romanistische Fakultät angestellt, bevor sie die Chance bekam, sich im Europareferat zu beweisen. Es gehen natürlich einige Jahre ins Land  bevor man sich die notwendige Berufserfahrung holt. Es bietet sich aber auf jeden Fall an, im öffentlichen Dienst seine Laufbahn zu beginnen, denn gerade höhere Stellen werden nur innerhalb des öffentlichen Dienstes ausgeschrieben. Man findet sie nicht in einer Stellenanzeige oder einer Jobbörse.

Zur Jobsituation im öffentlichen Dienst kann ich nur wenig berichten, da meine Interviewpartnerin vor einer relativ langen Zeit angefangen hat zu arbeiten. Anhand ihres Werdegangs kam man aber schon sagen, dass neben den Kompetenzen auch ein bisschen Glück dazugehört. Zu ihrer eigenen Anstellung im Europareferat sagte sie, dass sei auch großes Glück gewesen.  


\section{Schlussbetrachtung}

\quest{Welches sind aus ihrer Sicht die wichtigsten Erkenntnisse aus der Erkundung des Berufsfeldes hinsichtlich des eigenen Werdegangs?}
Die Erkenntnisse aus dem Interview sind vielfältig. Zum einen ist es wichtig, sich sozial immer weiter zu entwickeln. Im Berufsleben ist zum Beispiel Teamwork sehr gefragt, was man in seinen Haupt- und Nebenfächern nicht wirklich lernt. Da sollte man sich bewusst machen auf welcher Ebene man steht und ob das fürs Berufsleben ausreicht.
Zum anderen sollte man sich auch stärker mit der Politik und Wirtschaft beschäftigen. Man sollte es lernen über den Tellerrand zu blicken. Klar gewinnt man mit einem geisteswissenschaftlichen Profil einen guten Blick auf die Kultur, aber das ist auch nur ein Teil des Ganzen. Will man die Aufgaben erledigen in dem  vorgestellten Berufsfeld, sollte man Kompetenzen in allen Richtungen haben -- Wirtschaft, Politik, Kultur. Gerade im geisteswissenschaftlichen Studium werden die ersten zwei Bereiche vernachlässigt, sodass man sich hierbei selbst weiterbilden müsste. Sieht man vom Berufsfeld ab, so ist es immer lohnend auch über seinen Bereich zu schauen, denn in einer globalisierten Welt vermischen sich auch Politik, Kultur und Wirtschaft. 

Stichwort Globalisierung: Es ist auch wichtig, zu kommunizieren. Englisch ist natürlich Pflicht. Deswegen sollte man es nicht einrosten lassen. Man sollte sich auch so viel Selbstehrlichkeit aufweisen, um über seine Fremdsprachenkenntnisse kritisch zu urteilen. Weiterbildungskurse -- auch im fremdsprachlichen Bereich -- gibt es an jeder Universität.
Auf jeden Fall schreckt mich die Bandbreite der Aufgaben nicht ab, später in die gleiche berufliche Richtung zu streben. Auch wenn die Bezahlung vielleicht nicht die beste ist, so ist das Aufgabenfeld sehr attraktiv und die sichere Anstellung allemal. 

Das Auskundschaften dieses Berufsfeldes hat mir aber auch verdeutlicht, dass ich ohne Berufserfahrung kaum Chancen auf eine solche Anstellung habe. Der Werdegang meiner Interviewpartnerin hat mir klar gemacht, dass man möglichst einen vollen Lebenslauf haben sollte -- zum Beispiel auch durch Anstellung in der Universität. Mit ein wenig Glück kann man dann eine bessere Anstellung im öffentlichen Dienst bekommen.
Ich denke die Attraktivität des öffentlichen Dienstes ist über alle Zweifel erhaben und deswegen ist es auch schwer reinzukommen, aber mit einer guten Qualifikation wird man nicht jede Tür, an die man klopft, geschlossen finden. 
Was ich auf jeden Fall in Angriff nehmen werde, ist ein Praktikum in der Senatskanzlei, wodurch ich hoffe, besseren Einblick in die Strukturen und Arbeitsabläufe innerhalb der Senatskanzlei zu bekommen.

\quest{Haben sich ihre Annahmen über das Berufsfeld bestätigt?}
Als ich mir den Titel angeschaut hatte, erwartete ich ein größtenteils politisches Aufgabenfeld, umso überraschter war ich dann auch, dass der Beruf sehr viel mit Kultur zu tun hat. Hier und da kulturelle Projekte zu organisieren, hätte ich in der Senatskanzlei nicht erwartet, doch tatsächlich ist es so. Natürlich gibt es auch Aufgaben, die mit der Politik oder mit Recht zu tun haben und ihre Wichtigkeit ist nicht zu  unterschätzen, aber nimmt man die Aufgaben in ihrer Quantität, so treten sie in den Hintergrund.
Eine weitere Überraschung war für mich die fremdsprachliche Kompetenz. Ich hätte nicht erwartet, dass die Sprachen der Region mit der man sich beschäftigt, relativ unbedeutend sind und Englisch die Nationalsprachen überflüssig machen kann. Dass Englisch ein immer wichtigeres Kriterium in der Vita ist, wurde mir durch das Interview klar.

Weiterhin habe ich nicht erwartet, dass man so viele Aufgaben alleine erledigen kann -- besonders von so verschiedener Natur. Doch es geht und zwar mit hoher Selbstdisziplin und ständiger Weiterbildung. 
Auch ist es zu meiner Beruhigung so gewesen, dass auch Geisteswissenschaftler Beschäftigung in der Senatskanzlei finden können. Die Mehrzahl der Mitarbeiter hat zwar einen juristischen Hintergrund, aber auch Geisteswissenschaftler konnte man auf höheren Positionen finden. Nicht zuletzt ist auch der Chef der Senatskanzlei ein promovierter Historiker.
Ich habe mir über das Gehalt oder die Arbeitszeit keine Illusionen gemacht und wurde von daher auch über den vergleichsweise niedrigen Lohn auch  nicht stutzig. Die im Mittel liegende Arbeitszeit ist aber natürlich mit Vorsicht zu betrachten, da die tatsächliche notwendige Zeit zur Bewältigung der Aufgaben höher liegen kann.
Im Endeffekt wurden einige Annahmen glücklicherweise nicht bestätigt wie zum Beispiel der Charakter der Aufgaben und die notwendigen Kenntnisse in nichtgeisteswissenschaftlichem Gebieten.

\quest{Wie bewerten Sie den Bezug zwischen dem geisteswissenschaftlichen Studium und dem von ihnen betrachteten Beruf?}
Ich habe nicht gedacht, dass der Bezug zwischen dem geisteswissenschaftlichem Studium und dem vorgestellten Berufsfeld so groß sein kann. Gut, die fachlichen Kompetenzen sind nicht so offensichtlich, im Gegensatz zu einem juristischen Fach, wo das Paragraphenpauken einen Sinn hat. Aber die  Kompetenzen, die meine Interviewpartnerin beim Romanistikstudium erworben hat, haben ihr geholfen, die Berge von Texten durchzuarbeiten und wichtige Informationen zu sortieren. Sie denkt auch nicht, dass ihr ein Jura-Studium ihre heutige Arbeit vereinfacht hätte.
Im Laufe seines Studiums erwirbt man des Weiteren auch ein passives Verständnis für Kultur. Wie schon beschrieben ist eine Reihe von Aufgaben kulturell angehaucht, sodass das geisteswissenschaftliche Studium sich auszahlt.
Es ist nur schwierig solche Kompetenzen sichtbar in der Vita zu machen. Da sind besonders Jurastudenten im Vorteil. Trotzdem werde ich den Sinn meines Studiums in Zukunft weniger hinterfragen.

\quest{Welchen Einfluss haben die gesammelten Erfahrungen auf ihre eigenen Berufswünsche?}
Das Interview hat mich in meinem Berufswunsch auf jeden Fall bestärkt. Ich habe mir dieses Berufsfeld nicht als Notlösung gewählt, sondern weil mein Interesse tatsächlich hier liegt. So werde ich meinen Kurs auch fortsetzen. Natürlich hat der öffentliche Dienst den klaren Nachteil, dass der Verdienst niedriger ist als in der  Marktwirtschaft. Die Aufgaben sind aber mindestens genauso interessant und die relative Jobsicherheit ist auch den Verzicht auf eine steile Karriere oder höhere Bezahlung wert. Schwierig erscheint es, Fuß zu fassen im öffentlichen Dienst. Da muss ich mich auf jeden um ein Praktikum bemühen.
Mir wurde aber auch klar, dass einen am Anfang keine Superjob erwartet. Erst mit zunehmender Erfahrung kommt man auch für wichtigere Stellen in Frage. Hier werde ich mir sicher keine Illusionen in Zukunft machen.

\quest{Was könnten Sie außerhalb des Studiums tun, um sich weiter für das Berufsfeld zu qualifizieren?}
Natürlich muss man seinem Englisch  auf die Sprünge helfen. Sofern man nicht Anglizistik/Amerikanistik studiert, ist es ein generelles Problem sein Englisch auf das Niveau der Erwartung seitens des künftigen Arbeitgebers zu bringen. 
Grundlegende Kenntnisse in Politik, Recht und Wirtschaft sollte man auch erwerben. Ich denke, dass ich dazu den Wahlbereich nutzen kann.

Weiterhin sind Kenntnisse in IT zunehmend von Bedeutung, wie mir meine Interviewpartnerin sagte. In Zukunft wird darauf von Arbeitgebern mehr geachtet werden. Solche Defizite kann man in Kursen außerhalb der Uni wunderbar nachholen.
Zuletzt ist ein Auslandsaufenthalt ein absolutes Muss, um Aufgaben im Europareferat oder in der Osteuropakooperation zu meistern. Das muss nicht unbedingt zum Studium gehören, sondern kann auch auf eigene Faust unternommen werden.

\quest{Welche Einfluss haben die Erkenntnisse aus der Erkundung auf die Wahl eines Praktikums im folgenden Modul „Berufs- und Bewerbungspraxis“?}
Ein Praktikum in Senatskanzlei ist möglich, das versicherte mir meine Interviewpartnerin. Jährlich begrüßt man mehr als ein dutzend Praktikanten in den verschiedenen Bereichen.
Allerdings muss man sich schon bereits 1 Jahr im Voraus bewerben und auch dann kommt natürlich nicht jeder an einen der begehrten Plätze. Als Praktikant muss man telefonieren, kleinere Dinge organisieren und vielleicht seine sprachlichen Fähigkeiten einsetzen. Wenn man zum Beispiel Russisch kann, würde man das auch gebrauchen.
Mir persönlich gefallen die gesammelten Erkenntnisse und ich werde mich auf jeden Fall um einen Praktikumsplatz in der Senatskanzlei bemühen. Wenn es damit nicht klappt, werde ich mich in einem anderen Bereich des  öffentlichen Dienstes für ein Praktikum bewerben. Die Aufgaben hier sind nämlich keineswegs langweiliger als in der Wirtschaft. Dazu gibt es noch viele andere Vorteile, die alles in allem den öffentlichen  Bereich attraktiv erscheinen lassen.
\end{document}
