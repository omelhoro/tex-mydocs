\documentclass{../../sem_paper}

\begin{document}
\titlepg
{Reflektion über eigene interkulturelle Erfahrungen}
{2014}
{SLM ABK Schlüsselqualifikationen}
{Seminar 50-014 Interkulturelle Kompetenz}
{SS 2014}
{Naida Mehmedbegovic Dreilich}

\tocpaper

\closuresection{Einleitung}

Was ist interkulturelle Kompetenz? Mit dieser Frage ließe sich sehr lange beschäftigen wie ich aus unserem Kurs erfahren habe. 
% Auf jeden Fall gibt es keine Lösung, die schnell eingesetzt werden kann. 
Und trotzdem lohnt es sich definitiv für bewanderte Menschen einmal über die eigenen Erfahrungen zu reflektieren und für sich dabei die Wichtigkeit der interkulturellen Kompetenz zu entdecken. 
Ich will meine Erfahrungen anhand von drei Fällen reflektieren, die mir in Erinnerung geblieben sind:
\begin{enumerate}
 \item Eine 4-wöchige Sommerschule in Almaty (Kasachstan) im September/Oktober 2011
  \item Auslandssemester in St.\ Petersburg (Russland) im Frühjahr/Sommer 2013
  \item Praktikum in der Deutschen Botschaft in Chisinau (Moldawien) im Juli/August 2013
\end{enumerate}

Dabei will ich zuerst die Situation beschreiben und diese danach innerhalb des Frameworks der interkulturellen Sensibilität betrachten.

Ich habe auch lange überlegt, ob ich meine längste Begegnung mit einer anderen Kultur reflektieren soll, nämlich meine Ankunft in Deutschland im Alter von 9 Jahren. 
Allerdings sind meine Erinnerungen daran nicht wirklich groß und ich bin mir selbst nicht bewusst, wie ich mich damals in eine wirklich andere Kultur integriert habe. Ich schätze, je jünger man ist, desto schneller integriert man sich, kann aber auch die Erfahrungen kaum analysieren.

\section{DAAD-Sommerschule in Kasachstan}
Innerhalb der 4-wöchigen Sommerschule in Kasachstan traf ich das erste Mal auf interkulturelle Missverständnisse, lernte diese aber auch zu relativieren. 
Wir waren eine Gruppe von 15 Studenten, die in Gastfamilien untergebracht waren. 
Größtenteils waren wir in Almaty, hatten aber auch einen Wochendausflug nach Taraz (etwas südlicher). 

Interessant war dieser Zeitraum, weil ich eigentlich aus Kasachstan komme und mir sicher war, gur zurechtzukommen. 
Ich hätte nicht gedacht, dass der Umstand, dass ich aus dem Norden (Temirtau) komme, keine Hilfe sein würde.

Tatsächlich fingen die Unterschiede bei Essgewohnheiten an, gingen über die Umgangsformen und Sprache bis hin zur Religion. 

Nicht nur mir, sondern der ganzen Gruppe wurde Pferdefleisch serviert. 
Manche fanden das nicht so toll, aber in der kasachischen Kultur bedeutete das eine große Anerkennung – eine frühere Sitte, dass man dem Gast das wertvollste serviert, was man hat, nämlich das Pferd (Kasachen waren früher ein Nomadenvolk). 
Vegetarier in unserer Gruppe hatten keine leichte Zeit, da dieser Lebensstil in Kasachstan und Osteuropa nicht präsent ist. 
Andererseits unterschieden manche Familien kein spezielles Frühstücksessen -- es wurde das vom vorherigen Abend gegessen. 

Während des angesprochenen Wochendausflugs nach Taraz waren wir ebenfalls in Gastfamilien untergebracht und dort erlebte ich zum Beispiel zwei größere Missverständnisse. 
Im Gegensatz zur ehemaligen Hauptstadt Almaty, wo wir den größten Teil verbracht haben, war Taraz eben eher religiöser und vielleicht auch tradioneller. 
Beim Essen vor der Abreise wurde gebetet und ich wurde angehalten, ein Gebet zu sprechen, ich aber keines kannte. 
Ich habe von der Familie nicht wirklich viel gesehen, weil wir viel unterwegs waren und in diesem Augenblick war ich etwas glücklich, dass die Abreise anstand. 
Umso überraschter war ich, als man uns (ich war zusammen mit einem anderen Studenten) bei der Abreise Geschenke in die Hand drückte. 
Das waren u.a.\ ein Aschenbecher und ein muslimisches Käppchen – also eher Sachen, die für uns keinen großen Wert hatten.

\subsection*{Reflektion}
\addcontentsline{toc}{subsection}{Reflektion}

Ich denke, ich habe zuerst den Fehler gemacht, zu glauben, dass selbst im Süden Kasachstans die gleiche Kultur wie im Norden geben würde. 
Die Annahme, dass ich wie selbstverständlich zwischen den Kulturen (der Deutschen und der Russischen) pendeln würde, erwies sich schon deshalb als fehl, da die kasachische trotz russischer Elemente doch eine eigenständige Kultur ist. 
Hilfreich dabei war, dass wir ja auch immer wieder Kulturkurse hatten und die DAAD-Ausrichter gute kasachische Kooperationspartner hatten. Dies half dabei, die Kultur unvoreingenommen, besonders aber im Lichte der sehr alten Geschichte dieses Volkes zu betrachten. Insofern konnte man die angesprochene Situation mit dem Pferdefleisch zumindest danach im Kontext sehen. 
Ein Gebet vor der Abreise ist im Kontext eines so großen Landes wie Kasachstan (4x Deutschlands Fläche) verständlicher, ebenfalls die emotionale Verabschiedung von Gästen. 
Ebenfalls erfuhr ich später, dass die Geschenke auch nicht dispektierlich gemeint waren.

Für mich war es außerdem sehr hilfreich in einer größeren Gruppe zu sein, wo man seine Erfahrungen austauschen konnte. 
So übertrug sich ein bisschen an Neugier auf die Kultur auch auf mich und ich wurde auch gewahr, dass die Unterschiede eben aus einem größeren historischen Kontext kommen als dem post-sowjetischen Umbruch oder einfach einer Abneigung gegen den Westen (so interpretierten manche von uns die ersten Erfahrungen).

In der ersten Zeit haben wir aber doch einen Fehler gemacht, den wir zuerst gar nicht gemerkt haben: 
Wir haben einen Fahrer angeheuert, der uns zur Uni fuhr. 
Wir machten dabei ein Studentin zu unserer Wortführerin bei der Preisverhandlung, der Fahrer war aber sehr verdutzt als er mit einer Frau den Preis verhandeln sollte; 
nach ein paar Tagen mussten wir mit dem Bus fahren, weil wir mit dem Fahrer nicht verhandeln konnten. 
Wir hatten angenommen, er hätte was gegen uns aus dem Westen, dabei ist die Rolle der Frauen im (südlichen) Kasachstan eher tradioneller.

\section{Semesteraufenthalt in St.\ Petersburg}
Einer meiner Gründe für den Aufenthalt in St.\ Petersburg war, dass ich glaubte, die russische Kultur zu kennen. Dies erwies größtenteils als Trugschluss. Tatsächlich war die Zeit von vielen Missverständnissen gekennzeichnet. War ich am Anfang noch in einer Verleugnungphase, entwickelte ich mich im Laufe des Semester in eine Abwehrphase.

Besonders in der Uni traf ich auf Schwierigkeiten, die ich gar nicht erwartet hatte. Zum Beispiel war für mich die Anwesenheitsmoral sehr überraschend, denn die Studenten kamen tatsächlich sehr wenig zum Unterricht, wobei die Dozenten das auch locker aufnahmen.

Größere Probleme hatte ich mit dem Unterrichtsstil wie auch ein Mitkommilitone aus der Schweiz. Bis zuletzt fehlte uns die Motivation, Zeit in den Unterricht zu investieren. Der Unterricht war sehr frontal. Wir konnten uns nicht mit dem Gedanken anfreunden, in jedem Kurs jeden Gedanken des Dozenten mitschreiben zu müssen. Fragen seitens der Studenten waren dabei eher überraschend. Schwierigkeiten kamen auf, als ich gut gemeinte Kritik übte, wie man den Unterricht besser machen könnte. Was ich damals als ganz normal empfunden habe, wurde seitens der Dozenten eher so wahrgenommen, dass ich als Deutscher meine Kultur einfach für besser hielt.

Als meine Ratschläge nicht ankamen, fing ich wohl unterbewusst an, meine Kultur tatsächlich für besser zu halten und die russische Kultur mehr und mehr zu stereotypisieren. Naheliegend war zum Beispiel das in westlichen Medien häufige Bild von Russland als einem Land, das nur von Rohstoffen lebt und wo “geistige Errungenschaften” eine Nebenrolle spielen. Interessant war aber auch, dass die Russen von mir nun als emotional eingeschätzt wurden, während ich aus Deutschland gewohnt war, dass man Emotionen zumindest vom akademischen Bereich trennt. Dies vermischte sich in Russland außerdem mit einem sehr hohen Geschichtsbewusstsein (umso überraschender, da ich dort nicht Geschichte, sondern moderne Linguistik studiert habe): Nicht selten wurden die akademischen Leistungen der USA geschmälert, indem man ihnen Opportunismus im 2. Weltkrieg vorwarf (ein Dozent ging sogar bis zu den Indianerkriegen) und dies in sehr emotionaler Rhetorik vortrug. Weiterhin wurden eben auch viele Namen im Kurs erwähnt: Während in Deutschland eher die Vorgänge und Tendenzen wichtig sind, so war in Russland die Kenntnis von möglichst vielen Namen und Fakten wichtig. So waren Vorträge über Schriftsteller gefüllt mit Lebensdaten und Namen von ihren Werken.

Dazu kamen auch Sorgen über die Administration. Mein Stipendium wurde erst mit Verspätung ausgezahlt. Und die Anforderungen für den Transcript of Records blieben fast bis zuletzt im Unklaren.

\subsection*{Reflektion}
\addcontentsline{toc}{subsection}{Reflektion}

In meinem Semester in St.\ Petersburg bin ich im Nachhinein nicht über die Schwelle der Abwehrphase hinausgekommen. Ich denke, dass es wohl an mir lag, dass das Semester eher negativer Natur war. Beispielsweise war ich sehr motiviert, akademisch in St.\ Petersburg weiterzukommen. Dabei war ich vorher über die sehr vielfältigen Kurse beeindruckt, die die Uni dort im Programm hatte. Ich dachte nicht, dass diese Kurse ganz anders geführt würden als in Deutschland. Diese Erwartungshaltung mit dem Druck eine gewisse Punktzahl anzusammeln, führte wohl zu Konfikten, dass ich die Kurse an meine Wünsche anpassen wollte. Später, als dies nicht gelang, sah ich in den Dozenten eher inkompetente Menschen und ging nicht mehr zu Kursen aus der Überzeugung heraus, dass ich dort nichts lernen kann.

Gleichermaßen habe ich wohl aus der falschen Überzeugung, dass ich eben die Kultur kenne, falsche Maßstäbe an die Adminstration angelegt. Schon 2 Wochen nach meiner Ankunft wollte ich das Stipendium, wobei es erst nach 2 Monaten überwiesen wurde.

Mir fällt es etwas schwer, eine genaue interkulturelle Phase zu nennen, wo ich angefangen habe: Womöglich habe ich mich vor dem Abflug schon bereits in der letzten Phase der Integration verortet  und daher keine Schwierigkeiten erwartet. In Wirklichkeit war ich aber wohl in der Verleugnungphase, denn ich habe unterbewusst die gleichen Maßstäbe angelegt wie in Deutschland. 

Schlussendlich hätte ich definitiv offener sein müssen und auch mehr Vertrauen in die Uni haben müssen, da ja trotz allem die Uni so immer funktioniert hat und ich alles bekam, was ich brauchte. Die Dozenten waren nicht so unnahbar wie das schien, sondern waren sich über die Lage der Auslandsstudenten im Bilde und die gelegentlichen geschichtlichen Ausschweifungen kann man im Kontext der politisierten sowjetischen Bildung verstehen. Das ToR habe ich bekommen, weil die Administration trotz meiner Bedenken funktioniert hat. Im Nachhinein wurde mir klar, dass ich auch meine BA-Arbeit aus einem Kurs abgeleitet habe, von dem ich zu der Zeit enttäuscht war.

Im Hinblick darauf, wieso ich nicht über die Abwehrebene hinausgekommen bin, so denke ich, waren mein Ehrgeiz und eine Übermotivation die schuldigen Faktoren. Obwohl sich meine Erfahrungen wirklich nur auf bestimmte Situationen bezogen haben, bezog ich das generell auf die russische Kultur. Dies führte wohl auch zu einer generell negativen Wahrnehmung von anderen Situationen.

So war eine negative Wahrnehmung von einzelnen Personen verantwortlich für eine voreingenommene Begegnung mit anderen. Später, fast am Ende des Semesters, erfuhr ich, dass  nicht alle Dozenten so waren. Augehend davon, dass ich zuerst dachte, ich könnte zwischen den Kulturen pendeln, entwickelte ich mich eher in die Richtung, dass meine deutsche Kulturorientierung für mich besser war und die russische empfand ich als schlechter. Dies verhinderte eine Minimierung der Unterschiede und eine Anpassung an diese. Ich denke, ich war auch ein bisschen geschockt, dass solche Unterschiede zu Tage traten, daher merkte ich wohl auch nicht, dass ich diese “Ungereimtheiten” mit gültigen Stereotypen zu erklären versuchte, anstatt über meine Erwartungen zu reflektieren.

\section{Praktikum in der Republik Moldau}

Gleich im Anschluss an das Semester in St.\ Petersburg flog ich zum 6-wöchigen Praktikum in der dt. Botschaft in Chisinau. Dort beobachtete ich durchaus Probleme in Zusammenarbeit mit Moldauern, wobei ich diese Erfahrung eher positiver als in Russland empfand. Nicht selten wurde die Arbeit behindert durch Behörden, die etwas als Gegenleistung erwarteten. Dies war eigentlich so verkappt, dass mich die Ausreden der Behörden zu Anfang sehr irritierten. Als wir eine Ausstellung organisieren wollten, wurden wir ein lange Zeit im Archiv hingehalten. Später erfuhren wir Praktikanten, dass ohne Geld in Moldawien sich nichts bewegt. Die Deutsche Botschaft wie auch alle westlichen Botschaften sind aber natürlich an bestimmte Regeln gebunden, sodass man sehr oft persönliche Kontakte zu führenden Ministerien nutzen musste.

Gleichzeitig war es für mich sehr befremdlich, zu erfahren, dass man in moldauischen Krankenhäusern den Ärzten diskret Geld geben musste, um eine eventuelle Operation zu bekommen oder zu überstehen. In wichtigen Position sitzen auch meistens Verwandte von führenden Politiker.
\subsection*{Reflektion}
\addcontentsline{toc}{subsection}{Reflektion}

In Moldawien hatte ich im Gegensatz zu Russland nicht auf mir vertraute kulturelle Aspekte spekuliert. Dies war wohl der Grund, wieso ich mich trotz der Unterschiede, die meine Arbeit sehr beeinträchtigt haben, nicht unwohl gefühlt habe. Dass ich die anfänglichen negativen Erfahrungen nicht übergeneralisiert habe, lag sehr an den moldauischen Mitarbeitern in der Botschaft. Nicht dass sie deutsch geworden sind, sondern sie hatten einfach feste und gutbezahlte Jobs, was in Moldawien sehr selten ist. In Behörden oder Krankenhäusern ist man aber wegen der niedrigen Gehälter auf “zusätzliche Einnahmen” angewiesen. So seltsam das für die erste Zeit war, später habe ich das mehr und mehr begriffen.

\closuresection{Schlussbetrachtung}

Für mich spiegeln die besprochenen Fälle drei unterschiedliche Situationen wider: Meine Erfahrung in Kasachstan fing negativ deswegen an, weil ich wohl wirklich dachte, ich würde dort nichts neues sehen. Ähnlich war ja auch die Annahme in Russland. Gleichzeitig beendete ich die Sommerschule in Kasachstan eher positiv. Ich denke, dies ist wohl darauf zurückführen, dass ich dort umgeben von anderen Studenten meine Erfahrungen austauschen konnte, aber auch vom DAAD vor Ort extra Kurse in kasachischer Kultur angeboten wurden. Im Vergleich zum Auslandssemester ist auch die Tatsache wichtig, dass der DAAD die Kosten der Sommerschule sehr großzügig beglichen hatte und kaum Resultate erwartete; für das Semester in St.\ Petersburg musste ich aber selbst nicht wenig investieren und außerdem für andere Stipendien einen ToR vorweisen. Wohl auch dieser Druck sorgte dafür, dass ich in St.\ Petersburg in der Abwehrphase geblieben bin, während ich mich in Kasachstan zumindest angepasst habe.

In Moldawien hatte ich keinen Druck und kam auch ohne Erwartung an, für die Situation gewappnet zu sein. Wohl deshalb habe ich auch die erste Zeit aufmerksamer auf die Kultur geschaut und habe mich relativ schnell angepasst. Später konnte ich meine Aufgaben in Zusammenarbeit mit Moldauern viel besser wahrnehmen, weil ich mehr auf die persönliche Ebene geachtet habe und die für mich kritischen Punkte ausgeklammert habe und andere Aspekte wie Professionalität relativiert habe. Dies habe ich beispielsweise in St.\ Petersburg gegenüber den Dozenten nicht getan.

Schließlich hilft es ungemein, einen interkulturellen Kurs zu absolvieren, da man seine Erfahrungen danach sehr viel besser analysieren kann. Vorher hätte ich den änfänglichen Fehler in Russland und Kasachstan nochmals gemacht, doch nun würde ich wissen, dass ich eben doch stärker von der deutschen Kultur geprägt bin, als ich gedacht hätte. 

Für eine erfolgreiche Erfahrung in einem fremden Land sind länderspezifische Kurse sehr relevant wie der DAAD in Kasachstan, aber ein etwas abstrakterer Kurs wie an der Uni Hamburg ist vielleicht sogar wichtiger, weil er einen mit einem Framework versorgt, um in jedem Land die richtigen Schritte zu tun und die falschen richtig zu deuten.
\end{document}
