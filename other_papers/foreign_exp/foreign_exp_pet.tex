\documentclass[a4paper]{scrartcl}
\usepackage[]{graphicx}
\usepackage{booktabs}
\usepackage[T1]{fontenc}
\usepackage[utf8]{inputenc}
\usepackage[ngerman]{babel}
\usepackage{caption}
\usepackage{pifont}
\usepackage{enumitem}
\usepackage{listings}
\usepackage{bibgerm}
\usepackage{url}
\usepackage{eurosym}

\begin{document}
\author{Igor Fischer}
\title{Erfahrungsbericht zum Auslandssemester in St. Petersburg}
\maketitle
\begin{description}

\item[Name] Igor Fischer
% \item[E-Mail-Adresse] igor.fischer@studium.uni-hamburg.de
\item[Fach] Slavistik
\item[Jahr/Semester] SS 2013
\item[Dauer] Ein Semester
\item[Land] Russland
\item[Parthochschule] Staatliche Universität St. Petersburg (SPbGU- Sankt Peterburgskij Gosudarstvennyj Universitet)
\item[Einverständnis] Ich bin damit einverstanden, dass mein Bericht und meine E-Mail-Adresse an andere Studierende, die ein ähnliches Vorhaben planen, weitergegeben wird.

\end{description}
 

\section*{Vorbereitung und Anreise}
Die eigentliche Vorbereitung für das Auslandssemester begann bereits 1,5 Jahr vor der Reise. Es galt die Papiere  zu sammeln für die Bewerbung. Nach dem Einreichen wusste man auch relativ schnell Bescheid, dass man ausgewählt wurde. Danach konnte man sich mit der Sammlung weiterer Unterlagen für die Gastuni beschäftigen.
Für die Erstellung des vorläufigen Studiumplanes kann man die Internetseite der Uni besuchen, auch wenn man dort nicht immer die nötige Information findet, denn das meiste ist in Russland immer noch auf Aushängen notiert und so eigentlich nicht zugänglich.
Man bekommt ca. 2 Monate vor der Reise von der Uni die notwendigen Angaben zu Unterkunft sowie auch zur Visa-Beantragung. Es dauert dann noch eine Zeit bis man die notwendigen Papiere vom russischen Außenministerium bekommt, um das Visa zu beantragen. In meinem Fall konnte ich erst 20 Tage vor meiner geplanten Abreise zum Konsulat gehen. Ich habe dann aber auch innerhalb von 10 Tagen mein Visa bekommen.

Zum Flug muss man nicht viel sagen, außer dass man sich unbedingt bei Abt. Internationales erkundigen sollte, ob es die Möglichkeit einer Erstattung des Fluges gibt. Dann sollte man auch die Flugpapiere aufbewahren und nach seiner Rückkehr einreichen. Flüge gibt es genug aus Deutschland, Direktflüge aus Hamburg werden u.a. durch Rossiya Airlines geleistet.
Am Flughafen kann man ohne Probleme zuerst den Bus zur Metro-Station Moskovskaya nehmen und dann mit der Metro zur Station Primorskaya fahren, wenn man denn im Wohnheim an der Straße Kapitanskaya leben wird. Die Uni in St. Petersburg hat einen guten Hilfsguide, den man vor seiner Reise bekommt. Da sind die Anfahrtsinformationen detaillierter beschrieben.
\section*{Unterbringung und Verpflegung}
Untergebracht war ich im Wohnheim Kapitanskaya 3. Dort werden generell die meisten ausländischen Studierenden der SPbGU untergebracht sowie auch Russen, wobei die in der Minderzahl sind. Mit diesem Wohnheim hatte ich durchaus gute Erfahrungen gemacht. Es wurde regelmäßig geputzt, Probleme wie kaputte Steckdose oder Glühbirnen wurde zumeist recht schnell gelöst. Die Lage ist ebenfalls sehr gut. Es gibt viele Geschäfte in der Nähe, einen großen Markt, die Metro ist in 10 Minuten zu Fuß zu erreichen; es fahren ansonsten alle 5 Minuten Busse/Sammeltaxis zur Metro. Die Umgebung ist schön für Spaziergänge.

Im Zimmer lebt man entweder mit einem Studenten (2er-Zimmer) oder mit 2 Studenten (3er-Zimmer) zusammen. Darauf hat man keinen Einfluss; bezüglich der Nationalität der Zimmerbewohner wurde ich gefragt, ob ich eher mit Russen oder mit Ausländern leben will. Für mich war es egal und ich denke, dass es keinen großen Unterschied macht. Ich kam in ein Zimmer mit einem Mongolen und einem Taiwaner, deren Russisch nicht schlecht war, sodass man sich mit ihnen auch auf Russisch unterhalten konnte. Auf jeden Fall hatte ich das Glück von den beiden viel über ihre Kultur zu lernen.

In der Wohnung gab es also zwei 2er-Zimmer und ein 3er-Zimmer. Ein 2er-Zimmer wurde nur von zwei Russen bewohnt, das andere von einem Chinesen und einem Polen. Alle Mitbewohner waren angenehm im Umgang und in Bezug auf die Russen wurden alle meine Vorurteile im Laufe der 5 Monate zerstreut.
In der Wohnung hat man eine Küche mit einem Kühlschrank und einem Herd -  und eigentlich mit Ofen, der aber in manchen Wohnungen ausgeschaltet ist. Auf jeden Fall kann man ohne Probleme Essen zubereiten und Produkte lagern.
Die Qualität der Verpflegung ist daher nur von einem selbst abhängig, denn die Ausstattung der Küche erlaubt schon viel (eine Mikrowelle fehlt allerdings). Man sollte aber zu Anfang seines Aufenthalts 20-30 \euro\ in ein bisschen Geschirr und Pfanne und Topf investieren, denn die vorhandenen sind  nicht unbedingt die saubersten.
\section*{Kosten}
Ein großer Vorteil bei meinem Austauschprogramm war, dass die Uni und das Wohnheim umsonst waren. So wie ich glaube, ist das generell beim Vertrag zwischen der Uni Hamburg und der SPbGU so. Es spart die 170\euro\  Unterbringung und die bis zu 3000\euro\ Studiengebühren. Auch der Flug kann rückerstattet werden. Das Visa ist ebenfalls kostenlos. 
Es bleiben dann nicht mehr so viele Kosten. Man muss mit ca. 30-40\euro\ Verpflegungskosten in der Woche rechnen, wenn man häufig im Wohnheim kocht. Auswärts essen würde natürlich mehr kosten. Preis der Produkte ist häufig nicht niedriger als in Deutschland, da viele Produkte importiert sind.
Ein weiterer regelmäßiger Kostenfaktor ist die studentische Fahrkarte. Für sie muss man jeden Monat 20\euro\ ausgeben und kann damit die Metro und die Busse verwenden. Für die Freizeit muss man nicht viel Geld ausgeben, da man als Student bei allen Museen Ermäßigungen, häufig sogar freien Eintritt hat. Auch Reisen nach Peterhof und Zarskoe Selo sind nicht teuer. Im Monat kann man für seine Freizeit sicherlich 20-30\euro\ einplanen ohne Diskotheken und Bars. Insgesamt kann man bei einem ruhigen Lebensstil mit monatlichen Kosten von 200-250\euro\ rechnen.

Es gibt die Möglichkeit, ein Zimmer in der Stadt zu mieten. Dafür muss man neben der Mühe des Suchens auch einen Preis von 300-400\euro\ pro Monat veranschlagen. Manche Kommilitonen aus anderen Ländern nutzten diese Möglichkeit, wohl weil für sie das Wohnheim nicht kostenlos war.
Im Übrigen gibt es für Hamburger Studenten ein überdurchschnittliches Stipendium von immerhin 60\euro\ pro Monat - das ist doppelt so viel wie für polnische Studenten. 
\section*{Gastuniversität}
Die Staatliche Universität St. Petersburg ist auf dem Papier die zweitbeste Universität in Russland. Das jährliche Budget ähnelt der Hamburger Uni: ca. 300 Mio Euro. Es studieren dort nur ca. 30.000 Studenten. Eine Umstellung ist, dass es keinen richtigen Campus gibt, sondern eher nur über die ganze Stadt verstreute Fakultäten. Es gibt eine Menge Kurse zur Auswahl und viele Annehmlichkeiten für ausländische Studenten. Die Lehrenden sind überaus erfahren. Aus meinen Gesprächen mit den Wohnheim-Bewohnern kann ich sagen, dass die politische und die wirtschaftliche Fakultät gut aufgestellt sind. Toll ist das Kulturprogramm  der SPbGU: Fast jede Woche gibt es Konzerte mit Sängern und Musikern im Festsaal der Uni. Dabei wird man auf die 300 Jahre lange Tradition der Uni aufmerksam.

\section*{Alltag und Freizeit}
Petersburg als Kulturhauptstadt bietet sehr viele Freizeitmöglichkeiten. Es gibt viel außerhalb zu besichtigen: Novgorod, Vyborg, Helsinki sowie die Residenzen der Zaren Peterhof und Zarskoe Selo. Da sollte man in jedem Fall hinfahren. Die Museen in Petersburg sind fast alle sehr gut ausgestattet. Nett ist die Nacht der Museen im Mai. Vor allem im Frühjahrssemester hat man in Petersburg Glück, da viele Feste stattfinden: Uni-Geburtstag, Ostern, 9.Mai, Stadtgeburtstag, Abschlussfest für Studenten usw. Viele Diskotheken und Bars befinden sich auf dem Festland, sodass man aufpassen muss, wie man zurückkommt auf die Vassili-Insel zum Wohnheim, wenn man denn nicht die ganze Nacht auf dem Festland verbringen.
\section*{Fazit}
Das Auslandssemester in St. Petersburg hat mich auf jeden Fall vorwärts gebracht. Die Erfahrung im Wohnheim hat meine interkulturelle Kompetenz in jedem Fall gesteigert. Die dortigen Mitbewohner waren eine fantastische Gesellschaft. Die Ideen aus einigen Uni-Kursen werden mich noch eine Weile beschäftigen. Ich konnte während meines Semesters die russische Mentalität zur Genüge kennenlernen. Meine Russisch-Kenntnisse haben sich merklich verbessert. Als Abenteuer war das Auslandssemester wegen der vielen Herausforderungen großartig. Die Hilfe der Universität Hamburg hat mir das Auslandssemester ermöglicht und mich motiviert, das Beste aus dem Semester rauszuholen.
\end{document}
