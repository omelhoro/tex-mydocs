\documentclass{../../sem_paper}

\begin{document}
\titlepg
{Praktikum bei der Deutschen Botschaft in Moldawien}
{2013}
{Arbeitsstelle Studium und Beruf}
{Aufbaumodul 50-043 Berufs- und Bewerbungspraxis}
{WS 2010/2011}
{Günther Domes}

\tocpaper

\section{Vor dem Praktikum}

\subsection{Anforderungen ans Praktikum}
Nach sechs Semestern Uni habe ich sehr viel Theorie gelernt und Fähigkeiten angesammelt. Es war für mich wichtig, zu sehen, ob die bisher erworbenen Kenntnisse einen Wert außerhalb der akademischen Welt haben. Nicht nur über meine fachlichen Aspekte war ich mir im Unklaren, sondern auch über die sozialen. Ich war gespannt, wie die Soft-Skills gebraucht werden.
Über genaue Tätigkeiten habe ich mir keine Gedanken gemacht, wohl eher über allgemeinere Tendenzen nachgedacht. Im Praktikum sollten meine Russisch-Kenntnisse zur Verwendung kommen sowie meine Kenntnisse Osteuropas allgemein. Mein kultureller Studienschwerpunkt sollte auch eine Rolle spielen. Mit diesen Ideen war der Grundriss eines Praktikums von mir gezeichnet worden.

Bei der ersten Gedanken über das Praktikum fiel mir eigentlich als erste Idee das Auswärtige Amt ein. Es war gewissermaßen mein Traum seit meiner Schulzeit. Es vereinte für mich vieles von dem, was ich machen wollte. Angesichts einer gewissen Idealisierung habe ich mir von dem Praktikum versprochen, ein echtes Bild vom AA zu gewinnen. Ich wollte lernen, wie man Deutschland repräsentiert und die Pflichten eines Diplomaten kennenlernen. Vor allem habe ich mir vom Praktikum versprochen, dass meine o.g. Erwartungen dort verwirklicht werden. 

\subsection{Die Bewerbung beim Auswärtigen Amt}
Das Auswärtige Amt bietet Studenten aller Fachrichtungen an, ein Praktikum an einer Botschaft oder in der Zentrale in Berlin zu machen. Für mich kam eigentlich nur ein Praktikum im Ausland in Frage, da ich gerade an den Pflichten eines Diplomaten im \emph{Ausland} interessiert war. Die Bewerbungen für alle Möglichkeiten laufen elektronisch ab. Auf der Website des AA muss man einige Unterlagen hochladen, ein Motivationsschreiben machen, seine Sprachkenntnisse nachweisen usw. Da es kein persönliches Gespräch gibt, muss man auf der Website also versuchen, sich möglichst kurz und knapp vorzustellen. In der Bewerbung kann man ebenfalls bis zu 9 Orte angeben, an denen man eingesetzt werden möchte. Ein Ort reicht dabei aus, mehrere erhöhen die Chance. Am nächsten Tag nach dem Absenden der Bewerbung hatte ich bereits ein Angebot der Botschaft in Chisinau (Moldawien) vorliegen, welches ich dann nach ein paar Tagen auch angenommen habe. Ich habe kurz auf andere Angebote gewartet, da ich ebenfalls Stationen in Zentralasien und Kaukasus angegeben habe, wo ich eher als Moldawien hin wollte. Damit hatte ich überraschenderweise gleich einige Tage nach dem Einreichen der Unterlagen ein Praktikum bekommen. Alles in Allem habe ich vom anfänglichen Sammeln der Bewerbungsdokumente bis zur Annahme des Angebotes ca. 2 Wochen gebraucht. Das alles fand 9 Monate vor dem Praktikumsbeginn im Oktober 2012 statt.

Mein Praktikum fand in der Deutschen Botschaft in Chişinău, der Hauptstadt Moldawiens, statt. Insgesamt hat die Botschaft \_ Beschäftigte, davon sind \_ deutsche Entsandte sowie \_ Bundespolizisten und der Rest waren Ortskräfte. Die Entsandten haben leitende Funktion; sie beaufsichtigen die Ortskräfte in der Visastelle und bei den Kulturprojekten.

\subsection{Vorbereitung des Praktikums}
Ca. zwei Monate vor meinem Praktikumsantritt habe ich einen Leitfaden bekommen, wo alles Notwendige stand. Ich konnte mir also schon im Voraus ein Bild machen von dem, was mich dort erwarten könnte. Danach konnte ich meine Russisch-Kenntnisse verwenden, um einen Raum zu reservieren. Leben konnte ich ungefähr 15 Minuten von der Botschaft in einer Pension. Ansonsten habe ich keine weitere Vorbereitung getroffen. Wie ich danach herausfand, kann sich man auch während der ersten Wochen des Praktikums einarbeiten. Rumänisch(oder Moldauisch) braucht man für die kurzen Zeiten des Praktikums nicht zu lernen. Die nötigen Floskeln lernt man schon während des Praktikums. Alles in allem habe ich für das Praktikum ein Budget von 800\euro\ geplant, das ich einhalten konnte. Das Praktikum war unbezahlt, daher wurde ich zum Teil von meinen Eltern unterstützt.

\newpage
\section{Die Zeit während des Praktikums}

\subsection{Der Diplomatenberuf}
Als Diplomat hat man es nicht leicht. Allein schon Diplomat zu werden ist aber auch sehr schwer. Es werden hohe Anforderungen gestellt, an Physis und Psyche. Mit der Arbeit im AA ist immer der Wechsel von Arbeitsorten verbunden, in der Regel alle 3-4 Jahre. Dabei hat man natürlich einen Einfluss, wohin man geschickt wird. Prinzipiell muss man aber dem AA zusagen, überall auf der Welt eingesetzt werden zu können. Nach Aussagen der Entsandten ist es auch kein 9-17 Job, sondern auch immer mit Überstunden verbunden, sodass man auch mal bis in den späten Abend reinarbeitet. Die Entsandten müssen in regelmäßigen Abständen die Pflichten der Bereitschaft auf sich nehmen und so Notfälle am Wochenende bearbeiten.

Schließlich ist eine Botschaft auch ein großes Team und man kann Probleme im Team lösen können. Man muss das richtige Feingefühl gegenüber den Ortskräften und der Mentalität vor Ort haben.

Eine wichtige Qualität ist Vielseitigkeit und Interesse an Aufgaben anderer. Denn es kommt vor, dass jemand im Urlaub ist, sodass man die Pflichten mit übernehmen muss. Das kam während meines Praktikums recht häufig vor. So musste die Person aus der Zahlstelle während einiger Tage in der Visastelle aushelfen. Das Kulturreferat musste die Arbeit der Wirtschaft machen, weil sich da durch die angesprochene Rotation Vakanzen ergeben haben.

\subsection{Leben in Moldawien}
Moldawien war für mich ein neues Land. Ich habe bis zu meiner Bewerbung beim Auswärtigen Amt nichts davon gehört. Das war sicherlich eine Motivation für die Entscheidung dort mein Praktikum zu machen. Obgleich ich etwas Sorgen hatte, da das Land rein von Zahlen wie BIP pro Kopf auf Ebene mancher afrikanischer Staaten steht, habe ich das Land als sehr europäisch wahrgenommen. Interessant aus diplomatischer Sicht war, dass Moldawien keine Entscheidung über seine Zukunft getroffen hatte. Die beiden Möglichkeiten, EU und Euroasiatische Gemeinschaft (Russland, Kasachstan, Weißrussland), waren ungefähr gleich in der Beliebtheit. Viele ältere Moldauer haben immer noch eine gewisse Tendenz zu Russland, während die Jüngeren sich mehr nach Westeuropa orientieren. Andererseits wird Moldawien von allen Seiten bedrängt, sich für einen Block zu entscheiden, was wegen der ökonomischen Verflechtungen zu beiden Blöcken nur Nachteile für das Land bringen würde. Durch den weiteren großen Spieler, die USA, wird Moldawien wie ein Spielball größerer Mächte behandelt, diese Analysen wurden mir von mehreren Moldauer vorgetragen.

Weil Moldawien gewissermaßen an der Schnittstelle zwischen Ost und West liegt, gibt es dort auch viele Vertretungen von Institutionen. So hat die EU hochrangige Offizielle vor Ort, die NATO hat ein Büro, die OSZE hat eine Mission in der Hauptstadt und im abtrünnigen Teil Transnistrien. Von deutscher Seite ist die Konrad-Adenauer-Stiftung, Friedrich-Ebert-Stiftung und auch die Bosch-Stiftung vertreten. Die Letztere finanziert zwei Boschlektoren, die an den Hauptunis Deutsch unterrichten.

Der Mentalität in Moldawien war für mich gewöhnungsbedürftig. Ich habe schon bei einem Auslandssemester in Russland erlebt, was es bedeutet sein Leben im Ausland zu organisieren; für Moldawien war das eine große Hilfe. Ich erlebte dort eine größere Dynamik als in Deutschland, aber alles ging dort auch chaotischer zu. Oftmals war die Verlässlichkeit auf Seiten der Moldauer nicht gegeben. Nicht selten mussten wir in der Botschaft manche Papiere dreimal von den Ministerien anfordern bis man uns sie geschickt hat. Ich habe oft bemerkt, dass in Moldawien Demokratie herrscht. Man konnte ohne Probleme über Politik reden und seine politischen Ansichten äußern. Es gab mehrere Parteien, die miteinander kooperierten. 

Allerdings wurde mir gewahr, dass die Korruption ein großes Problem in Moldawien ist. Ich war auch überrascht, in wie vielen Lebenslagen bestochen wird. Ich konnte in die Bemühungen der OSZE blicken, Korruption durch Aufklärung zu mindern, habe aber einige resignierende Stimmen vernommen, die sagten, dass Korruption in solchen Ländern viel zu sehr im sozialen System verwurzelt ist, als dass man viel dagegen tun könnte. Ich selbst hatte keine Fälle der Korruption miterlebt und kann mir gar nicht vorstellen, dass man einem Arzt zusätzlich Geld bezahlen muss, dass man überhaupt operiert wird und das Ergebnis von der Höhe der Summe abhängt. Es war interessant zu hören, dass von den Einwohnern solche Zahlungen nicht als Korruption gesehen werden, sondern eher als Aufpreis oder als eine Art Trinkgeld.

\subsection{Anforderungen und Aufgaben}
Ich habe während meines Praktikums nicht wirklich erfahren, wieso man mich ausgewählt hat. An mich wurden auch zu Anfang des Praktikums keine genauen Anforderungen gestellt. Ich glaube, dass man mich und meine Fähigkeiten erst einmal kennenlernen wollte. Dann habe ich gemerkt, dass ich natürlich meine Russisch-Kenntnisse verwenden könnte. Gewisse andere Anforderungen unterschieden sich nicht von anderen Berufen: Sorgfalt, Aufmerksamkeit, Eigeninitiative. Besonders für letzteres gab es genug Raum. Die meisten Anforderungen konnte ich mit einfachen Kenntnissen der Computerprogramme lösen. Für andere musste ich mich an die Umgebung anpassen. Damit meine ich vor allem, von meinem Kollegen lernen.

Es ist schwer zu sagen, welche Schlüsselqualifikationen ich gebrauchen konnte, da ich nach 6 Semestern Uni immer noch nicht richtig verstanden habe, was das bedeutet. Mein in der Uni erworbenes Fachwissen konnte ich nur bedingt einsetzen. Ich kann nur sagen, dass es sich gut macht, wenn man im Diplomatenberuf sich so verhält, wie man sich wohl fühlt.

In meinem sechs Wochen in der Botschaft hatte ich mit mehr Aufgaben zu tun als ich gedacht hätte. Die ersten drei Wochen hatte ich aber zugegebenermaßen ein bisschen Leerlauf. Das kam daher, dass noch ein anderer Praktikant da war, der die Aufgaben bekommen hat. Der Botschafter war außerdem im Urlaub und der Stellvertreter zog gerade wieder nach Berlin um. Ich habe diese Zeit genutzt, um mich in die politische Situation einlesen. Bereits in der 3. Woche gab es viele Aufgaben wie Repräsentation der Botschaft auf einigen Ereignissen, Erstellen von Protokollen für Gespräche und einige Rechercheaufgaben. In den verbleibenden Wochen war ich sehr beschäftigt mit organisatorischen Dingen und Analyseaufgaben. Die zweite Hälfte meines Praktikums ließ nichts zu wünschen übrig. Ich konnte sehr viele meiner Fähigkeiten einsetzen und wurde dazu von der Botschaft auch ermutigt.

Um konkreter zu werden: Eine Beschäftigung, die sich durch mein Praktikum zog, war die Beaufsichtigung des Prozesses gegen einen Deutschen in Transnistrien. Da ich Russisch konnte, erstellte ich nach den Sitzungen Vermerke zu den Vorkommnissen. Danach musste man auch häufig mit Verwandten reden. Die Aufgabe der Beaufsichtigung des Prozesses nahm die Deutsche Botschaft zusammen mit der OSZE war. Insgesamt konnte man spüren, dass die Anwesenheit der Deutschen Botschaft einen positiven Einfluss auf den Fortgang des Prozesses hatte.

Besonders in der letzten Woche war ich mit der Erstellung eines Posters zu den Kleinstprojekten der Deutschen Botschaft beschäftigt. Dem ging eine Woche Informationssammlung voraus. Dafür musste ich ins Archiv und auch bei der Zentrale in Berlin Informationen einholen. Als nächsten Schritt musste ich die Informationen in Ordnung bringen, da nicht immer die richtigen Orte oder die Schreibweise vorhanden waren. Als letzten Schritt habe ich daraus ein Poster gemacht. Ich hätte gerne noch gesehen, dass es vielleicht auch der Öffentlichkeit vorgestellt wird, aber dann war meine Zeit auch schon vorbei. Auf jeden Fall war es auch für die Mitarbeiter überraschend, dass es seit dem Jahr 2004 95 Projekte gab. Mittlerweile soll das Poster auch als Handout den Journalisten zur Verfügung gestellt werden.

Einige kleinere Aufgaben beschäftigten mich für 1-2 Tage. Als Beispiel musste ich einen Vermerk über ein Gespräch mit dem Kulturminister erstellen. Dies war sehr wichtig, da auch einige andere Ministerien der deutschen Seite beteiligt waren. Der Vermerk wurde dann nach meinem ersten Entwurf bei den Beteiligten nochmals verfeinert.
Eine schöne Aufgabe war die Kontrolle eines Kleinstprojektes. Diese Projekte unterstützen Antragssteller mit durchschnittlich 7.000\euro , ca. 10 Projekte gibt es im Jahr. Ich habe mich sehr viel mit der Geschichte dieser Hilfe beschäftigt, hatte aber auch einmal das Glück, bei einer Kontrollfahrt in ein Dorf dabei zu sein. Dies muss man machen, um die vereinbarte Verwendung der Gelder zu beaufsichtigen. Dort wurden wir freundlich empfangen und wir sahen, dass das Geld sehr effektiv verwendet wurde, um ein Altersheim zu renovieren. Das Dorf dankte uns dafür. Dies war eine sehr schöne Erfahrung.

Eine eher reguläre Aufgabe war die Aktualisierung des Standes der Projekte. Dabei musste ich in Besprechungen dabei sein und alles zum Fortschritt notieren. Die Notizen musste ich in eine Tabelle eintragen. In diese Richtung ging auch die Aufgabe, unser Material zu den Projekten zu kategorisieren. Dies betraf insbesondere das Bessarabienprojekt. Da hatten wir viel Material gesammelt für das 200. Jubiläum der Ansiedlung der Deutschen am Schwarzen Meer und dies musste in eine Übersicht gebracht werden.

Es gab während meiner Zeit auch typischere Praktikantenaufgaben wie Kopieren, Zahlen ausrechnen, Drucken, Mappen ordnen und Emails mit Anhängen senden. Dies war aber überraschenderweise selten der Fall.

Insgesamt empfand ich die Aufgaben als vielfältig und manche als fordernd. Prinzipiell wurde auch meine Initiative bei Aufgabenfindung begrüßt und so habe ich in Zeiten, in denen mir die Kollegen keine Aufgaben gaben, mir selber was gesucht. Durch den Zugang zu den Laufwerken der Botschaft hatte ich durchaus das Gefühl, dass ich eingebunden bin.

\subsection{Unterstützung durch Kollegen}
Die Leute in der Botschaft haben auf jeden Fall versucht, mich in die Projekte einzubinden und mir die Aufgaben und  den Kontext so gut wie möglich zu erklären. Das war eine große Hilfe für mich. Ich hatte aber auch Glück, dass noch ein Praktikant da war und mir helfen konnte. Bei Fragen waren die Kollegen sehr freundlich und geduldig. Es gab von ihrer Seite viele Angebote, zu Veranstaltungen mitzukommen. Ansonsten konnte man in der Botschaft auch gut arbeiten, weil der Praktikant immer ein eigenes Büro bekommen hat. Es war zwar ein kleines Büro mit der Maße 2x3 Meter, hatte aber den Vorteil, dass es direkt am Büro des Stellvertreters lag und auch nicht weit vom Büro des Botschafters war. Von diesen beiden Personen hat man die meisten Aufgaben bekommen.
Ansonsten hat jede Botschaft einen Beauftragten für die Praktikanten. Diese Person ist meistens die Sekretärin des Botschafters und kann einem auch sehr helfen. 

Durch die zuvorkommende Art der Kollegen gab es eigentlich gar keine Schwierigkeiten mit ihnen oder den Aufgaben. Die Aufgaben konnte man allerdings nicht alle in Botschaft lösen, sodass ich manchen Abend und manches Wochenende durchaus in die Aufgaben investieren musste. Weil die Aufgaben auch spannend waren, habe ich das gerne getan. Fürs Lösen bestimmter Aufgaben habe ich häufig das Internet benutzt. So z.B. für Übersetzen oder Anleitung zum Benutzen eines Programms.

Bei manchen Aufgaben hätte ich durchaus Rumänisch-Kenntnisse gebraucht, ich kam aber auch sehr gut mit Russisch zurecht. Mit manchen Ortskräften in der Botschaft habe ich häufig auch nur Russisch geredet.

Die erwähnten Erfahrungen waren für mich ein tolle Abwechslung zum Studienalltag. Ich habe es sehr genossen, abseits von Büchern und Theorie, meine Fähigkeiten einzusetzen. Während ich früher die Pflicht zum Praktikum eher als lästig wahrgenommen habe, erkannte ich während des Praktikums, wie wichtig es ist. Es war ebenfalls ein tolles Gefühl inmitten von erfahrenen Leuten zu sein, die für mich als Vorbilder in Erinnerung bleiben werden. Ich war auch der Jüngste in der Botschaft, was ich etwas seltsam fand, da ich im Studienalltag anderes gewöhnt bin. Während meines Praktikums wurde mir gewahr, dass die Uni einem nur Sachen beibringt, die man auch nur in der Uni gebrauchen. Insofern war es ein schönes Gefühl, als Praktikant Dinge zu lernen, die man eher im Berufsleben verwenden kann.

Es war allgemein ein wunderbares Gefühl, zumindest für kurze Zeit die Privilegien eines Diplomaten zu genießen. Ich hatte Zutritt zu vielen internationalen Organisationen wie NATO, UNO und Weltbank. In einem Diplomatenfahrzeug  waren Grenzübergänge keine lange Angelegenheit.  

\newpage
\section{Resümierende Gedanken zum Praktikum}

\subsection{Erfüllung der Erwartungen}
Das Praktikum hat meine Erwartungen erfüllt. Nach einer gewissen Eingewöhnungszeit konnte ich die Zeit sehr gut nutzen und meine erste Enttäuschung über den Leerlauf war nicht mehr von Bedeutung. Ich konnte tatsächlich sehen, wo ich Defizite habe und was ich bereits kann. Es wurde mir durch das Praktikum klar, dass ich keinen Master machen werde und auch generell nach dem Bachelor mein bisheriges Studienfach nicht weiter verfolgen werde. Andererseits hat mir das AA so gefallen, dass ich mich auf jeden Fall dort bewerben werde.

Das Praktikum hat offen gezeigt, dass trotz der rigiden Auswahl der Diplomaten, es doch alles nur Menschen sind, deren Level ich auch erreichen könnte. Das Diplomatenleben, das wurde mir klar, ist nicht so einfach. Besonders in Bezug auf die Familie ist das AA eine große Belastung. Auch in der Hinsicht war das Praktikum ein Gewinn. Ich habe auch die negativen Seiten des Diplomatenlebens erzählt bekommen.

Durch die neuen Bekanntschaften im AA und durch ein wunderbares Praktikumszeugnis wurde durchaus ermutigt, mich später beim AA zu bewerben. Ich konnte sehr viele meiner Fähigkeiten verwenden und war am Ende von der Vielfalt der Aufgaben und der Vielseitigkeit meiner Kollegen beeindruckt. Die Erzählungen über den Bewerbungsablauf haben mir die Angst vor dem selektiven Bewerbungsprozess genommen. Ob es nun mit dem AA klappt oder nicht, durch das gute Praktikumszeugnis habe ich eine gute Chance auf viele Berufe. Die Erfahrung während der sechs Wochen in der Botschaft wird mich auch weit bringen.

\subsection{Lehren aus dem Praktikum}
Das Praktikum hat mir alles in allem sehr gut gefallen. Die Mitarbeiter waren sehr freundlich und versuchen mich in die Projekte miteinzubeziehen. Es war gut, dass ich viele Aufgaben kennenlernen konnte und von den Mitarbeitern auch zu vielen Veranstaltungen mitgenommen wurde. Man hat mir einige Male das Vertrauen geschenkt, auch alleine auf Veranstaltungen Deutschland zu repräsentieren. Nicht so gut war der Leerlauf der ersten Wochen. Ich hätte sicherlich auch effektiver sein können, wenn man in der Botschaft bessere Technik gehabt hätte. So musste ich sehr oft zu Hause arbeiten. Auf gewisse negative Dinge hatte ich keinen Einfluss, so z.B., dass die Sommerferien dazu führten, dass nicht alle Mitarbeiter in der Botschaft waren, manche waren im Urlaub. Die Politiker Moldawiens waren auch zum größten Teil im Urlaub, sodass nicht so viele Besuche oder Konferenzen stattfanden. Mit meiner Abreise fing auch gerade die Geschäftigkeit an anzusteigen.

Ich bin aber trotz allem begeistert vom Praktikum, weil ich die Möglichkeit hatte, in eine große Organisation wie das AA reinzublicken. Es war eine tolle Erfahrung zu sehen, wie man innerhalb der Botschaft Probleme anging. Ich sah, dass hinter einer Organisation wie einer Botschaft sehr viel Arbeit steckt. Es lässt sich vieles planen, aber man muss dann doch die Flexibilität haben, um auf kurzfristige Herausforderungen zu reagieren. Schlussendlich habe ich gelernt, dass Fachwissen wertvoll ist, aber der richtige Umgang mit Kollegen ist dann doch noch wichtiger, da man die meisten Projekte nur im Team lösen kann. Eine gute Arbeitsatmosphäre schafft auch die Grundlage für gute Resultate.

\subsection{Künftige Herausforderungen}
Ich werde zunächst keine Kontakte unmittelbar verwenden können, da die Botschaft ja keine Diplomaten einstellen kann. Sofern es mir gelingt, beim AA unterzukommen, so kann ich mir auf jeden Fall vorstellen, Kontakt zu meinen ehemaligen Kollegen aufzunehmen. Es ist aber klar, dass sie wegen der Rotation an ganz anderen Orten sein werden. In Chisinau habe ich außerhalb der Botschaft durchaus einige Bekanntschaften geschlossen, die ich bei künftigen Besuchen auch treffen werde. Von wertvollen Kontakten kann ich aber trotz allem nicht reden, da der Einfluss der Botschaft natürlich außerhalb der Landesgrenzen gering ist. Das Praktikum ist ein gutes Argument für eine spätere Arbeit beim AA, ist aber auch keine Garantie, dass man übernommen wird.

Mein Studium hat mir bisher Spaß gemacht und mein Selbstbewusstsein gestärkt. Wichtig ist da vor allem das Beibringen der Sprache und Einblick in analytische Fähigkeiten. Dass ich das Fachwissen meines Studiengangs Slavistik verwenden kann - darüber mache ich mir keine Illusionen. Ich konnte fast gar nichts dessen, was ich in sechs Semestern gelernt habe, verwenden. Nur die Sprache Russisch gebrauchte ich jeden Tag. Was dann also nach dem Studium für die Arbeit bleibt, sind die Sprachkenntnisse und andere passive Fähigkeiten wie analysieren und vorstellen. Vielleicht kann ich noch was aus meinem Kulturwissen gebrauchen, wenn ich einmal in Russland eingesetzt werde.

Wichtig ist allerdings die Komponente interkulturelle Kommunikation. Ich habe nun gelernt, dass es durchaus andere Mentalitäten gibt. Diese zu verstehen, hat mir mein Studium erleichtert. Dieser Aspekt ist kritisch für Diplomaten und ich glaube, dass mit meinem Studium einen Vorteil habe.

Aktiv konnte ich nur meine Sprachkenntnisse aus dem Studium verwenden. Ansonsten merkte ich nicht so richtig, dass ich anderes aus dem Studium verwendet habe. Ich denke aber, dass ich mein bisheriges Studium im Praktikum passiv eingesetzt habe, z.B. wenn es darum ging, etwas zu organisieren, verschiedene Seiten zu vereinigen und mit anderen Kulturen zu kommunizieren. Ich könnte kein Beispiel für diese Dinge nennen, aber ich glaube, dass sie passiv immer während des Praktikums anwesend waren.

Dafür konnte ich ein Menge selbst gelernter Sachen einsetzen aus dem Bereich EDV. Es war für mich ein großer Vorteil, dass ich einige Bildbearbeitungsprogramme kannte und auch Statistiken erstellen konnte. Das hat mir geholfen Daten auszuwerten und dann zu präsentieren. Hier liegt auch definitiv einer meiner Defizite. EDV ist sehr wichtig und es gibt immer Spielraum sich zu verbessern.

Im Weiteren werde ich unbedingt Französisch lernen müssen, da es in der Diplomatenwelt sehr wichtig ist. In allen Ländern gibt es eine starke Kooperation zwischen den französischen und deutschen Botschaften. Meine Englisch-Kenntnisse ließen sich auch verbessern. Ich fühle mich zz. nicht in der Lage eine ganze Konferenz auf Englisch zu führen. Da Diplomaten auch sehr viel diskutieren müssen, werde ich mich mit der Rhetorik beschäftigen müssen. Wichtig wäre es für mich ebenfalls, alle meine Sprachkenntnisse zu zertifizieren. Ein TOEFL-Test wäre wohl sehr wichtig bei der Bewerbung. Schlussendlich muss ich mir passende Antworten suchen zu Fragen über Familie und ob der Partner das schwierige Leben eines Diplomaten aushalten könnte.

\end{document}
