\documentclass{../../sem_paper}
\addbibresource{blackmonk.bib}

\begin{document}
\titlepg
{Der mentale Zustand A.V. Kovrins in der Erzählung
"`Der schwarze Mönch"' von A.P. Čechov}
{2011}
{Insitut für Slavistik}
{Seminar Ib zum Thema "`Die literarische Figur"'}
{WS 2010/2011}
{Daniela Chmelik}

\tocpaper

\closuresection{Einleitung}
Der Wahnsinn hat viele Gesichter und viele Ausdrucksmöglichkeiten. Das habe ich im Seminar
gelernt, wo wir Personen und Erzählungen mit ganz unterschiedlichen Typen von Wahnsinn
behandelt haben. Vor allen Dingen waren Halluzinationen das hauptsächliche Instrument bei der
subjektiven Umformung der Welt, aber auch eine gefühlsmäßige Anormalität. In „Der schwarze
Mönch“ liegt beides vor. Es drückt sich beides auf eine subtile Art und Weise aus, was die
Erzählung zu einem besonderen Werk in Čechovs Lebenswerk macht.
Trotzdem enthält die Erzählung typische Merkmale von Čechovs Schaffen. Besonders für seine
Prosa der mittleren Jahre sind durchaus geläufige formale Eigenschaften festzustellen, die eng mt
der Perspektive des Erzählers verbunden sind. So erscheint einmal die formale Distanz zu Kovrin
durch das reguläre Nennen seines Nachnamens, andererseits ist auch eine hohe Einfühlung zu
bemerken.\autocite[76]{hielscher} Weiterhin ist ein aus anderen Werken bekannter Kunstgriff der „inneren indirekten
Rede“ präsent, bei dem fremdes Bewusstsein wiedergegeben wird, aber andererseits eine
Identifikation des Lesers mit diesem erschwert wird.\autocite[81]{hielscher} Auch ist im „Schwarzen Mönch“ das häufig
bei Čechov verwendete Motiv des Lebenssinns – auch „allgemeine Idee“ genannt – vordringlich.
Was bei Čechov damit einhergeht, ist die unklare Antwort auf die Natur dieser Idee. Auch der
schwarze Mönch kann im Grunde genommen keine Antworten liefern, noch weniger das Ende.\autocite{freise97}\autocite{sonnleitner} Es
ist wohl auch zu verneinen, dass Čechov irgendetwas beantworten wollte.\autocite{winner}
Die Erzählung zog trotz der aufgezählten Gemeinsamkeiten wie kaum ein anderes Werk Čechovs
das Interesse der Literaturwissenschaftler und Kritiker auf sich. Bereits nach dem Erscheinen im
Jahr 1894 gab es zahlreiche Standpunkte, was dieses Werk aussagen sollte. In der sowjetischen
Forschung dominierte später ein ideologischer Einfluss.\footnote{Rezeptionsgeschichte in \autocite{meve} und \autocite{kataev}} In der neueren Forschung wurden mit
Erfolg auch experimentelle Ansätze ausprobiert.\footnote{z.B. Anwendung der Archetypenlehre in \autocite{woern} oder Podtext-Fabel
in \autocite{freise97}
} Die Schwierigkeit kommt meines Erachtens auch
aus der Vielfalt an Aspekten in der Erzählung.\autocite[108]{kluge}
Čechov hat viele Motive anderen seiner Werke
entnommen. Einige davon sind: „Das Duell“, „Eine langweilige Geschichte“, „Palata No.6“ oder
auch kürzere Geschichten wie „Student“.\autocite{woern} Tendenziell flossen auch zeitgenössische Ideen ein.\autocite{dudek} Die
Figur des schwarzen Mönchs hatte zu der Zeit ebenfalls eine Rolle in anderen literarischen Werken
gespielt.\autocite{winner} Nicht zuletzt wurden auch die Ähnlichkeiten zu Dostojewskis „Doppelgänger“\autocite{woell}, Garšins
„ Rote Blume“\autocite{freise97} und Heldensagen\autocite{senderovich} hervorgehoben.
Es ist kein Geheimnis, dass „Der schwarze Mönch“ eine Krankheitsgeschichte ist. Čechov selbst
gab das in einem Brief zu: \textcyr{Черного монаха я писал без всяких унылых мыслей, по холодному
размышлении. Просто пришла охота изобразить манию величия.} Und \textcyr{Это рассказ
медицинский -- historia morbi}.\footnote{ aus
Polnoe Sobranie Sočinenij i Pisem. Tom 8. (1892-1894), Moskva 1977}

In meiner Hausarbeit interessiert mich die Frage, inwiefern es auch aus der heutigen Zeit eine
historia morbi ist. Dazu will ich im ersten Teil den Größenwahn mit seinen in der heutigen Zeit
erforschten Merkmalen auf Kovrin übertragen. Es gibt einige den Wahnsinn begünstigende
Faktoren, die sich beispielsweise der Kritikfähigkeit bemächtigen. Die Frage ist, wie nah Kovrins
Zustand den Theorien kommt und damit den der Theorien zu Grunde liegenden
Krankheitsgeschichten. Würde Kovrin also auch in der heutigen Zeit als Wahnsinniger gelten?
Nach der Analyse der Krankheitsgeschichte will ich versuchen die Entwicklung Kovrins auf
verschiedenen Ebenen aufzuzeigen. Es ist nämlich auch typisch für Čechov, dass „das Bild der Welt
sich aus einzelnen subjektiven Wahrnehmungen, Stimmungen, Erlebnissen, Gefühlen
zusammensetzt, die auch zeitlich als Momenterfahrungen kenntlich gemacht werden.“\autocite[83]{hielscher} Der
Gesamteindruck von Kovrin soll aufgeschlüsselt werden, um beispielweise die Veränderung in
seinem Gefühlsleben zu verdeutlichen. Auch die Wahrnehmung Kovrins durch andere Personen ist
interessant, da sie sich im Laufe der Zeit verändert. Vieles davon ist außerhalb der Theorie zum
Größenwahn angesiedelt, da solche Theorien nicht wirklich die Sicht aus der Perspektive der
Kranken wiedergeben können. Umso interessanter ist daher, wie Čechov die Umgebung in die
Entwicklung Kovrins eingebunden hat.

Als Letztes will ich mich dem Einfluss der Umgebung auf Kovrin widmen. Darunter sind vor allem
die Personen auf dem Landgut zu verstehen. Es wird im ersten Teil der Erzählung deutlich, dass die
Beziehung zwischen ihnen auf der einen Seite und Kovrin auf der anderen herzlich ist, in einer
kurzen Zeit aber umschlägt. Außerdem hat Kovrin die Halluzinationen erstmals auf dem Landgut.
Welche Gründe könnte es haben, dass aus einer leichten nervlichen Beeinträchtigung, die nur eine
Empfehlung des Arztes nach sich zieht, eine Megalomanie entsteht? In welchem Verhältnis steht
diese Entwicklung zu der Familie Pesockij?
\section{Zusammenfassung}
Um seine Nerven in Ordnung zu bringen, fährt der Psychologie- und Philosophiestudent A.
V. Kovrin zur Familie Pesockij auf ein Landgut, wo er unter der Obhut von Egor
Semenyč aufgewachsen ist.

Die Erinnerungen aus der Kindheit machen ihn glücklich und er kommt Tanja, der Tochter
seines ehemaligen Vormunds, näher. Kovrin findet auch ein herzliches Auskommen mit seinem
ehemaligen Vormund. Sein Leben dort wird allerdings nicht ruhiger, da er kaum schläft und dem
Studium seine Zeit widmet. An einem Abend erscheint ihm der schwarze Mönch, eine
Halluzination. Diese Erscheinung versichert Kovrin im Gespräch seiner Auserwähltheit und des
hohen Wertes seiner Arbeit. Das sorgt für eine Hochstimmung bei Kovrin. Pesockij trägt Kovrin die
Idee vor, ihm seine Tochter zu Frau zu geben. Innerhalb des Sommers auf dem Lande erscheint ihm
der schwarze Mönch regelmäßig und erfreut Kovrin mit seinen schmeichelnden Aussagen. In einer
glücklichen Anwandlung macht er Tanja einen Heiratsantrag. Beide heiraten und ziehen in die
Universitätsstadt. Dort offenbart sich Tanja, dass ihr Ehemann halluziniert und Selbstgespräche
führt. Er wird zu einem Doktor gesandt.

Nach der Heilung ist Kovrin seine Halluzinationen los, allerdings auch seine Hochstimmung.
Er erkennt seine Mittelmäßigkeit. Sein gesundheitlicher Zustand ist gekennzeichnet durch
Symptome der Tuberkulose. Seine wissenschaftliche Karriere scheitert dadurch. Seine
Beziehung mit Tanja nimmt nach der Heilung ebenfalls ein Ende. Nach der Trennung von
Tanja kommt er mit einer neuen Frau zusammen.

Bei einer Reise in den Süden liest er einen vorwurfsvollen Brief von Tanja, die vom Tod ihres
Vaters und von ihrem Unglück berichtet. Kurz darauf erscheint ihm der schwarze Mönch ein letztes
Mal und Kovrin stirbt.
\section{Kovrins mentaler Zustand als Größenwahn}
Es gibt einige Faktoren, die stimulierend auf die Psyche wirken und der Entstehung von
Größenwahn zuträglich sind. Es wird besonders unter drei Faktoren unterschieden: dem Autismus,
dem pathischen Empfinden und der Kritikfähigkeit, die ich nacheinander auf Kovrin zu
übertragen versuche.  Daneben werden noch sekundär die körperlichen Faktoren und der Einfluss
der Umgebung aufgezählt. Es sind modernen Beispielen abgeleitete Merkmale, aber die
Trefflichkeit ist, meines Erachtens, bemerkenswert und wird der Art der Erzählung „Der schwarze
Mönch“ als historia morbi gerecht. Die Kriterien stützen sich auf die Arbeit von Richard Avenarius, einem
Psychologen, der sich seit den 1970ern viel mit Patienten mit Größenwahn befasst hat.

\subsection{Autismus}
An erster Stelle verzeichnet Avenarius den Autismus als begünstigende Bedingung. Die gängige
Definition wurden dafür bereits 1911, also im ausklingenden ``nervösen'' Jahrhundert, von E.
Beuler gegeben: Eine autistische Verhaltensweise ist das ``Vorwiegen des Binnenlebens mit aktiver
Abwendung von der Außenwelt. Die schweren Fälle ziehen sich ganz zurück und leben wie in
einem Traum; in leichteren finden wir geringere Grade dieser Erscheinung. Das autistische Denken
ist vom Streben nach Lust und Vermeidung von Unlust geleitet, es spiegelt die Erfüllung von
Wünschen oder Strebungen vor, denkt Hindernisse weg und Unmöglichkeiten denkt es in
Möglichkeiten um''\autocite[38]{avenarius}. 

Sicher ist Kovrin kein schwerer Fall, aber in jedem Fall liegt der Fokus in der
Erzählung auf dem Binnenleben. So gibt Kovrin eine dem Autismus ähnliche Darstellung seines
Gefühlslebens: \textcyr{Мне кажется странным, что от утра до ночи я испытываю одну только
радость, она наполняет всего меня и заглушает все остальные чувства. Я не знаю, что такое
грусть, печаль или скука.}(248/VII)\footnote{Die Zitate aus der Erzählung basieren auf \autocite{blackmonk} 
; die erste Zahl steht für die Seitenzahl, die zweite für das Kapitel}

Der Autismus macht sich auch bemerkbar durch eine Überhöhung der eigenen Vergangenheit.
\textcyr{[...]и вдруг в груди его шевельнулось радостное молодое чувство, какое он испытывал в
детстве, когда бегал по этому саду}.“(232/I) Bei Kovrin finden wir häufig diese Assoziationen,
die auch direkt sich auf sein Gefühlsleben auswirken. Diese Rückbesinnung auf die Vergangenheit
zeugt von einer befreiten Stimmung, weil „die die Macht des Ichs eingrenzende Realität“\autocite[85]{avenarius}
vergessen wird.
\subsection{Pathisches Empfinden}
Ein anderer Faktor ist die „pathische Erlebnisweise“. Darunter ist zu verstehen, dass „die Kranken
in ihrer vermeintlichen Größe nicht die Auswirkungen der eigenen Kraft, sondern vielmehr Gnade,
Auftrag oder Beschenkung durch eine höhere, meist transzendente Macht erblicken. In diesem
Bewusstsein, Werkzeug eines Größeren zu sein, gleichen sich die Gegensätze von Macht und
Ohnmacht in einer Verschmelzung der Gefühle von Geborgenheit und Selbstsicherheit aus.“\autocite[48]{avenarius} In
einem Gespräch mit dem schwarzen Mönch ist ebenfalls von einem höheren Bewusstsein die Rede:
\textcyr{Ты один из тех немногих, которые по справедливости называются избранниками божиими.
Ты служишь вечной правде. Твои мысли, намерения, твоя удивительная наука и вся твоя
жизнь носят на себе божественную, небесную печать, так как посвящены они разумному и
прекрасному, то есть тому, что вечно.}(241f./V)

Aber auch der Mönch selbst ist Symbol einer höheren Macht. Der Mönch ist wie der Priester auch
``ein Stellvertreter der göttlichen Gewalt auf Erden. Durch sein Zölibat bekundet er seine
ausschließliche Bindung an Gott, durch sein vorbildliches Leben regt er die Gemeinschaft zur
Nachahmung an.''\autocite[248]{brittnacher}
\subsection{Verminderte Kritikfähigkeit}
Ein bei Kovrin zu findender Faktor ist die Senkung der Kritikschranke. Dabei werden „die der
Megalomanie entgegenstehenden Gegebenheiten nicht mehr vollständig erkannt und können
deshalb unterschätzt werden“\autocite[85]{avenarius} . Das trifft insofern zu, weil Kovrin ein Wissenschaftler ist und in
den Gebieten der Psychologie und Philosophie auch kritisch mit Legenden und dem Wahn umgehen
müsste. Die Darstellung der Legende dagegen ist nicht sehr wissenschaftlich\footnote{\autocite{kluge}
verwendet den Begriff ``pseudowissenschaftlich''
} , auch der Glaube an
das Erscheinen des Mönchs ausgerechnet auf dem Landgut ist kaum mehr wissenschaftlich. Dem
entspricht, ``dass die Kritik nicht vernichtet wird, sie stellt 
sich in den Dienst des Wahns''\autocite{avenarius}
%TODO: actually Jaspers (1913) zitiert in Avenarius (1978)
. Ein
weiterer Anhaltspunkt für seine veränderte Einstellung zur Wissenschaft bezeugen die poetischen
Gefühle, die sich in den vielfältigen Genres der Literatur zeigen.\footnote{Siehe Kap. 4.4} Ein anderes Anzeichen ist die
anfängliche Kritik am Mönch, dann aber eine Fortsetzung des Gespräches mit diesem: “Das
Wahnbedürfnis muss stärker als das Bedürfnis zu Kritik sein“\autocite[59]{avenarius}. Die beschriebenen Ursachen einer
verminderten Kritikfähigkeit sind zum Teil auch im Gefühlsleben begründet. So „können starke
Emotionen das kritische Denken vorübergehend außer Kraft setzen. Ferner können abnorme
Dauerstimmungen, Euphorie oder Depression, durch einseitige Auswahl der Apperzeptione und
Assoziationen kritikbehindernd wirken“\autocite[63]{avenarius}. Die Euphorie und die Assoziationen mit der Kindheit
sind übereinstimmend ein fester Bestandteil des mentalen Zustands von Kovrin.\footnote{Siehe Kap. 4.4 und 4.5}

\subsection{Weitere Faktoren}
Häufige Ursache eines Wahns können körperliche Beeinträchtigungen sein. Gleichzeitig
werden diese vom Erkrankten nicht als hinderlich oder leistungsmindernd empfunden\autocite[85]{avenarius}: Gleich am
Anfang erfährt der Leser von einer übergeordneten Instanz, dass Kovrin durch Überarbeitung sich die
Nerven zerrüttet hat. Dass er auf dem Lande kein ruhigeres Leben führt und an Schlaflosigkeit
leidet, sind weitere physische Faktoren, die von Kovrin bagatellisiert werden. Weiterhin ist die
Vernachlässigung des körperlichen Zustandes zu Gunsten des Geistes Gegenstand eines Gespräches
mit dem Mönch: — \textcyr{Римляне говорили: mens sana in corpore sano. — Не все то правда, что
говорили римляне или греки. Повышенное настроение, возбуждение, экстаз — все то, что
отличает пророков, поэтов, мучеников за идею от обыкновенных людей, противно животной
стороне человека, то есть его физическому здоровью.}(243/V) Da der Mönch selbst nur das
wiedergibt, was Kovrin denkt, wird in den Gesprächen auch deutlich, wie abwertend Kovrin über
die physische Gesundheit denkt.

Unter medizinischer Sicht wird auch der Umgebung eine wichtige Rolle zugeschrieben. Diese
Beziehung wird allgemein als „Leidentlastungstendenz“ bezeichnet. Die Umgebung tritt in diesem
Fall neben der Natur, die eine autistische Situation wiedergibt, besonders durch Personen auf. Im
Falle Kovrins treffen wir auf ein Nichtverstehen der Vorgänge im Garten, sowie generell eine
Schwierigkeit die Umgebung zu verstehen und mit ihr zu kommunizieren. Der Drang trotz seiner
„Insuffienz“ mit der Umgebung in Einklang zu kommen schlägt sich in der Megalomanie nieder. In
dieser Entwicklung tut Kovrin keine Anstalten mit der Umgebung in einen Dialog zu treten -- „er
erhöht sich im autistischen Raum über sie“\autocite[84]{avenarius}.

\subsection{Vergleich der Krankheitsgeschichte}
Wenn man Kovrin nun unter dem Aspekt des Typus des Wahns\footnote{Typen des Wahns wie bei \autocite{avenarius} vorgestellt} anschaut, so trifft darauf die
Bezeichnung „Durchsbruchstyp“ zu, da „die Kräfte von Antrieb, Triebhaftigkeit, Emotionalität [...]
auf das aktuelle Befinden und Verhalten einwirken, ohne -- oder nur in geringem Maße -- zuvor von
den Wünschen und Tendenzen der Lebensgeschichte in deren Dienst gestellt worden zu sein.“\autocite[73]{avenarius}
Als weitere Gruppierung des Größenwahns werden die folgenden Gesichtspunkte vorgeschlagen:
Was einer ist, was einer kann und was einer hat.\autocite[23]{avenarius} Bei Kovrin dominiert vor allem der erste Aspekt.
Darunter findet man die Gruppe des metaphysischen Wahns und der wahnhaften Geistesgröße.
Beiden ist gemeinsam, dass sich die Patienten durch Mittelmäßigkeit auszeichnen, sich aber
hoffnungslos in ihren Zielen übernehmen. Schließlich führt diese Menschen eine Halluzination zu
neuer Motivation und zu einer neuen eigenen Wahrnehmung. „Die Selbsterhöhung betrifft das, ...,
woran es in Wirklichkeit mangelte.“\autocite[24]{avenarius} Infolge der Therapie „wurde eine Beruhigung und
Distanzierung, aber keine eigentliche Korrektur erreicht“\autocite[24]{avenarius}. Auch bei Kovrin führt die somatische
Therapie zu zweifelhaftem Erfolg. Er ist „solider“ geworden, aber schließlich vergleicht er sich in
den Vorwürfen an die Pesockijs immer noch mit großen Persönlichkeiten und am Ende ist das
Bedürfnis, eine Krise durch den Größenwahn zu überwinden, spürbar. Aus diesem Grund erscheint
wohl auch der schwarze Mönch.

Zu sagen ist noch, dass bei Kovrin ein „rollenloser Autismus“\autocite{avenarius}, wie bei der Mehrheit der Patienten in
neuerer Zeit, vorliegt, da er nicht versucht seinen Größenwahn irgendwie den Mitmenschen zu
vermitteln, sondern eher zum Binnenleben hin ausgerichtet ist.
\section{Ebenen der Entwicklung}
\subsection{Wahrnehmung der Natur}
Bereits im ersten Kapitel gibt es deutliche Anzeichen für eine hohe Empfindlichkeit. Diese ist
besonders auf die Natur bezogen, wo zum einen Details ins Auge fallen und die gefühlsbetonte
Wortwahl zu merken ist.

Es fängt mit dem englischen Park an, der einen „strengen und mürrischen Eindruck“ macht:
\textcyr{Старинный парк, угрюмый и строгий, разбитый на английский манер[...].}(226/I) Für die
Natur finden sich in dieser Phase seiner Überempfindlichkeit noch eine Reihe herausragender
Worte. Die Blumen, z.B. Rosen und Lilien werden als „herrlich“ beschrieben. Die Umgebung wird
als „Königreich zarter Farben“ beschrieben und der Reichtum an Farben macht großen Eindruck auf
Kovrin. \textcyr{Таких у дивительных роз, лилий, камелий, таких тюльпанов всевозможных цветов,
начиная с ярко-белого и кончая черным как сажа, вообще такого богатства цветов, как у
Песоцкого, Коврину не случалось видеть нигде в другом месте.}(226/I) Besonders Details
werden herausgehoben: \textcyr{почувствовать себя в царстве нежных красок, особенно в ранние
часы, когда на каждом лепестке сверкала роса}(227/I). Auch die Erinnerung an die Kindheit
enthält vor allen Dingen Naturmotive, sowie eine starke emotionale Bindung, was sich in der
erlebten Rede bemerkbar macht: \textcyr{Каких только тут не было причуд, изысканных уродств и
издевательств над природой!}(227/I)

In der Wortwahl (carstvo, bogatsvo) ist eine deutliche Überhöhung der Natur festzustellen.
Nach der Heilung ist die Wahrnehmung der Umgebung eine ganz andere. Besonders als Kovrin
durch den Garten spazieren geht, ist ein deutlicher Verlust der Empfindlichkeit zu bemerken. So
sind Blumen für ihn kein Anziehungspunkt. Die Kiefern sind für ihn nicht mehr jung und fröhlich.
Die Interaktion der Naturelemente ist für ihn nicht mehr sichtbar. In dieser Phase, wo er noch im
vergangenen Jahr die Natur überhöhte, ist bei ihm Ernüchterung eingekehrt. \textcyr{Не замечая
роскошных цветов, он погулял по саду, посидел на скамье, потом прошелся по парку; дойдя
до реки, он спустился вниз и тут постоял в раздумье, глядя на воду. Угрюмые сосны с
мохнатыми корнями, которые в прошлом году видели его здесь таким молодым, радостным и
бодрым, теперь не шептались, а стояли неподвижные и немые, точно не узнавали его.}
(250/VIII)

Zuletzt wird seine veränderte Wahrnehmung der Natur sichtbar, als er schon das
Landgut verlassen und sich von Pesockijs getrennt hat. Nach dem vorwurfsvollen Brief von Tanja
kehrt bei Kovrin die Empfindlichkeit der Natur vom Anfang der Erzählung wieder ein, beim
Ausblick auf die Bucht wird besonders die Farbigkeit hervorgestellt. \textcyr{Бухта, как живая, глядела
на него множеством голубых, синих, бирюзовых и огненных глаз и манила к себе. В самом
деле, было жарко и душно и не мешало бы выкупаться.}(256/IX ) In seinen letzten Momenten
ruft er sich den Park in Erinnerung und assoziiert damit ein Glücksgefühl.
\textcyr{Он
звал Таню, звал большой сад с роскошными цветами, обрызганными росой, звал парк, сосны
с мохнатыми корнями, ржаное поле.}(257/IX)
\subsection{Wahrnehmung der Personen}
Kovrins Wahrnehmung der Personen ist ebenfalls von einer gefühlsmäßigen Empfindlichkeit geprägt.
Bereits am Anfang wird aus der Sicht Kovrins die Stimmung als aufgeregt wiedergegeben. Nicht
nur die Stimmung ist Teil seiner Wahrnehmung, sondern auch die Personen. So rühren sich bei
Kovrin die Gefühle für Tanja, besonders wegen ihres Aussehens, aber auch die Vergangenheit
Tanjas ruft in ihm Emotionen hervor. \textcyr{Ее широкое, очень серьезное, озябшее лицо с тонкими
черными бровями, поднятый воротник пальто, мешавший ей свободно двигать головой, и вся
она, худощавая, стройная, в подобранном от росы платье, умиляла его.}(228/I) Eine
Zuneigung wird spürbar: \textcyr{Ему почему-то вдруг пришло в голову, что в течение лета он может
привязаться к этому маленькому, слабому, многоречивому существу, увлечься и влюбиться,
— в положении их обоих это так возможно и естественно!}(230/I)

Die Zuneigung zu Tanja ist geprägt von einer emotionalen Sicht auf sie. Es scheint, dass seine
Zuneigung zu ihr vor allem auf der körperlichen und nervlichen Schwächlichkeit basiert. \textcyr{И он
чувствовал, что его полубольным, издерганным нервам, как железо магниту, отвечают нервы
этой плачущей, вздрагивающей девушки. Он никогда бы уж не мог полюбить здоровую,
крепкую, краснощекую женщину, но бледная, слабая, несчастная Таня ему
нравилась.}(240/IV) Eine herausragende Empfindlichkeit wird deutlich im Vergleich zu der wenig
vorteilhaften Beschreibung Tanjas. Auf dem Höhepunkt der Zuneigung zu Tanja wird eine ganz
besondere Beziehung deutlich: \textcyr{После каждого свидания с Таней он, счастливый,
восторженный, шел к себе и с тою же страстностью, с какою он только что целовал Таню и
объяснялся ей в любви, брался за книгу или за свою рукопись.}(246/VI) Eine verhängnisvolle
Verbindung bahnt sich dadurch an, denn „die Leidenschaft für Tanja und die leidenschaftliche
Arbeit, deren Wertlosigkeit ihm nach der Heilung zu Bewusstsein kommen wird, werden eins.“\autocite[96]{freise91}
Pesockij wird als nervös wahrgenommen, was sich an mehreren Stellen zeigt. So ist die Reaktion
auf einen im Grunde nichtigen Vorfall im Garten nervös und auch seiner aufgeregten Art wird
Kovrin gewahr. \textcyr{Вид он имел крайне озабоченный, все куда-то торопился и с таким
выражением, как будто опоздай он хоть на одну минуту, то всё погибло!}(230/I) Seine Artikel
machen ebenfalls einen nervösen Eindruck auf Kovrin. \textcyr{Но какой непокойный, неровный тон,
какой нервный, почти болезненный задор!}(237/III)

Die Beziehung zu den Pesockijs allgemein ist vor allem durch die Vergangenheit geprägt und dabei
wird eine starke emotionale Verbindung vermittelt: \textcyr{...он, потерявший отца и мать в раннем
детстве, до самой смерти не узнал бы, что такое искренняя ласка и та наивная, не
рассуждающая любовь, какую питают только к очень близким, кровным людям.}(240/IV)
Ist die Beziehung zu Pesockij am Anfang herzlich und fast schon väterlich, ist sie nach der Heilung
distanzierter und schon fast feindselig. Seine veränderte Beziehung zu den Personen spiegelt sich
wider in den Vorwürfen an Pesockij und Tanja. Im Gespräch mit Tanja äußert er sich abwertend
über Pesockij. \textcyr{Он не добрый, а добродушный. Водевильные дядюшки, вроде твоего отца, с
сытыми добродушными физиономиями, необыкновенно хлебосольные и чудаковатые, когда-
то умиляли меня и смешили и в повестях, и в водевилях, и в жизни, теперь же они мне
противны. Это эгоисты до мозга костей.}(252f./VIII)

Auch die Zuneigung zu Tanja ist nach der Heilung verflogen. \textcyr{[...] в которой, как кажется, всё
уже умерло, кроме больших, пристально вглядывающихся, умных глаз, воспоминание о ней
возбуждало в нем одну только жалость и досаду на себя.}(254/IX)

Doch schließlich dominieren bei ihm wieder die herzlichen Gefühle aus der Phase des
Größenwahns, es zeigt sich aber auch, dass er zur neuen Frau an seiner Seite keine echte Beziehung
hat, da er Tanjas Name in diesem Moment ruft: \textcyr{Oн хотел позвать Варвару Николаевну, которая
спала за ширмами, сделал усилие и проговорил: — Таня! Он упал на пол и, поднимаясь на
руки, опять позвал: — Таня!}(257/IX)

\subsection{Wie wird Kovrin wahrgenommen?}
Kovrin wird bei der Ankunft auf das Landgut hohe Verehrung entgegengebracht. Tanja ist zum
Beispiel hin und weg von ihm. \textcyr{Вы мужчина, живете уже своею, интересною жизнью, вы
величина...}(228/I) Ihr Vater hat ebenfalls eine herausragende Meinung von Kovrin. 
\textcyr{Ведь вы
знаете, мой отец обожает вас. Иногда мне кажется, что вас он любит больше, чем меня. Он
гордится вами.}(228/I) Nach der Erscheinung des schwarzen Mönchs sind die positiven
Auswirkungen auch für andere sichtbar. \textcyr{[...]все, гости и Таня, находили, что сегодня у него
лицо какое-то особенное, лучезарное, вдохновенное, и что он очень интересен.}(235/II)
Kovrin wird von Pesockij als einer der Seinen wahrgenommen und die Einstellung zu ihm drückt
sich in familiären Worten aus, auch seine eigene Verschlossenheit legt sich im Beisein von Kovrin.
\textcyr{Ты человек умный, с сердцем, и не дал бы погибнуть моему любимому делу. А главная
причина — я тебя люблю, как сына[...] и горжусь тобой...}(237/III)

Sein gefühlsmäßiger Zustand wird aber auch schon als suspekt aufgenommen. 
\textcyr{Но что с вами? —
удивилась она, взглянув на его восторженное, сияющее лицо и на глаза, полные слез. —
Какой вы странный, Андрюша.}(243/V) Die Krankheit Kovrins tritt für Tanja offen zutage,
als sie ihn beim Sprechen mit dem Phantom sieht. \textcyr{— Ты болен! — зарыдала она, дрожа всем
телом. — Прости меня, милый, дорогой, но я давно уже заметила, что душа у тебя расстроена
чем-то... Ты психически болен, Андрюша...}(249/VII) Die positive Wahrnehmung Kovrins
durch Tanja verflüchtigt sich nach der Heilung. \textcyr{[...]муж стал раздражителен, капризен,
придирчив и неинтересен.}(251/VIII) Zum Ende schlägt sich die Wahrnehmung durch andere
Personen praktisch ins Gegenteil um. \textcyr{Его лицо показалось Тане некрасивым и неприятным.
Ненависть и насмешливое выражение не шли к нему.}(253/VIII)

Zu einer deutlichen negativen Wertung kommt Tanja im Brief. \textcyr{Я приняла тебя за
необыкновенного человека, за гения, я полюбила тебя, но ты оказался сумасшедшим...} Man
merkt, dass in diesem Satz eigentlich die ganze Geschichte aus der Sicht Tanjas in einer
resümierenden Art wiedergegeben wird.

\subsection{Die Entwicklung seines Gefühlslebens}
Die am Anfang harmonische Beschreibung der Natur korrespondiert eng mit Kovrins persönlichem
Befinden. Er tritt befreit auf: \textcyr{Он засмеялся и взял ее за руку.}(228/I) Die Assoziationen sind
recht künstlerisch (\textcyr{[...] что хоть садись и балладу пиши, ``казочное впечатление``}) und die
befreite Stimmung wird auch in dem Zitat aus einer Oper deutlich:
\textcyr{Oнегин, я скрывать не стану,\\
Безумно я люблю Татьяну...}(230/I)

Das erste Kapitel beinhaltet schließt mit einem deutlichen Höhepunkt seiner Stimmung. Dabei
werden die Naturerlebnisse mit dem Gefühlsleben und der Arbeit verwoben. \textcyr{Он внимательно
читал, делал заметки и изредка поднимал глаза, чтобы взглянуть на открытые окна или на
свежие, еще мокрые от росы цветы, стоявшие в вазах на столе, и опять опускал глаза в книгу,
и ему казалось, что в нем каждая жилочка дрожит и играет от удовольствия.}(232/I)
Die flüchtige Begegnung mit dem schwarzen Mönch wirkt sich positiv auf sein Gefühlsleben aus.
\textcyr{[...] приятно взволнованный, он вернулся домой.}(235/II) Auch das Denken an den schwarzen
Mönch weckt in ihm freudige Gefühle. \textcyr{[...] и ему опять стало хорошо.}(238/III) Was für eine
Begeisterung in ihm die Treffen mit dem schwarzen Mönch wecken, lässt sich nach der Begegnung
mit dem Mönch im Gespräch mit Tanja feststellen. \textcyr{Я больше чем доволен, я счастлив! Таня,милая Таня, вы чрезвычайно симпатичное существо. Милая Таня, я так рад, так
рад!}(243f./V) Es wird an dieser Stelle überdeutlich, dass seine Exaltiertheit eine Überhöhung der
umgebenden Personen zur Folge hat.

Sein Gefühlsleben nach der Heilung verkehrt sich ins Gegenteil um. Das wird dadurch deutlich,
dass ihn die Zeit auf dem Landgut langweilt. \textcyr{[...] в старом громадном зале запахло точно
кладбищем, и Коврину стало скучно.}(250/VIII), \textcyr{Теперь я стал рассудительнее и солиднее,
но зато я такой, как все: я — посредственность, мне скучно жить...}(250/VIII) In seinen
Vorwürfen wird eine Ironie und seine Hasserfülltheit spürbar. \textcyr{Если бы вы знали, — сказал
Коврин с досадой, — как я вам благодарен!}(252/VIII)

Eine deutliche Veränderung seines Gefühlslebens tritt nach der Trennung von Tanja ein in
Sevastopol. Besonders ihr vorwurfsvoller Brief ebnet den Weg für das letzte Erscheinen des
Mönchs. Nicht nur eine Unruhe angesichts seiner eigenen Schuld beseelt ihn, sondern auch die
Angst machtlos gegenüber dem Schicksal zu sein. \textcyr{Им овладело беспокойство, похожее на
страх.} und \textcyr{[...]распорядилась им опять та неведомая сила, которая в какие-нибудь два года
произвела столько разрушений в его жизни и в жизни близких.}(255/IX)

Am Ende, beim Erscheinen des Mönchs, ist er allerdings wieder den Gefühlen nahe, die er
besonders am Anfang der Erzählung und nach den Gesprächen mit dem Mönch spürte. \textcyr{Он видел
на полу около своего лица большую лужу крови и не мог уже от слабости выговорить ни
одного слова, но невыразимое, безграничное счастье наполняло все его существо.}(257/IX)

\subsection{Eigene Wahrnehmung}
Sein Größenwahn beginnt kurz vor dem ersten Auftauchen des Mönchs. Er spürt in sich eine
gewisse Größe und Bedeutung emporsteigen. \textcyr{И кажется, весь мир смотрит на меня, притаился
и ждет, чтобы я понял его[...]}(234/II)
Nach dem Erscheinen des Mönchs entsteht seine gute Laune aus der Verneigung negativer
Konsequenzen aus den Halluzinationen. \textcyr{«Но ведь мне хорошо, и я никому не делаю зла;
значит, в моих галлюцинациях нет ничего дурного», — подумал он, и ему опять стало
хорошо.}(238/III)

Das Gespräch mit dem Mönch gibt Kovrin das Gefühl ein Auserwählter zu sein. Besonders seine
eigene Arbeit wird von ihm überhöht. \textcyr{Быть избранником, служить вечной правде, стоять в
ряду тех, которые на несколько тысяч лет раньше сделают человечество достойным царствия
божия, то есть избавят людей от нескольких лишних тысяч лет борьбы, греха и страданий,
отдать идее все — молодость, силы, здоровье, быть готовым умереть для общего блага, —
какой высокий, какой счастливый удел!}(243/V)

Zurückgekehrt in der Universitätsstadt sieht Kovrin wieder den schwarzen Mönch und das
Gespräch gibt besonders Auskunft darüber, wie kritisch Kovrin auf seine glückliche Stimmung
blickt. \textcyr{И меня, как Поликрата, начинает немножко беспокоить мое счастье. Мне кажется
странным, что от утра до ночи я испытываю одну только радость, она наполняет всего меня и
заглушает все остальные чувства.}(248/VII) Als Erklärung versichert der Mönch Kovrin wieder
seiner Auserwähltheit. Seine Phantasie erreicht für ihn einen Grad der Wirklichkeit, sodass er sogar
zu seiner Verteidigung auf den Mönch weist. Schließlich gewahr, dass auch seine Mitmenschen ihn
für krank halten, revidiert auch er nun seine eigene Wahrnehmung. \textcyr{Только теперь, глядя на нее,
Коврин понял всю опасность своего положения, понял, что значат черный монах и беседы с
ним. Для него теперь было ясно, что он сумасшедший.}(249/VII)

Nach der Heilung nimmt er sich nun als mittelmäßig wahr, was sich in den Vorwürfen gegenüber
Tanja und ihren Vater bestätigt. In dem Zerreißen seiner wissenschaftlichen Arbeiten wird eine
veränderte negative Einstellung zu seiner Vergangenheit auf dem Landgut und der Wissenschaft
deutlich. \textcyr{Кстати же он вспомнил, как однажды он рвал на мелкие клочки свою диссертацию
и все статьи, написанные за время болезни, и как бросал в окно, и клочки, летая по ветру,
цеплялись за деревья и цветы;}(254/IX) und \textcyr{Коврин теперь ясно сознавал, что он —
посредственность,[...]}(256/IX). In dieser Phase erscheint für ihn auch sein lebenslanges Streben
in der Wissenschaft sinnlos. \textcyr{Он думал о том, как много берет жизнь за те ничтожные или
весьма обыкновенные блага, какие она может дать человеку.}(256/IX)

Durch Tanjas Brief wird bei ihm eine schwere Krise ausgelöst und der Drang sich dieser
Situation zu entbinden, kulminiert im letzten Erscheinen des Mönchs. In seinen letzten
Augenblicken erhöht er sich selbst, ähnlich der Zeit vor seiner Heilung: 
\textcyr{Коврин уже верил тому,
что он избранник божий и гений,[...].}(257/IX)

\section{Rolle der Umgebung bei der mentalen Entwicklung Kovrins}
Zunächst sollte man darauf hinweisen, dass die Umgebung sich nur auf wenige Figuren beschränkt.
Es gibt tatsächlich nur Tanja und Pesockij, die mit Kovrin kommunizieren. Andere Figuren werden
in den Hintergrund gedrängt, beispielsweise haben die Arbeiter eine durch Äquivalenzen
signalisierte nichtige Rolle: \textcyr{[...]и изредка им встречались работники, которые бродили в дыму,
как тени.}(228/I) Figuren mit Namen wie Ivan Karlovic oder Varvara Nikolaevna sind nicht in
Dialoge eingebunden. Zuletzt ist die Menschenleere für Kovrin immerzu spürbar. \textcyr{Ни
человеческого жилья, ни живой души вдали,[...]}(234/II) und \textcyr{[...]стало казаться, что во всей
гостинице кроме него нет ни одной души...}(256/IX).

Die Umgebung auf dem Landgut hat trotzdem einen großen Einfluss auf Kovrin. Der Erzähler legt
bereits am Anfang den Grundstein für die nervliche Beeinträchtigung Kovrins. Zunächst gibt ihm
eine umgebende Person den (vermeintlich falschen) Rat auf Land zu fahren. \textcyr{Он не лечился, но
как-то вскользь, за бутылкой вина, поговорил с приятелем доктором, и тот посоветовал ему
провести весну и лето в деревне.}(226/I) Wir können aus der Wahrnehmung der Umgebung,
nämlich der Personen und der Stimmung, schließen, dass die Atmosphäre auf dem Landgut
angespannt ist. Die Streitereien zwischen Pesockij und seiner Tochter wegen Kleinigkeiten sind
ebenfalls ein Anzeichen dafür. An einer Stelle wirkt sich das auf Kovrin aus. \textcyr{Коврин был
погружен в свою интересную работу, но под конец и ему стало скучно и неловко.}(239/IV) Es
ist deshalb durchaus anzunehmen, dass Kovrins eigene nervliche Anspannung sich auf dem Lande
verschlimmert. Auch die Ruhelosigkeit Kovrins findet ihre Übereinstimmung in dem hektischen
Betrieb des Gartens. \textcyr{От раннего утра до вечера около деревьев, кустов, на аллеях и клумбах,
как муравьи, копошились люди с тачками, мотыками, лейками[...].}(227/I) Es wurde von
Forschern besonders darauf hingewiesen, dass sich der Größenwahn Kovrins auch unter Einfluss
der schmeichelnden Kommentare Tanjas und Pesockijs entwickelt.\autocite{alekseev}

In der neueren wissenschaftlichen Literatur wurde besonders Pesockij eine treibende Rolle bei
Kovrins Entwicklung zugedacht.\autocite{freise91} Diese entspringt zum einen aus der charakterlichen Ähnlichkeit
mit seinem ehemaligen Vormund, die sich zum Beispiel durch Pesockijs zu Größenwahn neigender
Ansichten über den Garten äußert. \textcyr{Это не сад, а целое учреждение, имеющее высокую
государственную важность, потому что это, так сказать, ступень в новую эру русского
хозяйства и русской промышленности.}(236/III) In Kovrins eigenem Größenwahn ist das, was
der Mönch ihm sagt, ähnlich. \textcyr{Вы же на несколько тысяч лет раньше введете его в царство
вечной правды — и в этом ваша высокая заслуга.}(242/V) Auch zeugt Pesockijs
Ausdrucksweise von einer gewissen Hektik und Nervosität.\autocite[93]{freise91} Die Beschäftigung mit dem Garten
veranlasst ihn, wissenschaftliche Texte zu verfassen, in denen sich nicht nur seine eigene
Wahrnehmung als über den Anderen stehend zeigt, sondern auch sein nervöser Stil. Eine
„Anomalität“ tritt an dieser Stelle hervor.\autocite[90]{freise91} Die Nervosität und eine Neigung zur Schizophrenie
treten in der Szene zu Tage, kurz nach der Entdeckung eines Fehlers eines Gehilfen und der
freudigen Bekundung in Richtung Kovrin. \textcyr{Повесить мало! Успокоившись, он обнял Коврина и
поцеловал в щеку. — Ну, дай бог... дай бог... — забормотал он.}(231/I) Dies ist nur ein erstes
Anzeichen für eine gewisse Seltsamkeit Pesockijs. Später tritt diese auch deutlich nach dem
Heiratsantrag Kovrins an Tanja hervor. \textcyr{В нем уже сидело как будто бы два человека: один был
настоящий Егор Семеныч, [...] , и другой, не настоящий, точно полупьяный, [...].}(246/VI)
Unter psychologischem Standpunkt findet hier eine 
„doppelte Orientierung/Buchführung“\autocite[61]{avenarius} statt,
d.h., dass der Mensch nur in gewissen Situationen und bestimmten Themen in eine Zerfahrenheit
verfällt. Damit geht auch temporär eine „Einengung der Interessen und erstarrende Einförmigkeit
des Denkens“ einher“\autocite[64]{avenarius}. Dies wird „schizophrene Persönlichkeitsabwandlung“ genannt. Tanja selbst
hat ebenfalls eine sprunghafte Art. Man könnte dies dahingehend auffassen, dass Tanja in ihrer
Erziehung diese Art von ihrem Vater übernommen hat -- wieso dann auch nicht Kovrin?. Nach M.
Freise gibt es durchaus weitere Anhaltspunkte für den Einfluss von Pesockij. Zwar wird die Zeit
Kovrins unter der Obhut von ihm nicht detailliert beschrieben, aber ein Satz von Tanja lässt
schließen, dass der Einfluss existiert hat: \textcyr{Вы ученый, необыкновенный человек, вы сделали
себе блестящую карьеру, и он уверен, что вы вышли такой оттого, что он воспитал
вас.}(228f./I) In der neueren Psychiatrie ist es wiederum bewiesen, dass mentale Erkrankungen
Wurzeln in der Jugend haben könnten -- das beweist nach neueren Daten die Untersuchung
verschiedener Größenwahnsinniger.\autocite[10]{avenarius} Des Weiteren ist die Teilung des Gartens an sich ist schon ein
Ausdruck der Schizophrenie Pesockijs und tritt auch in symbolträchtige Korrespondenz zu seinem
Zustand.\autocite{freise91}

Es gibt auch Standpunkte, die im Gegensatz die Unterschiedlichkeit der 
Personen hervorheben.\autocite{winner}
Bezogen auf einige Aspekte wie zum Beispiel die Ebenen der Beschäftigung der Personen trifft das
auch zu -- Kovrin beschäftigt sich nämlich mit Theorie, während Pesockij praktisch veranlagt ist.
Aber diese Gegenüberstellung ist nur oberflächlich und in Bezug auf den durchaus vorhandenen
Drang Pesockijs sich wissenschaftlich zu betätigen auch nur bedingt richtig. Nicht zuletzt Suchich
hat bemerkt, dass die Gegenüberstellung von Personen für Čechov allgemein untypisch ist.
\footnote{Suchich (1983) zit. n. \autocite{freise91}} 
%TODO: actually Suchich (1983) zitiert in M. Freise (1991)
Tatsächlich führt M. Freise die These, dass Pesockij die treibende Kraft bei Entwicklung Kovrins
ist, weiter aus. So speist sich das Glücksgefühl im Garten aus der Vergangenheit und erinnert stark
an das Gefühl eines Säuglings oder eines Kindes: \textcyr{Прекрасное настоящее и просыпавшиеся в
нем впечатления прошлого сливались вместе; от них в душе было тесно, но хорошо.}
Weiterhin kann man die Heilung als ein Erwachsenwerden und als eine Abkapselung von Pesockij
auffassen. Demgegenüber steht aber eine gewisse Rückkehr zum bekannten Zustand der
Abhängigkeit. Das spiegelt sich wider vor allem in der Person der Varvara Nikolaevna. \textcyr{Жил он
уже не с Таней, а с другой женщиной, которая была на два года старше его и ухаживала за
ним, как за ребенком.}(253/IX) Ohne Zweifel ist diese Herangehensweise sehr nah an der historia
morbi orientiert und gleichzeitig auch interpretatorisch wertvoll. Das spätere Anlegen eines podtext-
Struktur\footnote{
Podtext: „... temporal organisierte zweite Ebene der nicht ausgesprochenen Motive und Ziele“, nach \autocite{freise97}
} im Sinne einer impliziten Fabel, im Gegensatz zur Krankheitsgeschichte als explizite
Fabel, kann man als sinnvolle Weiterentwicklung betrachten, weil es die Umgebung mehr
instrumentalisiert.

\section*{Schlussbetrachtung}
\markboth{Schlussbetrachtung}{Schlussbetrachtung}
\addcontentsline{toc}{section}{Schlussbetrachtung}
Als ich die Erzählung „Der schwarze Mönch“ das erste Mal gelesen habe, wurde ich keiner
Krankheit bei Kovrin gewahr. Vielleicht lag es daran, dass ich mir nicht vorstellen konnte, wie eine
Krankheit, wenn auch im seelischen Sinne, so positiv sich auswirken konnte, ohne die Umwelt
vollkommen umzugestalten.

Im Laufe des mehrmaligen Lesens wird aber deutlich, dass der Wahn eine Ausflucht ist aus der
mittelmäßigen Realität. Das sieht Kovrin selbst am Besten ein.\footnote{
Siehe Kap. IX in \autocite{blackmonk}} Nun wird der Größenwahn in der
Erzählung nicht von einer übergeordneten Instanz als tatsächlich existent deklariert. Er wird
subjektiv ausgedrückt. Es war daher gewissermaßen mein Ziel die subtilen und impliziten
Handlungen auf eine ausgereifte wissenschaftliche Theorie zu beziehen. Die Wahnhaftigkeit
irgendwie greifbar zu machen -- dem war der erste Teil meiner Arbeit gewidmet.
Es hat mich ein bisschen überrascht, dass die Definitionen gepasst haben. Andererseits ist es nicht
verwunderlich, weil ich im Laufe meiner Recherche auch in neueren Büchern Zitate und
Definitionen aus dem 19. Jahrhundert fand. Dieses Jahrhundert beherbergte wohl nicht
umsonst Persönlichkeiten wie Freud oder Nietzsche\footnote{
Inwiefern die Erzählung einen Bezugspunkt zu Nietzsche mit seinen Schriften hat, siehe \autocite{winner}
} , die auch heute noch in den
wissenschaftlichen Beiträgen zur Psychologie und Psychiatrie präsent sind.
Was sagt uns also „Der schwarze Mönch“? Bezogen auf die Analyse des Wahns, lässt sich sagen,
dass die Wurzeln der heutigen Ansichten in der Zeit Čechovs liegen. Kovrin würde nämlich, meiner
Analyse folgend, auch in der heutigen Zeit als mental krank eingestuft werden.
Der zweite Teil meiner Arbeit sollte den Gesamteindruck der mentalen Entwicklung in die Ebenen
teilen, die sich jeweils durch den Bezugspunkt definieren. Auch dadurch wird der Wahn und seine
Entwicklung begreifbarer. Vor allem unter der Einwirkung des stilistischen Mittels der
Äquivalenzen wird mehrmals deutlich, wie sich die Umgebung einerseits zum mentalen Zustand
Kovrins verhält und andererseits auch zueinander, z.B. vor der Heilung und nach der Heilung.
Zuletzt war es meine Absicht die Umgebung von Kovrin nach der Rolle in seiner Entwicklung zu
überprüfen. Tatsächlich kann man der Umgebung eine ursächliche Bedeutung für den Wahn bei
Kovrin beimessen. Denn die Ähnlichkeiten zwischen Kovrin und Pesockij sind nicht von der Hand
zu weisen. Auch sind die Bezüge auf Kovrins Erziehung zwar subtil, aber vorhanden. Die
Umgebung ist insofern nicht nur ein Bezugspunkt für Kovrin, sondern er ist auch ein Bezugspunkt
für die Umgebung.

Das sind die grundlegenden Resultate meiner Hausarbeit. In Bezug zum Wahnsinn streift meine
Hausarbeit aber nur einige Aspekte der Erzählung. Thematisch würde sich als Vertiefung
beispielsweise die Religion anbieten. So könnte man die Äquivalenzen in Bezug auf die Religion
und den religiösen Wahn analysieren. Bei meiner Recherche sind mir nur wenige Arbeiten
begegnet, die besonders die Religion berücksichtigt haben.\autocite{strelcova} Das scheint mir trotzdem ein
erfolgversprechender Ansatz zu sein, denn der schwarze Mönch hat neben seinen göttlichen
Aussagen auch was vom Teufel („lukavyj“ wie ihn Kovrin einmal beschreibt) und je nachdem wie
man das Ende auffasst, sind die Halluzinationen schlussendlich positiv oder negativ. Es wäre auch
insofern interessant, da bei vielen an Größenwahn Leidenden die Megalomanie mit religiösen Inhalt
ausgefüllt ist.

\literature
\end{document}
