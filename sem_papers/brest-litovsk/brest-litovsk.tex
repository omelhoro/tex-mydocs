\documentclass{../../sem_paper}
\addbibresource{brest-litovsk.bib}

\begin{document}
\titlepg
{Außenpolitik der Bolschewiki von der Novemberrevolution bis zum Frieden von Brest-Litowsk: Konzept oder Pragmatismus?}
{2014}
{Fachbereich Geschichte}
{Hauptseminar 54-329 "`Osteuropa im Ersten Weltkrieg"'}
{SS 2012}
{Frank Golczewski}

\tocpaper

\closuresection{Einleitung}
"`The forgotten peace"'-- so betitelte John Wheeler-Bennett 1956\autocite{wheeler1956} sein Buch zum Frieden von Brest-Litowsk. Dies galt aber wohl nur für die westliche Seite, da der Versailler Vertrag sehr viel umfänglicher war und zudem den ersteren für nichtig erklärte. Auf der sowjetischen Seite entdeckte man in den Verhandlungen so etwas wie eine Goldgrube. Zuerst um bestimmte Bolschewiki zu verunglimpfen\footnote{Dies lässt sich zum Beispiel lesen bei \fullcite{bailey1955}. Der Ngram Corpus von Google gibt auch interessante Ergebnisse: Suchbegriff: "`\textcyr{сорвать Брестский мир}"', Korpus: Russisch (2009). Ähnlich interessant ist es, mit Hilfe einer Wildcard zu suchen: "`\textcyr{* Брестский мир}"'}, später um sie für die friedlichen Absichten der Sowjets und Gier des Kapitalismus als Beweis herzuhalten\footnote{Als Beispiel: \fullcite{chuba1964}, \fullcite{maio1954} und \fullcite{osad1969}}.
Mit der Öffnung von Archiven war auch im Westen der Weg für tiefere Forschungen geöffnet. Besonders in den 90ern führte dies dazu, dass Brest-Litowsk zum wichtigen Bestandteil von Studien über sowjetische Außenpolitik\footnote{Als Beispiele: \fullcite{kenzie1994} und \fullcite{petro21997}} wurde. Der Perestroika und  dem Zusammenbruch der Sowjetunion folgten auch russische Arbeiten, die nicht mehr unter ideologischen Zwängen standen\footnote{\fullcite{pand1990} und \fullcite{fels_1992} . Das letzte Werk wird im Folgenden in der englischen Fassung verwendet.}.
  

Die vorliegende Arbeit soll sich auf eine Frage konzentrieren, die sich aus einer einfachen Tatsache ergibt: Mit der Machtübernahme der Bolschewiki gab es zum ersten Mal eine sozialistische Regierung an der Spitze eines Landes. \textbf{Auf welcher Grundlage wurde die Außenpolitik geführt? Gab es ein Konzept oder war die Außenpolitik von einer pragmatischen Linie gekennzeichnet?}
Das Interessante an dieser Frage ist, in welcher Lage die Bolschewiki an die Macht kamen: Es war das 3. Kriegsjahr, das Land hatte bereits eine Revolution hinter sich, Russland war militärisch praktisch am Ende.

Die vorliegende Arbeit soll die oben gestellte Frage zunächst in 2 kleinere Etappen teilen, denen jeweils ein Kapitel gewidmet ist: Konzept und Realpolitik:
Das 1. Kapitel soll die Konzepte der wichtigsten Bolschewiki Lenin und Trockij vorstellen. Das Wort Konzept soll zunächst dem Wort Ideologie vorgezogen werden, da das letztere schon eine Wertung ist. Das Unterkapitel 1.1 hat die Besonderheit, dass es größtenteils auf Sekundärquellen basiert mit vereinzelten Primärquellen. Der Grund ist trivial: Lenin und Trockij haben umfangreiche Werke geschrieben -- Lenins gesammelte Werke umfassen 55 Bände -- und es ist daher naheliegender Analysen von Historikern zu nehmen, die alles schon kompiliert haben. 

In 1.2 soll auch das Friedensdekret, ein Dokument, das die Sowjetregierung schnell nach der Machtübernahme veröffentlichte, vorgestellt werden. Sicherlich nimmt die Tatsache, dass das Friedensdekret innerhalb des Konzeptkapitels vorgestellt wird, schon die spätere Interpretation vorweg. Die Argumente werden dafür im Laufe der Arbeit vorgestellt.

Das 2. Kapitel hat das Ziel, die faktische Grundlage der Interpretation darzustellen. Dies beinhaltet eine kurze Zusammenfassung der Lage vor der Novemberrevolution, danach sollen die Verhandlungen in Brest-Litowsk in Phasen unterteilt zusammengefasst werden. Dies soll in etwa die Realpolitik darstellen und damit die Basis für das nächste Kapitel legen. 

Im 3. Kapitel soll anhand der vorgestellten Konzepte und Fakten eine Erklärung der Außenpolitik vorgenommen werden. Dazu soll chronologisch vorgegangen werden. Der Zeitpunkt, an dem die Arbeit aufhört, ist Ende Februar/Anfang März. Da wurden nämlich die Weichen gestellt für die Unterschrift des Brest-Litowsker Friedensvertrages. Die vorliegende Arbeit greift also den Vorgang der Ratifizierung nicht auf. Dieser Vorgang schien mir mit zu vielen formalen, innenpolitischen Dingen verbunden zu sein, wobei ja das Ziel der Arbeit die Außenpolitik ist und da war das Resultat schon Ende Februar/Anfang März erreicht.
Innerhalb der 4 Ansätze von McFadden\autocite{fadden1993}, die er als "`research design"' definiert, soll die Arbeit im 2. verortet werden. Dieser Ansatz zeichnet sich dadurch aus, dass man zu einer  begrenzten zeitlichen Phase theoretische (konzeptuelle) Ansichten der Bolschewiki mit der praktischen Implementierung vergleicht. 
\\\\
In der gesamten Arbeit wird der westliche (gregorianische) Kalender  verwendet, was die Revolutionen jeweils im März und November stattfinden lässt. Alle neu eingeführten Personen sollen mit vollem Namen und Lebensdaten genannt werden. Patronym wird abgekürzt; als Nachname wird bei Bolschewiki der "`Kampfname"' genommen. Danach wird auf die Personen nur mit Nachnamen referiert.

\npsection{Die konzeptuelle Ausgangslage}

\subsection{Außenpolitische Konzepte vor der Revolution}
\label{sec: for_policy}
"`Proletarier aller Welt, vereinigt euch!"'-- mit diesen letzten Worten beendete Karl Marx (1818-1883) "`Das Kommunistische Manifest"'. Seine grundlegende Idee war, dass die Arbeiter sich vorrangig mit ihrer Klasse (dem Proletariat) identifizieren und nicht mit der Nation. Die gesellschaftlich-ökonomische Theorie führte später zu einer politischen Entwicklung und einer Anpassung an Russland. In Russland taten dies vor allem Lev Trockij (1879-1940) und Vladimir Lenin (1870-1924).
%TODO: where said from Marx
1905 entwickelte Trockij das Konzept der "`Permanenten Revolution"'. 
%TODO: replace wikipedia with better resources
%\begin{enumerate}
%  	\item Übergang von der demokratischen Revolution zur sozialistischen
%	\item die Revolution sollte zu einem ständigen Neuverhandelt zu Permanenz 
%	\item den internationalen Charakter der sozialistischen Revolution, deren nationaler Beginn lediglich als revolutionäres Anfangsstadium zu werten ist.
%\end{enumerate}
Er entwickelte das Konzept am Beispiel Russlands, um zu erklären, wie man eine kommunistische Revolution in einem Staat auslösen könnte, in dem der größte Teil der Bevölkerung Bauern waren. Dazu umging er die von Marx aufgestellte Bedingung für den Kommunismus, dass ein Staat kapitalistisch voll entwickelt sein sollte. 
Ein wichtiger Gedanke zur Außenpolitik war, dass die Weltrevolution zeitlich nach der in Russland passieren soll.

Lenins Idee einer Außenpolitik unterschied sich zwar in Bezug auf die Beziehung von Proletariat und Bauernschaft von Trockijs Konzept, aber in Bezug auf die Weltrevolution waren seine Ideen ähnlich. Lenin als wichtiger Theoretiker seiner Partei übte damit auch Einfluss auf die restlichen Bolschewiki aus. Lenin sah keinen Sinn im Frieden mit den "`imperialistischen"' Mächten, sondern wollte einen revolutionären Krieg führen.\autocite[1]{maka2010}
Auch das Stichwort des demokratischen Friedens in der ersten der Aprilthesen lässt sich in diesem Kontext sehen: "`As Lenin defined "`demokratic conditions"' in a way which no existing government could accept, including the liberation of all colonies and dependent nationalities, he was pledging to wage revolutionary war agains world capitalism."'\autocite[7]{debo1979}
In seiner Theorie sollte die Kette der imperialistischen Staaten am schwächsten Punkt reißen, nämlich in Russland.

In dem Zeitraum zwischen den Revolutionen schrieb Lenin von der Sinnlosigkeit einer Friedensvereinbarung mit den Kapitalisten:
\begin{quotation}
\textcyr{Войну, которую ведут капиталисты всех богатейших держав, войну, которая вызвана десятилетней историей экономического развития, окончить отказом от военных действий с одной стороны, — это такая глупость, что нам смешно даже ее опровергать. [\ldots] Войну, которую ведут капиталисты всех стран, нельзя кончить без рабочей революции против этих капиталистов.}\autocite[97]{lenin_voj}
\end{quotation}

Da Lenin und Trockij die beiden bedeutendsten Theoretiker ihrer Partei waren, kann man den Überblick auch auf sie beschränken. Interessant ist, dass Trockij lange Zeit ein "`free-lance revolutionary"'\autocite[13]{debo1979} war; erst nach dem Sturz des Zaren fand er zu Lenins Partei. Das soll die Gemeinsamkeiten verdeutlichen. Dass Trockijs außenpolitische Linie der Lenins nicht im Gegensatz stand, kann man ebenfalls daraus ersehen, dass das Zentralkomitee Trockij zum Kommissar des Auswärtigen (Außenminister in der heutigen Terminologie) machte.

\subsection{Dekret über den Frieden}
\label{sec: dekret}
Als wichtiges Dokument für die bolschewistische Außenpolitik nach der Machtübernahme sollte man die Veröffentlichung des Dekretes über den Frieden sehen. Die Wichtigkeit des Dokumentes kann man aus der relativ schnellen Veröffentlichung am 8. November ersehen, sowie seiner Rolle als erstes Dekret der Bolschewiki. Gleichzeitig wurde es über alle möglichen Medien gesendet, so in Zeitungen, Radio und Flugblättern. Hauptsächlich wird ein "`demokratischer"' Frieden propagiert, d.h.
\begin{quotation}
\textcyr{[\ldots]
таким миром Правительство считает немедленный  мир  без  аннексий
(т.е. без захвата чужих земель, без насильственного присоединения
чужих народностей) и без контрибуций.}\autocite[12]{dekret}
  \end{quotation}
Es wird weiter ausgeführt, was die Bolschewisten unter diesen Bedingungen verstehen. Ausgeschlossen wird nämlich eine Vereinnahmung von Völkern  "`\textcyr{без точно, ясно  и  добровольно  выраженного   согласия   и   желания   этой
народности
}"'\autocite[12]{dekret}. Dies gilt für jede Nation, unabhängig ob "`\textcyr{в Европе или в далеких заокеанских странах эта нация живет}"'\autocite[12]{dekret}.
Insofern hat das Dokument Aussagen, die einen Vor-Weltkriegsstatus propagieren könnten.
Dass das aber nicht das Ziel des Dekretes ist, zeigen die Aussagen bezüglich der Selbstbestimmung der Völker. Und zwar wenn "`\textcyr{какая бы то  ни  было  нация  удерживается  в  границах данного государства насилием}"'\autocite[12]{dekret} (man bemerke den Präsens) und diesem Volk kein Recht auf eine Wahl zugestanden werden, dann ist dies eine Annexion.

Als weiteren wichtigen Punkt, der die Weichen für die Umsetzung der Außenpolitik stellte, sollte man die Forderung nach offener Diplomatie nennen. Dies verband man mit der Veröffentlichung von geheimen Abkommen zwischen den Regierungen:
\begin{quotation}

\textcyr{
Тайную дипломатию  Правительство отменяет,  со своей стороны
выражая твердое намерение вести все переговоры совершенно открыто
перед всем народом,  приступая немедленно к полному опубликованию
тайных договоров,  подтвержденных или заключенных  правительством
помещиков  и  капиталистов  с  февраля по 25 октября 1917 г.
}\autocite[15]{dekret}
\end{quotation}


Außerdem zählt die Regierung die vorgestellten Bedingungen als nicht ultimativ. Vorschläge sollten innerhalb des Prinzips der offenen Diplomatie gemacht werden und zwar "`\textcyr{без всякой  тайны  при  предложении  условий
мира.}"'\autocite[15]{dekret}
Um den Friedensschluss zu ermöglichen, wurde die Möglichkeit eines Waffenstillstandes angesprochen, der auf 3 Monate ausgelegt werden sollte. In ihrem Verlauf
\begin{quotation}
\textcyr{
вполне  возможно  как  завершение  переговоров о мире с
участием представителей всех без изъятия народностей  или  наций,
втянутых  в  войну  или вынужденных к участию в ней,  так равно и
созыв полномочных собраний народных представителей всех стран для
окончательного утверждения условий мира.}\autocite[15]{dekret}
\end{quotation}

Zusammenfassend liegen also die folgenden Forderungen dem Dokument zu Grunde (der letzte Punkt ist als Angebot zu verstehen):
\begin{itemize}
  \item Frieden ohne Kontributionen und Annexionen
  \item Selbstbestimmung der Völker
  \item Ende der Geheimdiplomatie
  \item Waffenstillstand
  \item Verhandlungsbereitschaft
\end{itemize}

An der Wortwahl wird klar, dass sich das Dekret nicht primär an Regierungen wendet, sondern direkt an die Menschen. Dies kann man indirekt an emotionalen Sätzen wie "`\textcyr{подавляющее  большинство  истощенных,  измученных  и  истерзанных войной рабочих и трудящихся классов всех воюющих стран}"' sehen. Aber auch direkter an "`\textcyr{обращается   также   в   особенности   к
сознательным рабочим трех самых передовых  наций}"'\autocite[16]{dekret}. Der gesamte letzte Absatz listet Errungenschaften der Arbeiter in England, Frankreich und Deutschland auf. Außerdem werden die Arbeiter in die Pflicht genommen, die arbeitenden Massen von Sklaverei und Ausbeutung zu befreien.
Es lassen sich damit starke Parallelen zu den Konzepten der Bolschewiki vor der Revolution feststellen.

\npsection{Die Verhandlungen in Brest-Litowsk}

\subsection{Politische Entwicklung nach der Märzrevolution}
Obwohl die Außenpolitik in Russland die ersten 2 Kriegsjahre von der Innenpolitik eher getrennt war, wurden die Proteste in Russland mehr und mehr von der Kriegsfrage dominiert. Dies führte zu Streiks und staatlichen Gegenreaktionen in Form von harschen Polizeieinsätzen mit Toten. Nach dem Sturz des Zaren, kam  die bürgerliche Provisorischen Regierung an die Macht, die es sich mit den Sowjets (in der damaligen Terminologie: Räte, die Gruppen wie Soldaten und Arbeiter  repräsentierten) teilen musste. Auch durch Druck der Allierten wollte die Provisorische Regierung den Krieg nicht beenden, sondern setzte alles auf eine Offensive in Galizien (westliche Ukraine), die allerdings verloren wurde. Im September bedrängte ein Vormarsch des Generals Lawr G. Kornilow (1870-1918) die Provisorische Regierung und die Sowjets in St. Petersburg (damals Petrograd). Erst mit Hilfe der Sowjets sowie inhaftierter Soldaten konnte sich die Provisorische Regierung retten. Nach einem Anwachsen der Unzufriedenheit konnte die Bolschewistische Partei die Provisorische Regierung stürzen und die Macht übernehmen. Als Machtzentrum etablierte sich das Zentralkomitee der Bolschewiki, das sich aus 13-15 stimmberechtigten Mitgliedern zusammensetzte.
%TODO: Central komitee vs. constuent assembly

\subsection{Ziele in der ersten Phase der Verhandlungen}
Die Veröffentlichung des Friedensdekretes kurz nach der Revolution führte in anderen Ländern zu Sympathiebekundungen mit Sowjetrussland. Allerdings blieben die Regierungen der Kriegsparteien  ohne Antwort auf das Dekret. 

Erst als sich Ende November Trockij offiziell an die Gesandten der Staaten mit dem Vorschlag wendete, Waffenstillstand zu schließen, konnte man zumindest die Mittelmächte für einen Waffenstillstand gewinnen. Nach 2 Wochen Verhandlungen wurde der Waffenstillstand am 15. Dezember geschlossen für eine Dauer von 4 Wochen. Durch die Annäherung zwischen Mittelmächten und Sowjet-Russland gab es für die Allierten einen Grund mehr, den Verhandlungen fern zu bleiben, da sie sich durch den Kriegseintritt der USA mittelfristig Siegeschancen ausrechneten.

Die Verhandlungen über einen Frieden begannen am 22. Dezember. 
Nachdem der Leiter der russischen Delegation Adolf A. Joffe (1883-1927) ein 6-Punkte-Programm auf Grundlage des Friedensdekretes vorgelegt hatte, legten die Mittelmächte ein Angebot vor, dass sich stark am vorgeschlagen Programm orientierte. Allerdings machten sie das Angebot von einem Frieden mit den Allierten auf Grundlage der gleichen Konditionen zur Bedingung. Am 28. Dezember wurde daher eine Verhandlungspause bis zum 8. Januar ausgegeben.

\subsection{2. Phase}
An diesem Übergang zur 2. Phase konnten die Allierten nicht dazu gebracht werden, an den Friedensverhandlungen teilzunehmen, sodass das Angebot der Mittelmächte nichtig wurde. Die 2. Phase der Verhandlungen kann man mit dem 8. Januar datieren, als die Verhandlungen nun mit Trockij als dem Haupt der sowjetischen Delegation begannen. Sein anfängliches Ziel, die Verlegung nach Stockholm, um die Verhandlungen noch öffentlicher zu machen, konnte er gegen die Mittelmächte nicht durchsetzen.

Zu Anfang der 2. Verhandlungsphase nahm auch eine ukrainische Delegation, die Rada, an den Verhandlungen teil. Trockij musste auch dies akzeptieren, obwohl die Rada national und anti-bolschewistisch gesinnt war.
Die Verhandlungen gingen deutlich langsamer vorwärts und der militärische Teil der deutschen Delegation wurde mit dem Verlauf unzufrieden. General Hoffmann legte Trockij am 18. Januar Forderungen vor mit konkreten Gebieten, die Russland abtreten sollte. Der Ablehnung am gleichen Tag folgte eine Unterbrechung für 10 Tage. Diese nutzte Trockij, um in Petersburg die Forderungen der Deutschen vorzustellen. Die Versammlung des Zentralkomitees am 24. Januar stimmte für die Linie Trockijs, die Verhandlungen fortzuführen, aber auch keinen Friedensvertrag zu unterzeichnen.

Wieder in Brest-Litowsk fuhr er mit seiner Verzögerungstaktik fort. Währenddessen begannen in Deutschland und Österreich Ende Januar Streiks. Die Mittelmächte konnten sie allerdings niederschlagen. Außerdem gelang es den Mittelmächten, einen separaten Friendensschluss mit der ukrainischen Delegation zu schließen. Dies passierte, während noch eine 2. ukrainische Delegation zu Verhandlungen dazukam, die mit den Bolschewisten verbündet war. Mit Hilfe der Bolschewiki eroberte sie immer größere Teile der Ukraine. In diese Zeit fällt die ultimative Forderung der Deutschen am 9. Februar, das ihnen noch ein größeres Einflussgebiet geben sollte.

Am 10. Februar lehnte Trockij das Ultimatum ab mit der Formel "`Weder Krieg, noch Frieden"' und verließ Brest-Litowsk.

\subsection{3. Phase}
Dies hatte die Mittelmächte und besonders die Deutschen sehr überrascht und ohne ein klares Vorgehen gelassen: "`Die fruchtlosen Verhandlungen fortsetzen? Die Bolschewiki mit militärischen Aktionen zwingen, das deutsche Ultimatum anzunehmen? Sie stürzen und statt ihrer ein akzeptables Regime einsetzen"'\autocite[416]{pipes1992}.

Die deutschen Militärs setzten sich durch. Am 18. Februar lief der Waffenstillstand aus und der Vormarsch der Mittelmächte ging weiter. Die Bolschewiki hatten im Gegenzug \textit{damit} nicht gerechnet; das Zentralkomitee konnte aber schnell eine Linie finden, nach der man den Mittelmächten wieder Frieden anbietet. Gleichzeitig nahm Trockij auch Kontakt mit den Allierten auf, um von ihnen Hilfe im Kampf gegen die Mittelmächte zu bekommen.

Nach einigen Tagen gaben die Mittelmächte ein Ultimatum durch, das das Zentralkomitee akzeptierte. Am 3. März wurde der Vertrag in Brest-Litowsk von einer zweitrangigen sowjetischen Delegation unterzeichnet und am 16. März ratifiziert. Die genauen Bedingungen des Brest-Litowsker Vertrages wurden vor der Öffentlichkeit so lange wie möglich geheim gehalten.

Der Vertrag sah vor, dass Russland Polen, das Baltikum sowie die Ukraine aufgibt und sämtliche Propaganda einstellt. Die Armee musste demobilisiert werden. Es mussten auch Reparationszahlungen geleistet werden. Durch den Vertrag verlor Russland 1/4 seiner Bevölkerung und fast 3/4 seiner Kohle- und Eisenproduktion\autocite[89]{rauch1990}.

\npsection{Konzept oder Pragmatismus?}

\subsection{Konzeptuelles Vorgehen}
Wie lassen sich die ersten Schritte der Bolschewiki in der Außenpolitik begründen? Das Friedensdekret kann man als Fortführung der theoretischen Linie sehen, nämlich die Arbeiter zu animieren und die Weltrevolution mit einem Weltfrieden zu verbinden. Es sollte durch die Veröffentlichung der geheimen Dokumente die Regierungen bloßstellen und war nicht zuletzt durch die Forderung der Selbstbestimmung der Völker keine Option für die an Kolonien reiche Entente (außer USA). Gleichzeitig stimme ich zu, der demokratische Frieden "`\textcyr{был продиктован прагмачитескими соображениями большевисткого руководства: показать капиталистическому окружению демократический характер новой рождавшийся России [\ldots]}"'\autocite{maka2010}.\footnote{Man muss aber Makarenko in einem anderen Punkt widersprechen. Denn das Dekret war keine Antwort auf Wilsons 14-Punkte Programm, da es zeitlich davor datiert. Insofern ist es nicht richtig, zu denken, dass das Dekret ein Vorbild hatte.}

Es ist trotzdem nicht verwunderlich, dass sich keine Regierung angesprochen fühlte. Dafür konnte man von Seiten der Bolschewiki mit dem Dekret innenpolitisch für Ruhe sorgen, denn es stellte Frieden in Aussicht und war damit ein Gegensatz zur Politik der Provisorischen Regierung: "`The war had been a crucial stumbling block for Kerenky's government. [\ldots] Lenin had no such illusions. An armistice was esential if the Bolshevik state was to survive."'\autocite[39]{marples2011}

Nach dem Dekret erfolgten die Friedensverhandlungen nicht unmittelbar: Dafür musste Trockij die Länder auf konventionellere Weise ansprechen. Schließlich fanden sich auch nur die Mittelmächte als Verhandlungspartner ein. Ihre Teilnahme kann man eher pragmatisch erklären, denn die österreichische Führung zweifelte daran, ein weiteres Jahr durchzuhalten. In Deutschland bereitete man eine Offensive auf der Westfront vor und brauchte die Divisionen von der Ostfront. Einem schnellen Frieden seitens der Mittelmächte stand nichts im Weg, doch für die Bolschewiki und besonders Trockij waren die Verhandlungen eher eine "`Weltbühne"', um für die Weltrevolution zu werben auf Grundlage des Friedensdekretes. Die Friedensverhandlungen in Brest-Litowsk waren dazu gedacht, sie "`als Agitationsbühne benutzen und aus dem Fenster hinaus Reden an die Welt halten, um das westeuropäische Proletariat zur Revolution aufzustacheln."'\autocite[21]{baum1966}
Die Friedenverhandlungen widersprachen also nicht dem Ziel der Bolschewiki, eine Weltrevolution zu entfachen. 

Die Änderungen im Personal der Delegation zu Beginn der 2. Verhandlungsphase kann man im Kontext sehen, dass sich die Friedensverhandlungen zu einem "`Boomerang"'\autocite[17]{maka2010} für die Bolschewiken entwickelten. Der Vehandlungsführer der deutschen Delegation Richard v. Kühlmann (1873-1948) notierte: Sein Plan war es "`auf dem Selbstbestimmungsrecht der Völker fußend, den Punkt des annexionslosen Friedens zu unterhöhlen[\ldots] und was wir an territorialen Zugeständnissen durchaus brauchten, uns durch das Selbstbestimmungsrecht der Völker hereinzuholen."'\autocite[21]{baum1966}
%TODO: Second hand citation:(
Mit dieser Taktik, mit dem Argument der Selbstbestimmung also, konnten die Mittelmächte die Teilnahme der ukrainischen Rada in Brest-Litowsk begründen und konnten mit dem gleichen Argument die Loslösung von Polen, Litauen, Lettland aus Russland fordern. Außerdem brachten die Verhandlungen der 1. Phase und 2. Phase keine Revolution in Deutschland. Durch diese Gründe wird verständlich, dass sich im Januar verschiedene Gruppen bildeten.

\subsection{Entwicklung einer pragmatischen Position}
Als Außenkomissar vertrat Trockij seine Linie, die Gespräche zu propagandistischen Zwecken zu nutzen. Auf der Versammlung des Zentralkomitees kam es zum Streit über die folgende Politik der Bolschewiki:
Lenin wollte mit den Mittelmächten Frieden schließen. Trockij war weiterhin gewillt, die Verhandlungen zu Gunsten der Propaganda zu verwenden. Die Annahme der Forderungen der Mittelmächte lehnte er ab.  Der 3. Teil der Bolschewiki wurde von Nikolai I. Bucharin (1888-1938) angeführt, der seinen größten Einfluss in Moskau hatte. Er befürwortete einen revolutionären Krieg nach dem Vorbild Frankreichs Ende des 18. Jahrhunderts.

Lenins 21 Thesen (später 22) beschreiben eine Politik ohne Friedensschluss als "`Abenteuerpolitik"'\footnote{Da die Thesen, aus welchem Grund auch immer, nicht im offizielen Protokollbuch: \fullcite{proto1958} auftauchen, wird stattdessen die deutsche Quellensammlung verwendet: \fullcite{baum1969}. Der Einheitlichkeit halber werden die Protokolle daher auch in deutscher Übersetzung zitiert.}. Und er erteilte auch zunächst Absage an andere Taktiken: "`Irgendwelche Mittelwege sind hier im Grunde ausgeschlossen. Ein weiterer Aufschub ist nicht mehr möglich, denn um die Verhandlungen künstlich in die Länge zu ziehen, haben wir bereits alles mögliche und unmögliche getan"'\autocite[102]{baum1969}. 
Obgleich er einem \textit{baldigen} revolutionären Krieg eine Absage erteilte, war dies nur kurzfristig, denn "`die Frage, ob man \textit{jetzt, sofort}, einen revolutionären Krieg führen kann, muss man entscheiden, indem [..] die Bedingungen seiner Durchführbarkeit und die Interessen der sozialistischen Revolution [\ldots] in Rechnung stellt"'\autocite[105]{baum1969}.
In seiner letzten These, die nach einem Gespräch mit Trockij eingefügt wurde, wertete er die Streiks in Deutschland als "`Tatsache, dass die Revolution in Deutschland begonnen hat."'\autocite[108]{baum1969} 
Damit sah er einen Grund, die Verhandlungen weiter zu verzögern.

Diese letzte These war ein starker Schwenk in Richtung Trockijs Position und nicht mit den 21 vorhergehenden Thesen vereinbar. Auf der Versammlung des Zentralkomitees am 24. Januar wurde die Taktik des Verzögerns zur Abstimmung gestellt und hat die meisten Stimmen bekommen. Auch Trockijs Formel "`Weder Krieg, noch Frieden"' bekam die Mehrheit als Schlusspunkt der Taktik des Verzögerns.\autocite[114]{baum1969} Bucharin unterstützte Trockijs Vorschlag ebenfalls.

Dass sich diese Resultate wenig an Realpolitik orientierten, zeigte sich allerdings, als die Deutschen mit der Ukraine eine Alternative zu einem separaten Vertrag mit Sowjetrussland hatten. Aus der Ukraine bekam Russland einen großen Teil seines Getreides und seiner Kohle; diese Ressourcen wären ins deutsche Einflussgebiet gefallen. Auch die Proteste in Deutschland waren eher begrenzt im Umfang und selbst auf dem Höhepunkt Ende Januar/Anfang Februar konnten die Deutschen sie niederschlagen.

Lenin erfasste die Lage also sehr pragmatisch. Auch eine einseitige Friedenserklärung schloss er aus wie man am Zitat in Kapitel \ref{sec: for_policy} sehen konnte: Das Beenden von "`\textcyr{[\ldots] военных действий с одной стороны, — это такая глупость[\ldots]}"'. Wieso hat Lenin seine Position also nicht stärker verteidigt? Wie man an der letzten These sehen kann, wollte Lenin die Partei zusammenhalten. Dies drückt auch ein Satz Lenins über Trockij aus: "`\textcyr{Уж ради одного доброго мира с Троцким стоит потерять Латвию с Эстонией.}"'\footnote{\textcyr{Моя жизнь,} http://www.magister.msk.ru/library/trotsky/trotl026.htm\#st28}. Seine Anhänger waren in der Minderzahl, wohingegen die Ideen von Trockij großen Einfluss hatten wie man an der Zustimmung zu seiner Idee am 24. Januar sehen kann. Insofern ist die von Lenin begrüßte Verzögerungstaktik bei Friedensverhandlungen eher zu verstehen als ein Zeitgewinn, um seinen Einfluss in der Partei zu festigen, und nicht als Zeitgewinn für die baldige Weltrevolution\autocite[106]{fels2012}.

Am 18. Februar fanden nach dem erneuten Angriff der Deutschen 2 Sitzungen des Zentralkomitees statt, bei der es um die Frage ging, ob man Friedensverhandlungen einleiten sollte.
Auf der ersten wurde die Position von Lenin knapp abgelehnt, wobei Trockij zunächst für Abwarten war, weil "`es nicht ausgeschlossen [ist], daß der deutsche Vorstoß eine ernste Explosion auslöst."'\autocite[117]{baum1969}
Auf der 2. Sitzung wurde der Vorschlag Lenins dann aber knapp angenommen durch die Stimme Trockijs, obwohl Bucharin weiter für den revolutionären Krieg argumentierte und die Ereignisse ins vorrevolutionäre Konzept einbettete: "`Alles was sich jetzt ereignet, haben wir vorhergesehen. [\ldots] entweder entfaltet sich die russische Revolution oder sie bricht unter den Schlägen des Imperialismus zusammen."'\autocite[119--120]{baum1969}.

Erst als sich Trockij mit seiner Position irrte (keine Revolution in Deutschland, Vormarsch der Mittelmächte), konnte Lenin seine Vorstellungen von einem Frieden mit den Mittelmächten umsetzen. Dies konnte er nun auch drängender tun, nämlich durch die Androhung seines Rücktritts. Während eine solche Drohung im Januar keine Wirkung erzielt hätte, hatte sie im Februar deutlich mehr Gewicht. Bucharin stimmte zwar nicht für den Friedensschluss, sein Einfluss auf andere Mitglieder verringerte sich aber durch die Ereignisse im Februar. 

Diese genannten Gründe sind die Erklärung hinter der Reaktion auf den deutschen Vormarsch und der Mehrheit für Lenins Position auf den Sitzungen des Zentralkomitees, aber auch ein Beweis, dass die Pragmatisierung der Außenpolitik das Konzept der Weltrevolution nicht verdrängte.

\newpage
\closuresection{Abschließende Worte und Ausblick}
Wie kann man nun die erbrachten Ergebnisse verstehen? Im ersten Teil wurde dargestellt, welche Konzepte zur Außenpolitik es gab. Tatsächlich könnte man sich zur Meinung verführen lassen, dass es keine Konzepte gab. Vor der Machtübernahme dachte man in bolschewistischen Kreisen, dass zeitig nach der Machtübernahme in Russland auch die Weltrevolution beginnen wird. Dazu passt ein Ausspruch von Trockij: "`\textcyr{Какая такая у нас будет дипломатическая работа? [\ldots] вот издам несколько революционных прокламаций к народам и закрою лавочку}"'\footnote{\textcyr{Моя жизнь,}http://www.magister.msk.ru/library/trotsky/trotl026.htm\#st28}
Damit lässt sich auch die These\autocite[389]{pipes1992} von Pipes belegen, nach der die bolschewistische Diplomatie eine Absage an die über Jahrhunderte praktizierte Diplomatie Europas war. Das Friedensdekret zeigte genau das. Durch seine Art war es nicht an Regierungen, sondern an das Proletariat gerichtet.

Die eher pragmatischen Elemente im Friedensdekret kann man vor allem durch den Einfluss von Lenin erklären:
Ich stimme der Interpretation\autocite[18]{debo1979} von Debo zu, nach der die Stellen im Friedensdekret, in denen Verhandlungsabsicht und Friedensabsichten geäußert werden, als eine Alternative gedacht waren, falls die Weltrevolution nicht sofort ausbrechen würde.
Der Beginn der Verhandlungen ließ sich ebenfalls in dieser Art interpretieren. Ein Frieden hätte wohl schon im November oder Dezember geschlossen werden können, \textit{falls} denn die Bolschewiki an einem Frieden interessiert gewesen wären. Stattdessen haben die Bolschewiki die Friedensverhandlungen in das Konzept der Weltrevolution eingebettet. Gleichzeitig merkte man den Bolschewiki an, dass sie Neulinge in der Außenpolitik waren. Sie haben die besser werdende Position der Mittelmächte unterschätzt. Trockijs Plan B "`Weder Krieg, noch Frieden"' erwies sich gegenüber den deutschen Militärs als nicht praktikabel.

Erst in der 2. Verhandlungsphase kehrte zumindest Lenin vom Konzept der nahen Weltrevolution und des revolutionären Krieges ab und verfolgte eine Taktik, die der Realpolitik Beachtung schenkte. Ihm fehlten zu der Zeit allerdings die Mittel, um seine Partei stärker zu beeinflussen. Lenin bewies einen Realitätssinn, nicht nur für die Außenpolitik, sondern auch für die eigene Partei: Während er erkannte, dass eine Verzögerungstaktik oder ein revolutionärer Krieg keinen Erfolg bringen würden, konnte er seine Position erst nach dem faktischen Scheitern der anderen  Ansätze durchsetzen. 

Die Entwicklung vom ideologischen Konzept zur pragmatischen Außenpolitik konnte des Weiteren mit der chronologischen Entwicklung der Friedensverhandlungen verknüpft werden. In Phase 1 dominierte das Konzept der Weltrevolution, in Phase 2 entwickelte sich in einer Minderheit der Partei ein realpolitischer Ansatz und zuletzt führte ebendieser in der 3. Phase zum Friedensvertrag. Der wichtigste Faktor dieser Entwicklung war Deutschland: Denn es war die "`alte"' Diplomatie der deutschen Delegation, die die Taktik der Bolschewiki aushebelte. Insbesondere die Ultimaten der Militärs, die Einigung mit der ukrainischen Rada und vor allem die konsequente Reaktion auf "`Weder Krieg, noch Frieden"' leisteten einen Beitrag zur Pragmatisierung der bolschewistischen Außenpolitik.

\subsection*{Ausblick}
Etwas weiter in die Zukunft schauend, kann man da sagen, dass diese Periode von November 1917 bis März 1918 eine Abkehr vom Konzept der Weltrevolution war?
Nach den vorgestellten Aussagen Lenins war diese Periode eher eine Verschnaufpause, um die Macht im Inneren zu stärken. Der spätere (erfolglose) Vormarsch der Sowjets nach Polen war ein Versuch, die Revolution auch durch militärische  Hilfe ausbreiten zu wollen. Zweckmäßiger ist es, die gegebene Periode als ein Glied in einer Kette von Verschnaufpausen zu sehen, vor der Genoaer Konferenz und dem Vertrag von Rapallo sowie der wirtschaftlichen Annäherung zum Westen während der 20er Jahre\autocite[4]{goro1994_2}.
Damit verbunden ist diese Periode als Beginn einer dualen Diplomatie zu interpretieren, "`to achieve national security by fostering diplomatic relations with the West, while simultaneously encouraging subversion and revolutionary activities when conditions seemed propitious"'\autocite[33]{goro1994}.


Vom Ergebnis der vorliegenden Arbeit könnte man noch weiter gehen und fragen, wieso das Konzept der Weltrevolution nicht erfolgreich war. War es tatsächlich so, dass für Arbeiter außerhalb Russlands die Bindung zur Nationalität stärker war als zur Klasse? Haben die deutschen Sozialisten Karl Liebknecht und Rosa Luxemburg nicht genug für die Bewegung getan oder haben die Bolschewiki bloß schlechte Propaganda gemacht? Welchen Anteil hatten die Erfolge der Sozialdemokratie in den westlichen Ländern an der relativen Immunität gegenüber dem Sozialismus?

Die Ergebnisse der Arbeit konnten eine herausgehobene Rolle von Lenin in der Partei sichtbar machen. Ausgehend davon wäre eine genaue Betrachtung der Beziehungen in Partei wertvoll, um zum Beispiel den Einfluss Bucharins zu erklären oder die Nähe Stalins zu Lenins Standpunkten.
Am vordringlichsten erscheint aber eine Zusammenführung der russischen/sowjetischen sowie der westlichen Historiographie. Die benutzte Literatur baut, wie ich gemerkt habe, in Bezug auf Sekundärquellen eher auf der einen oder anderen Seite auf. 

%Der Zeitraum, den die vorliegende Arbeit bearbeitet hat, ist de facto der Beginn der sowjetischen Außenpolitik. Dieses "`the first crisis of soviet diplomacy"' brachte für folgende Periode der Diplomatie mehrere Erkenntnisse: 

\literature
\end{document}
