\documentclass{../../sem_paper}
\addbibresource{january_uprising.bib}

\begin{document}
\titlepg
{Quellenanalyse von „The Polish National Committee's
Proclamation of January 22, 1863“}
{2011}
{Fachbereich Geschichte}
{Einführungsseminar I "`Russischer Imperialismus"'}
{WS 2010/2011}
{Kristina Küntzel-Witt}

\tocpaper

\closuresection{Einleitung}
Polen heute: Es ist ein Land mit 40 Millionen Einwohnern, ein fester Bestandteil der Europäischen
Union, eine der größeren Volkswirtschaften Europas und außerdem Mitglied der NATO.
Polen 1863: Ein Land, das auf der Landkarte nicht existiert. Polen ist ein kleiner Teil des riesigen
russischen Reiches, sowie in kleinerem Maße auch Teil Preußens und Österreich-Ungarns.

Dabei ist die Geschichte Polens im 17. und bis Mitte des 18. Jahrhunderts eigentlich einer
Großmacht würdig. Das Gebiet Polens (und Litauens) im Jahr 1618 nahm zum einen fast die
gesamte heutige Ukraine ein. Zum anderen breitete es sich auch ins heutige Russland, Weißrussland
und das Baltikum aus. Doch war Polen durch diese Präsenz auch prädestiniert für einen Konflikt mit
einer emporkommenden Macht des Kontinents: Russland. So war der Verlauf der Geschichte, dass
je größer Russland auf dem europäischen Kontinent wurde, desto kleiner wurde Polen. Doch der
Anfang vom Ende der polnischen Unabhängigkeit ist nur zum Teil im Ausland zu finden. Denn vor
allen Dingen war eine politische Krise verantwortlich dafür, dass Polen – von der Größe oder
Bevölkerungszahl den meisten Nachbarn mindestens ebenbürtig – schrittweise (1772, 1792 und
1795) unter die Herrschaft der umgebenden Mächte kam.
%TODO: Noch ein Wort zu dieser politischen Krise wäre gut

Damit war das polnische Nationalgefühl aber keinesfalls gebrochen. Ein guter Beweis dafür sind
herausragende Persönlichkeiten wie Mickiewicz, Juliusz Słowacki und Zygmunt Krasiński, die in
ihren Werken durchaus poetische und nationale Töne anschlugen. Die Aufstände in den Jahren
1794, 1830 und 1848 zeugen davon, dass die polnische Bevölkerung sich keineswegs mit ihrem
Schicksal abgefunden hat.

Dieser Arbeit liegt die Quelle zu Grunde, die zur letzten großen Erhebung der polnischen
Bevölkerung im Russischen Reich vor dem 1. Weltkrieg geführt hat.
Im ersten Teil, der Quellenbeschreibung, wird die Quelle mit ihren formalen Eigenschaften
untersucht. Beispielsweise wird der Anlass näher beleuchtet. Außerdem soll bestimmt werden, wer
das Manifest geschrieben hat und an wen es gerichtet ist. Des Weiteren hat der erste Teil zum Inhalt,
die unklaren Sachverhalte zu klären.

Der zweite Teil ist der Quelleninterpretation gewidmet. Zuerst will ich versuchen, die
gesellschaftliche und politische Entwicklung bis zum Januaraufstand zu skizzieren. Die Quelle soll
vor allen Dingen mit der Frage verbunden werden, welche Gruppen und Ideologien in diesem
Manifest vertreten sind und was die Quelle an inhaltlichen Eigenschaften enthält .
Im Schlussteil werden die Resultate zusammengefasst und ein kurzer Überblick über die Zeit nach
dem Manifest gegeben.

\npsection{Inhaltsangabe}
Das am 22. Januar 1863 veröffentlichte Manifest wird in der Forschungsliteratur als Beginn des
Januaraufstandes gewertet. Inhaltlich ist es bestimmt von einem kämpferischen Ton. So ist am
Anfang die Rede von einer verachtenswerten Regierung, die junge Männer in den Militärdienst
schickt. Als Reaktion darauf verheißt das Manifest Widerstand gegen diese Maßnahme. Diese
jungen Männer werden als Vorbild genommen für die Nation. Das Komitee erklärt sich zur einzigen
legalen Regierung in Polen, zum Befehlshaber und Tribunal. Den Leuten, die in den Kampf ziehen,
wird eine Erhöhung ihrer Stellung im heldenhaften Sinn in Aussicht gestellt, sowie eine materielle
Vergütung. Die Opferbereitschaft des Volkes wird beschworen.

Außerdem wird den Menschen Freiheit und Gleichheit versprochen. Das Land, welches die Bauern
in einer Abhängigkeitsbeziehung zu den Gutsbesitzern beackern, soll ihnen ohne Einschränkung
überlassen werden. Es werden dabei noch die Litauer und die Westrussen zum Kampf aufgefordert.
Zuletzt ergeht ein Appell an das russische Volk dem Zaren nicht zu gehorchen und nicht die Waffen
zu ergreifen. Falls es das doch tue, werde es gleichfalls als Feind betrachtet werden.

\section{Quellenbeschreibung}
Bei der vorliegenden Quelle handelt es sich um ein Manifest (vom lat. manifestus, „handgreiflich
gemacht“). Dies ist eine öffentliche Erklärung von Zielen und Absichten.

Der Verfasser des Manifestes ist das polnische Nationalkomitee, das sich aus mehreren Leuten,
zumeist politisch links stehend, zusammensetzte. Das Nationalkomitee wurde 1861 noch unter dem
Namen „Stadtkomitee“ gegründet – bezogen auf Warschau, dem hauptsächlichen Schauplatz der
Unruhen. Das Manifest richtet sich an die Litauer und Westrussen sowie die polnische Nation. Da
es aber zu der Zeit keinen Nationalstaat gab, kann man hier von der polnischen Bevölkerung
sprechen. Allerdings sollte man auch eingrenzen, dass sich das Manifest besonders auf die russische
Teilungszone, das sog. Kongress Polen, bezieht. Es ist nicht direkt gesagt, aber anhand einiger
Merkmale lässt sich darauf schließen. Wie man nämlich weiter sehen wird, beziehen sich viele
erwähnte Ereignisse und Sachverhalte auf das Russische Reich. Auch der Anlass, die
Zwangsaushebung, ist im russischen Teilungsgebiet zu finden.

Der Anlass dieses Manifestes ist die am 15. Januar 1863 erfolgte Zwangsrekrutierung der
polnischen Jugend in die russische Armee. In der polnischen Jugend und Studentenschaft war die
revolutionäre Stimmung besonders verbreitet. Die Einziehung musste deshalb ein schwerer Schlag
sein gegen die revolutionäre Bewegung. Daher waren die Revolutionäre unter Handlungsdruck. Ein
Satz zeigt, welche Bedeutung diese zur Einziehung vorgesehenen Männer hatten: „Legions of young
men, brave and devoted to the cause of their country, have sworn to cast away the abhorred yoke or
to die, and they place their reliance in the just assistance of the Almighty. Follow these, O Polish
nation!“(Abs. 2 S. 612).\footnote{
Die Zitate, soweit sie dem Manifest entstammen, basieren auf \fullcite[611]{source1972}; 
die kursive Schrift
markiert meine eigenen Hervorhebungen.}\footnote{Außerdem wird im Folgenden der Kürze halber auf die Quelle mit Absatz und Seite referiert.}
Die Männer werden als Vorbild gebraucht, um die übrige Bevölkerung
mitzureißen.

Eine an die ganze Bevölkerung gerichtete Aussage hat a priori auch einen starken
Öffentlichkeitsgrad. Die Quellensprache bestätigt das. Es finden sich nämlich viele Passagen, die
mit Gott zusammenhängen. Der Glaube ist immer in Polen sehr wichtig gewesen, die Identifikation
mit der Kultur betrifft auch heute noch besonders die Religion. Die wichtige Rolle der Religion war
besonders in der Solidarność-Bewegung und ihrem Anführer Lech Wałęsa sichtbar. Die Kirchen
avancierten auch in der Zeit vor dem Januaraufstand 1863 zu einem Ort, wo die Menschen sich
austauschen und der repressiven Atmosphäre entfliehen konnten.Unter anderem sorgte auch die
Schließung der Kirchen Mitte 1861 für einen Schub in der Bewegung. Stellen, in denen sich der
Bezug zur Religion ausdrückt, sind zum Beispiel: „... and they place their reliance in the just
assistance of the Almighty.“(Abs. 2 S. 612), „ [...] where it pledges itself to give you success before
God and Heaven.“(Abs. 3 S.612) , „[...] nay even every case of lack of sufficient zeal in our holy
cause.....“(Abs. 4 S.612) und „This being the first day of open resistance, the commencement of the
sacred combat [...].“(Abs. 5 S. 612)

Neben der religiösen Komponente sind volksnahe Elemente Bestandteil des Stils. So werden nicht
selten Superlative oder starke Ausdrücke verwendet: „[...] carrying away many thousands of its
bravest and most strenuous defenders,[...]“(Abs. 1 S. 611) oder „[...] have sworn to cast away the
abhorred yoke or to die,[...].“(Abs. 2 S. 612)

Es werden im Manifest einige heute kaum noch gebräuchliche Phrasen verwendet. Beispielsweise
werden die Russen „Muskoviter“ genannt. Es beinhaltet eine Anspielung auf die vorherrschende
Kraft bei der Entstehung Russlands, nämlich das Fürstentum Moskau. In diesem Fall ist es aber vor
allem verächtlich gemeint. Eine andere damals in Polen verbreitete Ansicht über die Russen wird im
letzten Satz aufgegriffen: „[...] the last fight of European civilization with Asiatic barbarity.“(Abs. 7
S.612). Dies entspringt zum Teil aus der Vergangenheit Russlands unter der Knechtschaft der
Mongolen. Eine andere Bezeichnung ist „Westrussen“. Darunter sind Weißrussen und Bewohner
Rutheniens (westliche Ukraine) gemeint.

Im Manifest gibt es weitere Ausdrücke, die einer Erklärung bedürfen, da sie sich auf vergangene
Ereignisse beziehen: „[...] therefore we forgive you the murder of our country, the blood of Praga
and Oszmiana, the violence in the streets of Warsaw, the tortures in the dungeons of the citadel:
[...].“(Abs. 7 S. 612) Zuerst wird Bezug genommen auf die stetige Teilung Polens unter den
benachbarten Großmächten Russland, Preußen und Österreich. Dabei sollte man erwähnen, dass die
Teilung hauptsächlich von Russland ausging und es ist sicher auch kein Zufall, dass es den
verhältnismäßig größten Teil Polens bekam, bekannt als Kongress Polen. Weiter wird ein der dritten
Teilung zeitlich benachbartes Ereignis erwähnt, nämlich „die Schlacht um Praga“ 1794. Praga selbst
ist ein Stadtteil Warschaus. Damals war es Schauplatz eines brutalen Massakers an der Bevölkerung
durch die russischen Truppen. Allerdings ist mit diesem Stadtteil auch noch ein anderes Ereignis
verbunden, nämlich mit dem Aufstand 1830/31. Damals bedeutete der Sieg über die Aufständischen
in diesem Stadtteil das Ende des Aufstandes.
%TODO: Hier Anmerkung, woher stammt diese Info

Mit diesem sogenannten „Novemberaufstand“ 1830 hat auch der Name Oszmiana zu tun. Es ist der
Name einer Stadt in Weißrussland, die während des Aufstandes von den russischen Truppen
niedergebrannt wurde. Weiter wird die Gewalt in den Straßen von Warschau aufgezählt. Das ist eine
Anspielung auf die dem Januaraufstand vorangehenden Demonstrationen. Diese wurden von
russischen Truppen blutig aufgelöst. Manche von den Demonstranten wurden auch verhaftet und in
die Zitadelle gebracht, d. h. in das russische Verwaltungsgebäude.

Notwendig zu erklären ist wohl noch die Phrase „Your sons have also been dangling on gibbets, or
have found a frosty death like our own people in the snows of Siberia.“(Abs. 7 S. 612). Das bezieht
sich auf die gängige Praxis im Russischen Reich politisch unbequeme Leute oder Aufständische
weit weg nach Sibirien zu schicken, wo die Verbannung häufig gleichbedeutend mit dem Tod war.
Die letzten Absätze dieses Kapitels bestätigen, dass sich dieses Manifest an die polnische
Bevölkerung richtet. Schließlich werden hier nationale Ereignisse nur mit dem Namen erwähnt,
wohl aus dem Wissen heraus, dass diese Begriffe im Gedächtnis besonders aussagekräftig
konnotiert sind.

\section{Quelleninterpretation}
Als wichtiger Auslöser der Veröffentlichung des Manifestes nicht nur im thematischen sondern auch
im zeitlichen Sinn, ist die bereits im ersten Absatz der Quelle beschriebene Konskription der
Wehrpflichtigen. „The contemptible government of the invaders, rendered furious by the resistance
of the victim that it tortures, has determined to strike a decisive blow by carrying away many
thousands of its bravest and most strenuous defenders,[...].“(Abs. 1 S. 611). Dieser Konskription
(=Zwangsaushebung) wird in der Quelle eine entscheidende Bedeutung zugemessen: „decisive
blow“. Unter diesem „entscheidenden Schlag“ versteht man die am 15. Januar erfolgte Einziehung
der Männer. Die Konskription in dieser Art abzuhalten war aber die Idee eines Polen, nämlich Graf
Wielopolski, der mit seiner Ausrichtung nach Russland nicht alleine war. \footnote{
R.F. Leslie weist darauf hin, dass die Konskriptionen in der Vergangenheit vor allem die landlosen Bauern betrafen.
Diesmal war die Überlegung von Wielopolski die Jugend in den Städten einzuziehen. 
Siehe \fullcite[154]{leslie1963}}
Wielopolski war seines
Zeichens der Chef der zivilen Verwaltung Polens. Wie man daran sieht, war die politische
Landschaft im Polen der 1850er und 1860er keineswegs einheitlich. Die Bevölkerung war in ihrer
Ablehnung gegen die russische Herrschaft nicht geeint.

Diese Teilung der Gesellschaft begann sich in Polen besonders in der 1830er Jahren zu vertiefen.
Bestimmende Faktoren dabei waren die ausbreitende Macht des Adels, gleichzeitig mit dem
Aufkommen der Industrie, des Bürgertums und der kapitalistischen Bourgeoisie. Zudem
verschlechterte sich die Situation der Bauern durch das repressive Regime der russischen Autokratie; 
es begünstigte häufig die Adeligen.\footnote{
Siehe \fullcite[51]{gentzen1958} und \fullcite[16]{kowalski1954}
} Infolge des Krimkrieges und der Abschaffung der
Zollbarrieren zum russischen Reich kam es zu einem Aufschwung in der Wirtschaft.\autocite[79--81]{gentzen1958} Trotzdem war
die Situation für die gesamte Bevölkerung, mit Ausnahme der Gutsbesitzer, unbefriedigend. Die
Vereinigung Italiens 1858 hatte für viele Polen einen Vorbildcharakter.\autocite[52]{gentzen1958} 
Besonders, dass dabei
Personen aus höheren Rängen zusammen mit dem Volk federführend waren, stieß in Polen auf
Begeisterung. Infolge der Bauernreform 1861 im Zarenreich entstanden viele Aufstände, denn
die Aufhebung der Leibeigenschaft und die Verbesserung der Stellung der Bauern – wesentliche
Punkte der Bauernreform von Alexander II.– waren für Kongress Polen nicht vorgesehen. Die
Dringlichkeit in dieser Sache etwas zu tun, wurde innerhalb mehrerer Demonstrationen
ausgedrückt. Diese wurden aber von Soldaten gewaltsam zerstreut – mit Toten unter den
Demonstranten. Die Folge war eine weitere Verbreitung der Unruhen. Das Entgegenkommen des
Zaren in der schrittweisen Entfernung der Repressalien wirkte nicht mindernd, sondern fördernd auf
die revolutionäre Bewegung. Zu dieser Zeit wurden mehrere politische Richtungen deutlich: Auf
der einen Seite standen viele Großgrundbesitzer und Adelige, unter ihnen o.g. Wielopolski (seines
Zeichens Graf), die sich gegen die Reformierung der gesellschaftlichen Ordnung wehrten und die
Unabhängigkeit fürchteten – aus Sorge ihren Einfluss auf die Bauern, ihre wirtschaftliche Existenz
und auch die Unterstützung der russischen Truppen bei Aufständen zu verlieren. Im Vergleich zur
„weißen“ Gruppe war diese eindeutig an Russland orientiert. Eine andere Seite repräsentierten
die „Weißen“. Dabei organisierten sich mittlere Großgrundbesitzer, die Bourgeoisie sowie
insgesamt diejenigen, denen ein vollkommener Umsturz der alten Ordnung nicht hilfreich sein
konnte. Diese Gruppierung wollte aber trotzdem einige gemäßigte Reformen einleiten, um die
Wirtschaft dem aufkommenden Kapitalismus anzupassen. Außerdem entstand in den Jahren vor
dem Januaraufstand eine linke Gruppierung, die „Roten“ oder auch „Demokratisches Lager“
genannt. Darunter waren folgende Gruppen zu finden: „Arbeiter, Handwerker, Stadtarmut, ein
gewisser Prozentsatz Bauern, Intelligenz, Kleinbürgertum, verarmter Adel, kleine und gewiß auch
so mancher mittlere Gutsbesitzer“\autocite[18]{kowalski1954} .

Welches Gedankengut ist in der Quelle enthalten? Eindeutig kann man das nicht beantworten. Die
Quelle ist zwar von der „roten“ Partei ausgearbeitet worden, doch die Grenzen zwischen den
Parteien waren nicht konkret, sodass es sogar personelle Überschneidungen gab.\autocite[19]{kowalski1954} Die Quelle ist
daher auch weder eindeutig dem rechten Lager („Weiße“) noch dem linken Lager („Roten“)
zuzuordnen. Das betrifft die wohl wichtigste inhaltliche Passage, wo es um die soziale
Reformierung der Gesellschaft geht: „This being the first day of open resistance, the
commencement of the sacred combat, the committee proclaims all the sons of Poland free and
equal, without distinction of creed and condition. It proclaims further that the land held heretofore
by the agricultural population in fee, for corvée labor or for rent, becomes henceforth their freehold
property without any restriction whatsoever. The proprietors will receive compensation from the
public treasury.“(Abs. 5 S. 612)

Zunächst ist die Rede von der Gleichstellung der Menschen in Polen. Dies ist etwas, was durchaus
dem Charakter der linken Bewegung entsprach. Gerade die Leibeigenschaft war zum Beispiel eine
gewaltige soziale Bürde, was wohl mit dem Wort „condition“ gemeint ist. Das Wort „creed“ bezieht
sich vor allem auf die ausgegrenzte gesellschaftliche Situation der Juden. Dass die Bauern ihr Land
von den Gutsbesitzern übereignet bekommen, trägt auch zur linken Färbung des Inhalts bei. Die
konservative Seite drückt sich wiederum aus in der Entschädigung für die Gutsbesitzer. Dies scheint
ein Versuch zu sein, eine Brücke zu der ideologischen Haltung der Gutsbesitzer zu bauen. Hier liegt
auch die Schwachstelle im Manifest. Schließlich kann man beide Seiten nicht so einfach
zufriedenstellen. Die Gutsbesitzer werden zwar vom Staat entschädigt, aber die Steuern dafür
müssen wiederum die Bauern zahlen. Insofern kann die angekündigte Abschaffung der Bezahlung
„in fee, for corvée labor or for rent“ unter Umständen auch als eine veränderte Ausdrucksweise für
eine weiterhin bestehende Beziehung interpretiert werden. Dieser Sachverhalt schwächt die Reform
ab. Gleichzeitig ist die Reform aber auch nicht geeignet, um die Gutsbesitzer auf die Seite der
Aufständischen zu ziehen. Denn hier wird nicht angegeben in welcher Größenordnung und nach
welchem Rechenschlüssel die Entschädigung erfolgen wird. Mit solchen Entschädigungszahlungen
gab es bei vorherigen Reformen bereits große Schwierigkeiten, einen gemeinsamen Nenner zu
finden.\footnote{
Am 16. Mai 1861 verkündete ein Erlass die Aufhebung der Frondienste und Einführung eines Geldzinses, der durch
Bodenklasse unterschieden wurde. Dieser war aber sehr kompliziert zu berechnen und deswegen kaum genutzt.
\autocite[82]{gentzen1958}}
Die Gutsbesitzer mussten außerdem fürchten, dass ihnen ohne Schutz der russischen
Truppen der Galgen droht\footnote{In Galizien hatten die Bauern 1846 die Gutsbesitzer aufgehängt oder ohne Entschädigung vertrieben.
\autocite[82]{gentzen1958}} 
und die Bauern hatten eine von Grund auf misstrauische Einstellung zu
den Gutsbesitzern oder Adeligen.\autocite[4]{kowalski1954}

Die Unentschlossenheit im politischen Standpunkt des Manifestes merkt man ebenfalls im nächsten
Absatz. „To arms, therefore, you Poles, you Lithuanians, and you [West] Russians. The hour of our
common deliverance has struck; the ancient sword is drawn from the scabbard; the sacred flag of
our common country is unfurled.“(Abs. 6 S. 612)

Die Unklarheit gründet sich darauf, aus welchem Grund die Litauer und die Westrussen kämpfen
sollten. Aus einem gemeinsamen Zusammengehörigkeitsgefühl? Das Manifest versucht diese
Beziehung zumindest herzustellen: „common deliverance“ und „the sacred flag of our common country“.
Diese Gebiete gehörten tatsächlich vor den Teilungen zu Polen. Worauf in diesem Absatz
möglicherweise Bezug genommen wird, ist die Herstellung der Landesgrenzen von 1772. Solche
Gedanken waren besonders in politisch rechts stehenden Kreisen verbreitet und die Formulierung
„ancient sword“ weist im Grunde auf die Vergangenheit hin.\autocite[25]{kowalski1954} Erwähnt werden sollte, dass auch
linke Kreise in Fragen der nationalen Unabhängigkeit Litauens und Weißrusslands teilweise einen
konservativen Standpunkt einnahmen.

Im Manifest sollte man sein Augenmerk auch darauf richten, was nicht gesagt wird: wie soll
nämlich mit denen umgegangen werden, die kein Land besitzen. Es ist insofern eine kritische Frage,
weil in Polen diese landlose Klasse von Bauern zahlenmäßig fast die gleiche Größe hatte wie die
Bauern mit Land.\autocite[15]{kowalski1954} Diese Gruppe wird aber hier übergangen. Es wird lediglich denen Land
versprochen, die auf Seiten der Aufständischen kämpfen: „All cottagers and laborers who shall
serve the families of those who may die in the service of their country will receive allotments from
the national property in land regained from the enemy.“. Dieser vielsagender Satz weist nochmals
darauf hin, dass die Verfasser des Manifestes es einerseits allen recht machen wollten, andererseits
aber kaum die grundlegenden Eigenschaften der einzelnen Gruppen zu berücksichtigen wussten.

\newpage
\closuresection{Schlussbetrachtung}

Die Quellenbeschreibung hat Aufschluss darüber gegeben, dass wesentliche Elemente der stilistischen und
in Teilen auch der inhaltlichen Aussagen eindeutig an das polnische Volk adressiert sind. Betonen
muss man, dass dieses Manifest sich nur auf den zu Russland gehörenden Teil Polens bezieht.
Gegenüber Preußen oder Österreich werden nämlich keine Vorwürfe erhoben. Wie man am Verlauf
des Aufstandes sehen kann, war das wohlkalkuliert gewesen. Die Lieferungen von
Waffen und Proviant wurden zum großen Teil über die umliegenden Grenzen zum Großherzogtum
Posen und Galizien organisiert.

Der Charakter des Manifestes ist religiös und volksnah. Die vielfältigen Anspielungen auf Gott und
die nur mit dem Namen erwähnten Ereignisse bezeugen das. Am Datum und dem Anlass kann man
weiterhin sehen, dass der Aufstand erwartet wurde. Aus der Kombination des Anlasses und der
Datierung des Manifestes lässt sich ebenfalls ersehen, dass die Initiative nicht bei den
Aufständischen lag, sondern in den Händen der russischen Verwaltung und von Graf Wielopolski.
Sonst wäre der Aufstand nicht zu dem denkbar schlechtesten Zeitpunkt, nämlich mitten im Winter,
ausgerufen worden. Die jungen Männer, die vor der Zwangsaushebung in die Wälder geflüchtet
waren, konnten nach dem 15. Januar nicht allzu lange bei der Kälte in den Wäldern bleiben. Von
daher war der Beginn des Aufstandes gewissermaßen von anderen diktiert.

Die Resultate des zweiten Teils knüpfen daran an, dass in der polnischen Gesellschaft verschiedene
Interessen herrschten. Für einen nationalen Aufstand wurden aber alle gebraucht: Die finanzielle
Unterstützung der Reichen und die Bauern als Kämpfer. An dieser Kombination scheitert das
Manifest auch. In der Forschungsliteratur wurde die Ursache folgendermaßen formuliert: „Die
Quelle für diese gemäßigte Form des sozialökonomischen Programms des Zentralen National-Komitees, das einen Kompromiß zwischen den Ansichten der Linken und der Rechten im Lager der
Roten darstellte, war [...] eine falsche Konzeption der nationalen Front, der nationalen Einheit, in
welche die Demokraten des Jahres 1863 sämtliche Klassen und Schichten der Gesellschaft ohne
Ausnahme, also auch den gesamten Adel einbeziehen wollten ...“.\autocite[23]{kowalski1954}

Daher resultiert zum Beispiel auch die Einbeziehung Litauen und Weißrusslands in das Manifest,
denn vor allem der Adel war in diesen Gebieten dominant. Nicht zuletzt ist das Manifest aus einer
höheren Position an die Bevölkerung gerichtet. Die Reform hat deswegen den Charakter als ob sie
von oben dirigiert wird und nicht aus einer Bauernbewegung entstanden ist.

Alles in allem merkt man in der Quelle, wieso es nur einer von weiteren Versuchen war, „die nur zu
einer halben Lösung führten, und die, während sie den Bauern ein wenig zugestanden, zugleich die
Großgrundbesitzer nicht kränken und abstoßen sollten.“\autocite[23]{kowalski1954}

Der mit diesem Manifest beginnende Aufstand dauerte ca. bis zum Sommer 1864. Die
Organisatoren hatten sich in vielem verschätzt: Es gab keine Intervention der westlichen Mächte.
Die von ihnen gerufenen Feldherren haben sich als erfolglos entpuppt. Weder beim Adel noch bei
den Bauern stieß dieser Aufstand auf große Teilnahme und die umliegenden Gebiete in Posen und
Galizien haben durch die Alvenslebensche Konvention ihren Nutzen verloren.
%TODO: Alvenslebensche Kon kurz erklären
Für die Russen war es deswegen auch ein Leichtes den Aufstand zu zerstreuen. Dazu war nur eine
Reform der Landverhältnisse zu Gunsten der Bauern notwendig, die vom Russischen Reich im
März 1864 im Kongress Polen inauguriert wurde.

\literature
\end{document}
