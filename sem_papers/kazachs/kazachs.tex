\documentclass{../../sem_paper}
\addbibresource{kazachs.bib}

\begin{document}
\titlepg{Die Unabhängigkeit Kasachstans 1985-1991}
{2012}{Fachbereich Geschichte}
{Übung 54-242 "`Geschiche und Identität in zentralasiatischen Filmen 1985-1991"'}
{WS 2011/2012}
{Moritz Florin}

\tocpaper

\closuresection{Einleitung}

Es gibt zur Zeit wohl kein Land in Zentralasien, dem eine größere Aufmerksamkeit
geschenkt wird, als Kasachstan. Die Gründe dafür sind vielfältig: Kasachstan ist sehr
rohstoffreich, besonders in Hinsicht auf Öl und Gas. Weiterhin hat sich das Land unter
Nursultan Nasarbaev zum wohlhabendsten in Zentralasien entwickelt. Auch könnte
man noch die politische Stabilität von Kasachstan anführen, besonders da andere
Staaten wie Kirgisistan, Usbekistan und Tadschikistan politische Umbrüche vollzogen
haben oder vollziehen. In der Gegenwart lässt sich also durchaus von einer besonderen
Rolle Kasachstans in Zentralasien sprechen. Doch auch in der jüngeren Vergangenheit
(seit den 1980ern) lässt sich eine besondere Rolle zusprechen.

Kasachstan spielte in der Sowjetgeschichte eine einzigartige Rolle: So
fingen in der Hauptstadt Almaty im Jahr 1986 die ersten Proteste gegen die
Sowjetmacht an; die Kasachische SSR (Sozialistische Sowjetrepublik) war die einzige
Republik, deren namensgebende Ethnie in Unterzahl war. Es war auch die letzte
Republik im Sowjetgebilde, die ihre Unabhängigkeit erklärte. Besondere Bedeutung
erlangte das Treffen von Präsidenten der sowjetischen Republiken am 21. Dezember
1991 in Almaty. Hier wurde mit dem Unterzeichnen der Verträge, die Gründung der
GUS als Nachfolgerin der Sowjetunion beschlossen und damit war die Sowjetunion
formell aufgelöst.

Die vorliegende Ausarbeitung zum Thema „Die Unabhängigkeit Kasachstans“ soll
aufzeigen, welche Entwicklungen in der Sowjetunion zur Unabhängigkeit des Landes
führten. Es sollen dabei größere Zusammenhänge beschrieben werden, wie die Lage der
Sowjetunion in den 1980ern und die Reformprojekte Perestroika und Glasnost'.
Außerdem sollen bestimmte Entwicklungen in Kasachstan dargestellt werden. Diese
Darstellung orientiert sich an den Dezember-Unruhen in Almaty 1986 sowie an den
Argumenten, mit denen Kritiker und Befürworter einer Unabhängigkeit ihren
Standpunkt vertraten.

\npsection{Zeit vor der Perestroika in der UdSSR}
Der Auflösung der UdSSR 1991 gingen mehr als zwei Jahrzehnte wirtschaftlicher
und politischer Stagnation voraus.
\subsection{Politische Entwicklung}
Als Leonid Brežnev 1964 dem abgesetzten Nikita Chruščëv auf den Posten des
Parteivorsitzenden folgte, waren schon Anzeichen einer Stagnation sichtbar, die sich in
Brežnevs Amtszeit manifestierte. Nach seinem Tod 1982 wurde die Unentschlossenheit
der sowjetischen Führungsriege deutlich, da man sich nur auf Übergangslösungen
einigen konnte: Auf Jurij Andropov, der im Jahr 1984 nach etwas mehr als einem Jahr an der
Macht starb, und seinen Nachfolger Konstantin Černenko, der auch nur ungefähr ein
Jahr regierte und starb.

\subsection{Probleme}
An der prekären Lage der Sowjetunion konnten diese beiden nichts ändern. Denn die
Herausforderungen waren groß: Zum einen war die Sowjetunion kurz davor, im Kalten
Krieg den Anschluss zu verlieren, da die USA ihre Militärausgaben auf ein Niveau
steigerten, das sich die Sowjetunion nicht leisten konnte. Der Krieg in Afghanistan
verbreitete immer mehr Kriegsmüdigkeit in der Bevölkerung und war finanziell eine
große Belastung für den Haushalt der Sowjetunion. Die Entwicklung der Landwirtschaft
blieb hinter dem Wachstum der Bevölkerung zurück \autocite[117]{kogel2007}. Das Bevölkerungswachstum
betraf dabei besonders die zentralasiatischen Gebiete und die muslimische
Gesellschaft \autocite[47]{zaslav1991} . Das Wohlstandsgefälle zum Westen wurde außerdem immer größer und
es fiel zunehmend schwerer, den Vorsprung des Westens in der sowjetischen
Gesellschaft auszublenden.

\section{Zeit nach Gorbačëvs Amtsantritt}
Nach dem Tod von Černenko 1985 wurde die Position des Parteivorsitzenden der
KPdSU vakant. Nach den Regierungsjahren von Andropov und Černenko
wurde nun ein perspektivvoller Politiker gesucht. Die Wahl zum neuen Mächtigen in der
Sowjetunion fiel auf das jüngste Mitglied im Zentralkomitee: Michail Gorbačëv.

\subsection{Angestoßene Reformen}\footnote{Hier nur eine knappe Darstellung, genauer \fullcite[206-267]{segbers1989} und \fullcite[30-65]{simon1993}} 
Zur Reformierung der Sowjetunion prägte Michail Gorbačëv zwei Schlagworte: Glasnost' und
Perestroika. Glasnost' sollte der Presse und der Bevölkerung die Möglichkeit geben, Aufklärung
über die Korruption und die in der Sowjetunion herrschenden Probleme zu betreiben
und auch Geschichtsfälschungen sollten zum Vorschein kommen.

Perestroika, das zweite Schlagwort, war vor allem auf gesellschaftliche und
wirtschaftliche Reformierung der Sowjetunion bezogen. Teile der Staatsbetriebe sollten
marktwirtschaftlicher orientiert werden.

\subsection{Auswirkungen auf die Republiken}
Die Reformen von Gorbačëv machten ihm ältere Parteigenossen und insbesondere
jene zu Feinden, die unter Brežnev zu ihren Posten gekommen waren. Außerdem
wurden immer mehr nationale Bestrebungen von den einzelnen Republiken artikuliert.
Den Beginn einer unruhigen Periode markierten die Dezember-Unruhen in Almaty
1986.

\subsubsection{Almaty 1986}
Für den Reformkurs brauchte Gorbačëv in den einzelnen Republiken Unterstützung
und löste altgediente Parteivorsitzende durch ihm bekannte Leute ab. Zum Beispiel
wechselte er in der Kasachischen SSR den langjährigen, kasachisch-stämmigen
Vorsitzenden der Partei Kunaev aus und stellte stattdessen einen Mann aus Russland an
die Spitze der Parteileitung, Gennadij Kolbin.

Nach der Bekanntgabe dieses Personalwechsels am 17. Dezember 1986
versammelten sich rund 100 Kasachen vor dem Gebäude des Zentral-Komitees in
Almaty, um zu protestieren. Einige Stunden später waren es einige Tausend, die gegen
diese Ernennung mit Slogans protestierten wie „Wir brauchen einen Kasachen als
Leiter“ oder „Jedem Volk seinen eigenen Leiter“ \autocite{trut1994} . Später kam es zu Zusammenstößen
mit den Sicherheitskräften. Die Auseinandersetzungen dauerten die Nacht über bis zum
nächsten Morgen des 18. Dezembers. Bei den Zusammenstößen wurden viele Gebäude
verwüstet, über 1000 Menschen verletzt und 3 Menschen getötet. Annähernd 100
Kasachen wurden zu Haftstrafen und einer zum Tode verurteilt.

\subsubsection{Nationale Bestrebungen in der Sowjetunion}
In den folgenden Jahren war die Lage in Kasachstan ruhiger, sodass Kolbin drei
Jahre bis 1989 Parteisekretär der Kasachischen SSR war, aber in dieser Zeit innerhalb der
örtlichen Hierarchie keine Unterstützung fand und abgelöst wurde durch Nursultan
Nasarbajev. In Kasachstan blieb die Lage ruhig; an der Peripherie der
Sowjetunion wie in Georgien oder den baltischen Gebieten fanden in den nächsten
Jahren dagegen ähnliche Proteste statt.

Die Unruhen wurden durch Reformen der Perestroika und Glasnost' verursacht, die
für eine Instabilität in der Sowjetunion sorgten. Durch die neue Freiheit in der
Meinungsäußerung wurden beispielsweise mehr nationale Bestrebungen aus den
Sowjetrepubliken laut. In Zentralasien fehlten solche separatistischen Strömungen
größtenteils \autocite[43]{zaslav1991}. Auch in Almaty 1986 wurde nur gegen das zentralistische System
protestiert, nicht gegen die Sowjetunion an sich. Die stärksten Verfechter einer
nationalen Souveränität waren die baltischen Republiken. Doch auch in der Russischen
SFSR setzte sich Boris Jelzin für nationale Eigenständigkeit ein.

\section{Unabhängigkeit Kasachstans 1991}
\subsection{Ende der Sowjetunion im Jahr 1991}
Dass Persönlichkeiten wie Jelzin mehr Einfluss bekamen, war der schlechten
wirtschaftlichen Entwicklung geschuldet. Stagnierte die Sowjetunion zu Anfang der 1980er
, so waren 1990/91 viele wirtschaftliche Indikatoren im Fallen. Die
Staatsbetriebe waren nicht mehr in der Lage Alltagsgüter wie Rasierklingen oder Hefte
zu liefern. In der Bevölkerung ging die Angst vor einer Hungersnot um \autocite[130]{neef2007}. Im August
1991 wurde ein Putsch von einigen Hardlinern unternommen, die Gorbačëv nicht mehr
unterstützen und die Reformen rückgängig machen wollten. Der misslungene Putsch
ermöglichte es dem Präsidenten der Russischen Republik Boris Jelzin, die
Kommunistische Partei zu verbieten. Gorbačëvs Einfluss als Vorsitzender der KPdSU
reichte in der Folgezeit nur noch aus, um mit den Mitgliedsstaaten eine lockere Bindung
auszuhandeln, in der Form der GUS, die Ende 1991 in Almaty gegründet wurde.

Kasachstan hat in den Jahren 1990 und 1991 bestimmte Schritte vollzogen, die es der
Unabhängigkeit näher brachten. Der sowjetische Rat wählte Nasarbajev im April 1990
zum Präsidenten. Im Oktober 1990 proklamierte Kasachstan die Souveränität. Im
Dezember 1991 wurde Nasarbaev vom Volk als Präsident bestätigt, der Name KSSR
wurde in Republik Kasachstan umbenannt und am 16. Dezember rief Kasachstan seine
Unabhängigkeit von der Sowjetunion aus.

In Kasachstan war die Stimmung in Bezug auf die Sowjetunion verschieden. Der
Großteil der Bevölkerung war gegen die Auflösung der Sowjetunion. Viele wollten aber
auch keinen allmächtigen Zentralstaat, der gewissermaßen über die Köpfe der
Menschen hinweg entschied: Nasarabjev beispielsweise kritisierte Gorbačëv 1990, weil
er einen Vertrag mit Chevron über die Ausbeutung eines Ölfeldes in Kasachstan
abschließen wollte ohne die kasachische Seite zu konsultieren.

\subsection{Argumente für die Unabhängigkeit}
Die Reformen von Gorbačëv ermöglichten eine Aufarbeitung der Vergangenheit und
Veröffentlichung kritischer Texte in Bezug auf die Rolle Kasachstans in der
Sowjetunion. Durch den Reaktorunfall in Tschernobyl wurde die Aufmerksamkeit der
Öffentlichkeit auf die Lage der Umwelt gelenkt. Durch Glasnost'
wurde aufgezeigt, wie stark der Fortschritt der Sowjetunion zu Lasten der Ökologie
Kasachstans ging. Folgende Argumente wurden gegen die Sowjetunion benutzt \autocite[60]{trut1994}:
\begin{enumerate}
 \item Atomwaffentests in der Nähe von Semey/Semipalatinsk – der Geburtsstadt des
kasachischen Nationaldichters Abay
\item Das Raumfahrtprogramm in Baikonur schadete durch die umfangreiche
Infrastruktur der Umwelt
\item Eines der größten Binnengewässer der Welt, der Aral-See, zeigte in den 1980ern
schon deutliche Anzeichen der Austrocknung. In der Uferstadt Aralsk traten sogar Fälle
von Pest auf
\item Die Neulandgewinnung, die in 50ern und 60ern sehr viele Siedler in die
kasachische Steppe führte. Dies veränderte nicht nur die gesellschaftliche Struktur,
sondern auch die Balance der Ökologie Kasachstans 
\end{enumerate}

\subsection{Argumente gegen die Unabhängigkeit}
Die Stimmung in den Führungspositionen war aber zu Gunsten des Fortbestandes der
Sowjetunion. Besonders Nasabajev ließ sich von einem wirtschaftlichen Pragmatismus
leiten und führte folgende Gründe an \autocite[62]{trut1994}:
\begin{enumerate}
 \item Das Titularvolk der Kasachen bildete nur die Minderheit. Die Mehrheit waren
damals russisch-stämmige Einwohner, besonders im Norden Kasachstans. Es wurde
befürchtet, dass im Falle der Auflösung der Sowjetunion die Grenzen nicht
aufrechterhalten werden könnten

\item Fast 90 Prozent der Industrieerzeugnisse kamen aus der Russischen Republik,
wohin Kasachstan vor allem Rohstoffe exportierte. Ein Wegfall des gemeinsamen
Marktes hätte eine Krise ausgelöst
\item Die Kasachen selbst waren in der Industrie kaum beschäftigt, sie waren eher in
der Landwirtschaft tätig. Eine mögliche Auswanderung von gut qualifizierten Russisch-
stämmigen hätte den industriellen Sektor stark geschwächt
\item Die Unabhängigkeit Kasachstans hätte eine nationalistische und islam-nahe
Stimmung begünstigt und so Instabilität verursachen können
\\
\\
Unter diesen Vorzeichen erklärte Kasachstan am 16. Dezember 1991 seine
Unabhängigkeit. Damit schloss sich gewissermaßen ein Kreis, denn fast bis auf den Tag
genau fand fünf Jahre zuvor die erste größere Erhebung in der Sowjetunion statt.
Kasachstan hatte gewichtige Gründe in der Sowjetunion zu bleiben, doch war dies nicht
mehr in der Hand von den Regierenden in Kasachstan. Denn alle anderen Länder hatten
ihre Unabhängigkeit von der Sowjetunion erklärt und Kasachstan folgte dem als letztes
Land. Es sei am Rande bemerkt, dass nicht nur in Kasachstan die Stimmung in der
Bevölkerung mehrheitlich für den Erhalt der Sowjetunion war, sondern auch in den
meisten anderen Sowjetrepubliken.
\end{enumerate}

\closuresection{Schlussbetrachtung}
Zusammenfassend lässt sich aus der vorliegenden Arbeit sagen, dass die
Unabhängigkeit
Kasachstans
keineswegs
getrennt
von
der
Entwicklung
der
Sowjetunion zu sehen ist. Im Zentrum inaugurierte Reformen der Perestroika und
Glasnost' hatten eher negative Auswirkungen auf die Stabilität der Sowjetunion.
Bemerkenswert ist in dieser Arbeit aber, dass es einen interessanten Umschwung
gegeben hat: 1986 gehörten die Kasachen zu den ersten, die gegen die Sowjetunion
protestierten. 1991 war Kasachstan dagegen die letzte Republik, die ihre
Unabhängigkeit ausrief. Dies ist ein weiterer Beweis für eine gewisse Besonderheit
Kasachstans in der Geschichte der Sowjetunion.

\literature
\end{document}
