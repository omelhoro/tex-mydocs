\documentclass{../../sem_paper}
\addbibresource{panslavism.bib}

\begin{document}

\titlepg
{Der Panslawismus in der ersten Hälfte des 19. Jahrhunderts}
{2012}
{Fachbereich Geschichte}
{Einführungseminar II "`Russischer Imperialismus"'}
{SS 2011}
{Kristina Küntzel-Witt}
\tocpaper

\closuresection{Einleitung}

\subsection*{Panslawismus in Gegenwart und Geschichte}
Slovio – ein Wort, das Zeuge eines Phänomens namens Panslawismus ist, welches vor
ungefähr 200 Jahren entwickelt wurde und das man heute vielleicht zur Geschichte erklären
würde. Aber gerade in Slovio, einer Plansprache auf Basis der slawischen Sprachen, findet
sich der panslawische Gedanke in der Gegenwart. Dabei ist Slovio nur ein Beispiel; Berger\autocite{berger} findet sogar ganze Communities, die ein panslawisches Gedankengut zum Inhalt
haben.

Der Panslawismus ist tatsächlich seit seinem Ursprung am Anfang des 19. Jahrhunderts
ununterbrochen in der Geschichte Osteuropas präsent. Beispielsweise im 20. Jahrhundert:
In den 90er Jahren lösten sich zwei panslawische Staaten auf, nämlich die Tschechoslowakei
und Jugoslawien.

Dabei sind die genannten Staaten nicht die einzigen Träger eines panslawischen Gedankens
im 20. Jahrhundert gewesen, auch die Sowjetunion übernahm im 2. Weltkrieg die
panslawische Idee, wenngleich diese nicht sehr lange verwendet wurde, da der Anspruch der
ethnischen Einheit kaum mit der sowjetischen Ideologie vereinbar war.\autocite[9]{troebst}
Doch auch zu Anfang des 20. Jahrhunderts ist der Panslawismus im Zusammenhang mit dem
1. Weltkrieg als möglicher Auslöser genannt worden, wie man an der 1915 erschienenen
Monographie vom Österreicher Richard Charmatz mit dem Titel „Zarismus, Panslawismus,
Krieg!“ sieht. Man müsste den Titel natürlich im Kontext begreifen, dass Österreich im Krieg
mit dem Zarenreich stand. Es zeigt sich aber, dass die panslawische Idee im gesamten 20.
Jahrhundert relevant war. Nicht zuletzt könnte man am Vorabend des 1. Weltkrieges die
Existenz eines Neoslawismus in Österreich für eine bestehende slawische Identität anführen
wie es Vyšný\autocite{vyshny} tut.

Noch bedeutender war diese Idee im 19. Jahrhundert. Um sich die Bedeutsamkeit zu
verdeutlichen, könnte man auf den Konflikt zwischen dem Russischen und dem Osmanischen
Reich 1877/78 schauen, welcher zu einem gewissen Teil durch panslawische Strömungen
ausgelöst wurde. In der zweiten Hälfte des 19. Jahrhunderts könnte man aber auch auf
bestimmte Leute schauen, die in Russland berühmt waren und einem Panslawismus das Wort
redeten; zu nennen sind dabei der Dichter Fedor I. Tjutčev und der weltbekannte Schriftsteller
Fedor M. Dostoevskij. Diese beiden wichtigen Persönlichkeiten vertraten panslawische Ideen.

\subsection*{Gegenstand der Hausarbeit}
Gewissermaßen im „Rückwärtsgang“ komme ich zum eigentlichen Gegenstand dieser
Hausarbeit – der ersten Hälfte des 19. Jahrhunderts. Wenn man die allgemein gebräuchliche
Phrase „Das lange 19. Jahrhundert“ anschaut, ist es nahe liegend, dass man eine Thematik, die
im ereignisreichen 19. Jahrhundert angesiedelt ist, für eine Hausarbeit zeitlich eingrenzt. Ich
habe mich für die erste Hälfte des Jahrhunderts entschieden, weil meiner Meinung nach sich
die panslawische Idee in dieser Zeit am deutlichsten mit ihren Konzepten, Ansichten und
Problemen geäußert hat. Besonders wichtig erschien mir die Entwicklung des Panslawismus
in den außer-russischen Gesellschaften wie der polnischen, kroatischen oder tschechischen
Gesellschaft.

Jedes Volk hatte am Beginn des 19. Jahrhunderts eine einzigartige Stellung. Die Russen waren
zum Beispiel das einzige slawische Volk mit einem eigenen Staat (bis auf die kleinen
Ausnahmen Republik Krakau (bis 1846) und Montenegro), die Tschechen befanden sich innerhalb der liberalen
Bewegung in Europa und die Polen waren gleich unter drei Staaten (davon ein slawischer:
Russland) aufgeteilt und den Status einer Großmacht eingebüßt. Jedes Volk
musste deswegen natürlich seine eigene Deutung von der panslawischen Idee entwickeln –
gleichzeitig
propagierte
aber
der
Panslawismus
eine
einheitliche
Identität
und
Zusammengehörigkeit. Dieses Paradoxon manifestierte sich in der ersten Hälfte des 19.
Jahrhunderts und ist für mich ein weiterer Grund, meine Betrachtung auf den vorgeschlagenen
Zeitraum zu richten.

Es fällt im Übrigen schwer, eine genaue Jahreszahl für die Zäsur zwischen den Hälften zu
bestimmen. Man könnte zum Teil den Prager Slawenkongress 1848 als Kulminationspunkt oder den Anfang des
Krim-Krieges 1854 bzw. die Thronbesteigung von Alexander II. anführen; meine Hausarbeit
baut aber erst zweitrangig auf Ereignissen auf, sondern orientiert sich vorrangig an
bedeutenden Denkern der einzelnen Völker. Dabei ist die Tendenz zu beobachten, dass die
bedeutenden panslawischen Vertreter in den 40er Jahren mit ihren Ansichten einen Höhepunkt
der Popularität erreichten. Die Ideen manifestierten sich gewissermaßen im gesellschaftlichen
Diskurs.

Die Ausarbeitung des Themas habe ich unter verschiedenen Gesichtspunkten verfolgt. Zuerst
war es für mich wichtig, die Grundlage der panslawischen Idee darzulegen und
gewissermaßen einen Zeitraum zu beschreiben als sich diese noch einigermaßen einheitlich
zeigte. Dazu habe ich mich auf Herders Theorie und slowakische Gelehrte als den ersten
Vertretern des Panslawismus konzentriert.

Die dann erfolgten Ausprägungen des Panslawismus habe ich versucht anhand der wichtigsten
Denker zu beschreiben. So sollte der Ursprung der einzelnen Ideen illustriert werden, die
Faktoren, die zur weiteren Entwicklung einer Idee führten und schließlich stand auch die
Frage, wie sich die anderen slawischen Völker zur jeweiligen Ausprägung der panslawischen
Idee verhielten.

\npsection{Grundlage und Ursprung der panslawischen Idee}

\subsection{Herders Ansichten zu den Slawen}
Im Folgenden soll auf die Grundlage der panslawischen Idee referiert werden, die ihren
Ursprung zu einem großen Teil bei Johann Gottfried Herder hatte. Er widmete das 4. Kapitel
seines Buches „Ideen zur Philosophie der Geschichte der Menschheit“ den slawischen Völkern.
Hier soll dieses für den Panslawismus wichtige Kapitel in seinen Grundzügen vorgestellt
werden:

Herder verglich in seinem Werk die Slawen mit den Germanen und stellte die friedfertige
Mentalität der Slawen dem kriegerischen Geist der Deutschen entgegen: „Trotz ihrer Taten
hie und da waren sie nie ein unternehmendes Kriegs- und Abenteuervolk wie die Deutschen;
vielmehr rückten sie diesen stille nach und besetzten ihre leergelassenen Plätze und Länder
[...].“\autocite[279]{herder} Weiter beschrieb er die vielen Gebiete, die von Slawen bewohnt
wurden und betonte die Einheit der Slawen: „In Pannonien wurden sie ebenso zahlreich; von
Friaul aus bezogen sie auch die südöstliche Ecke Deutschlands, also daß ihr Gebiet sich mit
Steiermark, Kärnten, Krain festschloß: der ungeheuerste Erdstrich, den in Europa eine Nation
größtenteils noch jetzt bewohnet.“\autocite[279]{herder} Herder betonte mehrmals den bäuerlichen, aber
auch offenen Charakter der Slawen: „Sie liebten die Landwirtschaft, einen Vorrat von Herden
und Getreide, auch mancherlei häusliche Künste und eröffneten allenthalben mit den
Erzeugnissen ihres Landes und Fleißes einen nützlichen Handel.“\autocite[280]{herder} Des Weiteren
idealisierte er die Slawen im Sinne der Romantik und schrieb, dass sie sich neben der
Wirtschaft den schönen Künsten widmeten: „In Deutschland trieben sie den Bergbau,
verstanden das Schmelzen und Gießen der Metalle, bereiteten das Salz, verfertigten
Leinwand, braueten Met, pflanzten Fruchtbäume und führeten nach ihrer Art ein fröhliches,
musikalisches Leben.“\autocite[280]{herder} Herder machte weiter seine Idealisierung der friedlichen
Slawen kund: „Sie waren mildtätig, bis zur Verschwendung gastfrei, Liebhaber der ländlichen
Freiheit, aber unterwürfig und gehorsam, des Raubens und Plünderns Feinde.“\autocite[280]{herder}
Trotz der Unterwerfung der Slawen durch die Deutschen und unbedeutenden Gegenwart
attestierte Herder den Slawen eine verheißungsvolle Zukunft: „[...] so werdet auch ihr so tief
versunkene, einst fleißige und glückliche Völker endlich einmal von eurem langen trägen
Schlaf ermuntert, von euren Sklavenketten befreiet, eure schönen Gegenden vom Adriatischen
Meer bis zum karpatischen Gebürge, vom Don bis zur Mulda als Eigentum nutzen und eure
alten Feste des ruhigen Fleißes und Handels auf ihnen feiern dörfen.“\autocite[281]{herder}

\subsection{Rezeption}
Neben Herder als den größten Impulsgeber für die Idee der slawischen Einheit, war die Zeit
um 1800 generell von einem anwachsenden Interesse an der Sprache und an den slawischen
Kultur gekennzeichnet. Hier seien Gerhard Friedrich Müller und August Ludwig Schlözer erwähnt. Johann
Christoh Adelung sowie Wilhelm von Humboldt beschäftigten sich mit der allgemeinen
Sprachtheorie, die auf die panslawischen Vordenker gewissen Einfluss hatte.\autocite{osterrieder}

Wie der Historiker H. Kohn\autocite[9]{kohn} aber treffend bemerkt, „hätten diese geistigen Einflüsse
ohne bedeutsame Veränderung in der ökonomischen und kulturellen Stellung der
österreichischen Slawen und der politischen und militärischen Position Rußlands kaum den
Panslawismus erwecken können“. Um die von Herder vorgestellten Ansichten zu rezipieren,
bedurfte es einer Schicht von gebildeten Slawen. Diese gebildete Schicht entwickelte sich am
Anfang des 19. Jahrhunderts zuerst in Österreich bei den Slowaken. Die Slowaken waren in
der Zeit protestantisch und daher war ihre Verbindung zu den mehrheitlich protestantischen
Deutschen enger als bei anderen slawischen Völkern. Deswegen wurden die Slowaken Ján
Kollár, Ján Herkel und Pavel Josef Šafarík nach ihrem Studium in Jena ab den 1820ern zu den
Trägern der panslawistischen Idee.\autocite[1]{orton} Der Begriff Panslawismus geht dabei auf
Herkel zurück, der die Bezeichnung einer Sprachfamilie mit dem griechischen Präfix „pan-“
(zu Deutsch: all-) kombinierte. Man muss aber betonen, dass sich die von den Slowaken
vertretenen Ideen auf die kulturelle Einheit der Slawen zusammen mit Russland bezogen; in
der Folge des wachsenden Pangermanismus und des Nationalismus in Europa fand die Idee
einer volksübergreifenden slawischen Identität besonders bei den kleinen slawischen Völkern
mehr und mehr Anhänger und wurde politisch weiterentwickelt.

\npsection{Ausprägungen der panslawischen Idee}
\subsection{Austroslawismus}
Man kann wohl sagen, dass sich die slawische Idee in Österreich sehr interessant entwickeln
musste, denn innerhalb der österreichischen Grenzen gab es eine Vielzahl von slawischen
Völkern aus den drei Gruppen (Südslawen mit den Kroaten, Serben und Slowenen;
Westslawen mit den Polen, Tschechen und Slowaken, sowie die Ostslawen mit den
Ukrainern), eine Anzahl von slawischen Völkern, die in keinem anderen Land oder Reich zu
finden war. Es ist außerdem wichtig zu erwähnen, dass im österreichischen Vielvölkerreich
die Slawen zusammengenommen die Überzahl bildeten vor allen anderen Nationalitäten wie
den Deutschen und den Ungarn. Diese Erkenntnis machte zuerst der slowenische Gelehrte
Jernej Kopitar, der betonte, dass „die slawische Bevölkerung die relative Mehrheit im
Habsburgischen Reich ausmache“\autocite[61]{hahn} (Hahn 2008, 61). Seine Vorstellungen zur slawischen Idee
waren vor allen Dingen auf den kulturellen Aspekt bezogen und nach Kopitars Meinung sollte
Österreich zum geistigen Mittelpunkt des Slawentums entwickelt werden.

Solche kulturellen Ideen konnten nicht mit der Politik unverbunden bleiben wie Hahn (2008,
60) formulierte: „Politik im ständisch-libertären Verständnis und Identitätswahrnehmung auf
sprachlicher und kultureller Ebene gehörten zusammen“. Denn in Österreich als
absolutistischem Staat war vor allem Deutsch als die Sprache des Bildungs- und
Verwaltungssystem verwendet worden, demgegenüber musste die Bedeutung von slawischen
Sprachen natürlich sinken. Diese Sprachenpolitik weckte bei den Slawen den Eindruck einer
Germanisierungspolitik. Mit dieser war für die slawischen Intellektuellen auch die Gefahr das
Pangermanismus verbunden, einer Bewegung, die, ebenfalls von Herders Philosophie
beeinflusst, den Anschluss des österreichischen Teils des Reiches mit den slawischen Völkern
der Slowenen und Tschechen an ein Großdeutsches Reich proklamierte. Während die Slawen
den
Deutschen
in
der
österreichisch-dominierten
Hälfte
des
Reiches
von
der
Bevölkerungsgröße her überlegen waren, konnte man als Slawe in einem Großdeutschen
Reich nur eine kleine Minderheit sein. Der Austroslawismus wurde von den österreichischen
Slawen daher auch vor allem während der Revolution 1848 vertreten. In diesem Jahr wurden
besonders bei den Deutschen nationalistische Tendenzen deutlich, was sich in der Frankfurter
Versammlung 1848 ausdrückte. Von den Slawen in Österreich wurde diese Versammlung
negativ aufgenommen. Als Wortführer des Austroslawismus profilierte sich der Tscheche
František Palacký. Die Gefahr, dass ein deutscher Nationalstaat mit der Auflösung Österreichs
verbunden wäre, ließ Palacký befürchten, dass die Frankfurter Versammlung die Absicht habe
„Österreich als selbstständigen Kaiserstaat unheilbar zu schwächen, ja ihn unmöglich zu
machen, - einen Staat, dessen Erhaltung, Integrität und Kräftigung eine hohe und wichtige
Angelegenheit nicht meines Volkes allein, sondern ganz Europas, ja der Humanität und
Zivilisation selbst ist und sein muß“\footnote{Palacký zit. n. \autocite[19]{moritsch1996}}. Dieser Aussage
erlangte große Bekanntheit, weil Palacký das Ziel des Austroslawismus als Erster nach
vorangegangenen inoffiziellen Diskussionen deutlich beschrieb – nämlich das Ziel der
Erhaltung des Habsburger Reichs. In weiteren Aussagen führte er aus, dass er sich für Österreich „einen
Grundsatz der vollständigen Gleichberechtigung und Gleichbeachtung aller unter seinem
Zepter vereinigten Nationalitäten und Konfessionen“\footnote{Palacký zit. n. \autocite[19]{moritsch1996}} wünsche.

Palacký formulierte außerdem nicht nur einen Gegensatz zum deutschen nationalistischen
Streben,
sondern beschrieb
den Austroslawismus
als
Gegensatz
zum
russischen
Hegemoniestreben: “Denken Sie sich Österreich in eine Menge Republiken und Republikchen
aufgelöst, -- welch ein willkommener Grundbau zur russischen Universalmonarchie.“
\footnote{Palacký zit. n. \autocite[19]{moritsch1996}}. Mit diesem Credo wurde der Austroslawismus als ein „Modell des Überlebens der kleinen
Völker zwischen Ost und West propagiert“\autocite[IIX]{busek}.

Interessant ist in diesem Zusammenhang das Ereignis des Prager Kongresses im Juni 1848.
Der Kongress war eine Antwort auf die Frankfurter Versammlung. Auf dem Prager Kongress
sollte gezeigt werden, dass die Slawen in Österreich dem pangermanischen Streben der
Deutschen eine eigene, friedliche Bewegung entgegenstellen können. Durch die nationalen
Einigungsbestrebungen wurde die Gefahr eines Zusammenbruchs des österreichischen
Vielvölkerstaates spürbar und der Kongress in Prag zeigte, dass „sich die österreichischen
Slawen vor die Notwendigkeit gestellt sahen, Perspektiven für ihre Zukunft zu entwickeln“\autocite[6]{moritsch}
. Nach dem Historiker Moritsch\autocite[7]{moritsch} befand sich die Monarchie auf Grund der
Lossagung der italienischen Gebiete, dem drohenden ungarischen Aufstand und der Gefahr
die galizischen Gebiete zu verlieren in einer „tödlichen Krise“. Die zum ungarischen Teil der
Monarchie gehörenden Slowaken und Kroaten waren ebenfalls wegen der möglichen
Abspaltung der Ungarn vom Reich besorgt und unterstützten den Prager Kongress in dem
Bestreben die Einheit Österreichs zu wahren.

Die Zusammensetzung des Kongresses gab dann auch die Unterstützung der Slawen wieder;
eine Mehrzahl der Teilnehmer waren Tschechen und Slowaken, außerdem waren in geringerer
Zahl Südslawen sowie Polen und Ukrainer vertreten.\autocite[17]{moritsch}
Aufgrund der Überzahl der tschechischen und slowakischen Teilnehmer avancierte der
Kongress trotz der feindlichen Propaganda der deutschen und ungarischen Zeitungen, die dem
Kongress russisch-panslawische Interessen unterstellten\autocite[69]{pokorny}, zu einem
Ausdruck von friedlichen und demokratischen Ideen. Die Gefahr eines Panslawismus
russischer Prägung bestätigte sich nicht auf dem Kongress, da beispielsweise die Russen nur
durch Michail Bakunin vertreten waren und selbst dieser war in seiner revolutionär-
anarchischen Einstellung kein Befürworter einer russischen Hegemonie. Außerdem
dominierten auf dem Kongress vor allen die österreichischen Slawen, während die Polen mit
ihrer Forderung nach einer Verankerung der Herstellung ihres Unabhängigkeit im offiziellen
Abschlussmanifest scheiterten.

An der Schwelle zur zweiten Hälfte des 19. Jahrhunderts formulierte ein anderer bekannter
Austroslawist, neben dem o.g. Palacky, Karel Borovský die Politik des Austroslawismus unter
der Berücksichtigung der vorangegangenen Ereignisse und mit einem Ausblick in die Zukunft:
„Der Weg, den die gegenwärtige Regierung [Schwarzenberg] bestreitet, ist die Zentralisierung
mit einer scheinbaren Ständeverfassung und einer scheinbaren Gleichberechtigung. Unser
Weg ist eine Föderation mit einer wirklich demokratischen Verfassung und wirklicher
Gleichberechtigung [...]“. Weiter wird von ihm das Ziel einer panslawischen Zusammenarbeit
innerhalb der österreichischen Grenzen proklamiert: „Ein Bund gegenseitiger Unterstützung
und gegenseitigen Schutzes unter Polen, Tschechen, Südslawen und Ruthenen ist die einzig
wahre Grundlage der Freiheit der schwächeren mitteleuropäischen Völker, es ist ein
amphyktionischer Bund zur Verteidigung der Rechte gegen mächtige Schadensstifter und
Gewalttäter“%\footnote{Borovský o.J.}
%TODO: find havlicek(Borovský o.J. zit. n. Havlíček 2002, 504f und 513)
.
Das hier formulierte Ziel des stärkeren politischen Gewichts der slawischen Völker in
Österreich war in der zweiten Hälfte des 19. Jahrhunderts das beherrschende Thema im
slawischen gesellschaftlichen Diskurs geworden.

\subsection{Russischer Panslawismus}
Der panslawische Idee, die von Kollár 1826 geäußert wurde, wurde auch in Russischen Reich
vernommen, wenngleich diese Idee nicht sehr verbreitet war, denn bis zum ersten Weltkrieg
war in Russland die Tendenz vorhanden, dass nicht-ethnische Russen hohe Positionen in der
Politik einnahmen, so zum Beispiel die ethnische deutsche Zarin Katherina II. und –
relevanter für den Zeitraum dieser Hausarbeit – Karl Robert Graf Nesselrod. Dieser war
Außenminister von 1816 bis 1856 und hatte als Aristokrat „keinerlei Sympathie für
Nationalismus oder Panslawismus“\autocite[115]{kohn}. Auch der Zar Nikolaus I. war kein
Anhänger des Panslawismus, weswegen der Panslawismus in der ersten Hälfte des 19.
Jahrhunderts keine Rolle in der Politik des Staates spielte.

Trotzdem war dieser Zeitraum in vielerlei Hinsicht bestimmend für Russland. In kultureller
Hinsicht entwickelte sich die russische Sprache durch Nationaldichter wie Alexandr Puškin
und Michail Lermontov zur einer Sprache des Adels und der Intellektuellen und löste langsam
die französische Sprache ab. Politisch und militärisch hatte Russland eine wichtige Rolle in
Kontinentaleuropa eingenommen, bedingt durch den Sieg über Napoleon; durch die
Ausrufung der Heiligen Allianz hatte Russland nicht nur den eigenen autokratischen
Charakter, sondern auch der Nachbarn Preußen und Österreich gestärkt. In Bezug auf das
Thema der Hausarbeit ist aber etwas anderes wichtig, nämlich, dass durch das Hervortreten
von Intellektuellen die „europäischen Ideen tiefer in das Bewußtsein der kleinen gebildeten
Schicht drangen“\autocite[117]{kohn} und dass die Geschichte Russlands durch die Russen
selbst nun stärker erforscht wurde, zum Beispiel durch Karamzin, der ab 1818 seine
Darstellung der russischen Geschichte veröffentlichte.

In der ersten Hälfte des 19. Jahrhunderts nahmen russische Intellektuelle verstärkt an der
Bewegung der deutschen Romantik teil. Die Romantik idealisierte die eigene Vergangenheit
und verband diese mit einem Volksgeist. Es begann sich unter dem Eindruck dieser Strömung
eine slawische Identität unter den Russen zu konstituieren. Als Beispiel dafür kann man das
Gedicht von Aleksandr Puškin „\textcyr{Клеветникам России}“ ansehen. Das Gedicht wurde nach
dem polnischen Aufstand 1830 geschrieben, als die westlichen Intellektuellen das Eingreifen
Russlands kritisierten :
„\textcyr{
Оставьте: это спор славян между собою / Домашний, старый спор, уж взвешенный
судьбою,/ Вопрос, которого не разрешите вы. [...] Уже давно между собою/ Враждуют
эти племена;/ Не раз клонилась под грозою / То их, то наша сторона. / Кто устоит в
неравном споре: /Кичливый лях, иль верный росс? /Славянские ль ручьи сольются в
русском море? / Оно ль иссякнет? вот вопрос.[...]}“\autocite[282]{pushkin}

U.a. Hunczak\autocite[89]{hunczak} weist auf die richtungsweisende Bedeutung des Gedichtes hin: „In
these words Pushkin expressed what was to become the credo of the Russian Pans-Slavists.“
Durch den Einfluss der deutschen Romantik sowie englischer und französischer Philosophie
entstand in Russland ab den 1820ern eine Gruppe von Leuten, die als Slawophile bezeichnet
wurden. Es lohnt sich, diese Gruppe in ihren Ansichten vorzustellen, weil die Slawophilen
häufig mit dem Panslawismus in Verbindung gebracht wurden.

Den Vertretern dieser Denkrichtung lag zu Grunde, dass sie ähnlich europäischen Denkern
wie de Maistre und Carlyle die Dekadenz und Oberflächlichkeit des modernen Europa
kritisierten und ihren Blick „sehnsüchtig auf eine romantisch idealisierte Vergangenheit“\autocite[120]{kohn}
 richteten. Im Sinne dieser Ablehnung der modernen europäischen
Gegenwart bezogen die Slawophilen Stellung gegen den Fortschritt und stellten die geistige
russische Isoliertheit als Vorbild und den russischen Bauern als dem westlichen Bürger
überlegen dar. Ein darauf und auf dem Glauben beruhendes Sendungsbewusstsein, das dem
polnischen Messianismus sehr ähnlich war, wurde innerhalb des Staates zum Teil unterstützt,
wenn auch vielleicht unbewusst. Sergej Uvarov propagierte 1833 seine drei Prinzipien der
Orthodoxie, Autokratie und narodnost'. Dabei war die Orthodoxie neben der narodnost' ein
wichtige Basis in dem Streben nach einer „konservativen Utopie“\autocite[408]{walicki} der
Slawophilen.

Die slawophilen Intellektuellen entwickelten sich ab den 1830ern immer mehr zu den
Befürwortern eines russischen Panslawismus, der in der Forschung von Kohn\autocite{kohn} und
Hunczak\autocite{hunczak} mit einem Panrussismus gleichgesetzt wird.

Einer, der diese russische Hegemonialstellung innerhalb der slawischen Gemeinschaft
propagiert hat, war der Moskauer Professor Michail P. Pogodin. Er war, was die politischen
Umrisse des Panslawismus in der ersten Hälfte des 19. Jahrhunderts angeht, der
herausragendste Slawophile. Zumindest wird er bei Nicoll\autocite[232]{nicoll} als einer der ersten
slawophilen Befürwortern des Panslawismus beschrieben: „The were a few Russians who
personally bridged the transition to Panslavism. Mikhail P. Pogodin war, perhaps, Russia's
first real modern Panslavist.“

Pogodin sah in den Slawen eine Einheit, in der allerdings die Russen die Mehrheit bildeten. Er
äußerte seine Gedanken zur slawischen Einheit 1838 in einem Brief an einen Beamten im
Erziehungsministerium: „Ein Volk von 60 Millionen und bald [...] 100 Millionen zählen wird.
Dieser Zahl laßt uns noch die dreißig Millionen Brüder und Vettern hinzufügen, die Slawen,
in deren Adern das gleiche Blut fließt wie in unseren, die die gleiche Sprache sprechen wie
wir und deshalb, einem Naturgesetz folgend, wie wir fühlen, die Slawen, die trotz
geographischer und politischer Trennung durch Ursprung und Sprache eine geistige Einheit
mit uns bilden [..].“ Nach dieser Konstruktion einer slawischen Einheit proklamierte er eine
Universalmonarchie unter dem russischen Zaren: „ Und bedenkt, dass diese Maschine ... von
ihren Vorfahren her nur einem einzigen Gefühl belebt ist: Ergebenheit, grenzenloses Vertrauen
und Hingabe an den Zaren, ihren Gott auf Erden.“\footnote{Pogodin o.J. zit. n. \autocite[128]{kohn}}
In diesem Satz äußerte sich der religiöse Aspekt der primär orthodoxen Religion in einem
künftigen slawischen Reich, da der Zar den orthodoxen Glauben hatte.

Neben Pogodin gab es weitere Slawophile, die einen russisch-dominierten Panslawismus
vorschlugen. Einer dieser Slawophilen war Ivan S. Aksakov. Seine Ansichten waren ebenfalls
stark durch den Gedanken einer Universalmonarchie geprägt, in der sich die anderen
slawischen Völker dem orthodoxen Glauben anschließen und die russische Sprache
übernehmen sollten.\autocite[232]{nicoll}

Ein besonderer slawophiler Denker war der Dichter Fedor I. Tjučev, weil er das slawophile
Sendungsbewusstsein ergänzte um die Verknüpfung des religiösen Aspektes mit der
Wiedereroberung Konstantinopels: „ [...] Fedor I. Tiuchev melded the messianic Orthodoxy
of the Slavophiles and Pogodin's Panslavism in poems and articles which envisioned an
eventual Slavic triumph in a great empire centered around the „reconquest“ of
Constantinople.“\autocite[232]{nicoll}

Der Kreis von slawophil-gestimmten Menschen beschränkte sich natürlich nicht nur auf die
Erwähnten. Es gab weitere Denker wie Chomjakov und Ivan Kireevskij. Diese waren
innerhalb der Bewegung sogar als führend anzusehen\autocite[132]{kohn}, doch waren
diese in der Rezeption der Slawophilen als Panslawisten nicht entscheidend. Pogodin sowie
Tjučev waren die hauptsächlichen Vertreter des Panslawismus, der in Russland vor allem
nach dem verlorenen Krim-Krieg in der Gesellschaft eine größere Rolle zu spielen begann.

\subsection{Polnischer Messianismus}
Die Polen hatten eine gewisse Sonderstellung innerhalb der slawischen Gemeinschaft. Zum
einen wurde  Polen von drei Mächten, nämlich Russland,
Preußen und Österreich, aufgespalten und zum anderen wurde das Schicksal der Polen von
Westeuropäern wie das kaum eines anderen slawischen Volkes verherrlicht und idealisiert.
Die Polen selbst sahen sich aber bis zum Beginn des 19. Jahrhunderts nicht vorrangig als ein
slawisches Volk an. Erst mit der Enttäuschung auf dem Wiener Kongress, als Polen, nach der
Niederlage der Franzosen, wieder unter die Herrschaft der umgebenden Mächte kam, wurde
eine Rückbesinnung auf die Vergangenheit unternommen. Je stärker die Repression im
russischen Gebiet wurde, desto stärker wuchs auch der polnische allslawische Gedanke als
Gegengewicht zu Russland und auch gegen Preußen.\autocite[252]{falkovic} Die polnische
Ausprägung der panslawischen Idee nannte sich im Ganzen Messianismus, ein Begriff, der
vom polnischen Philosophen Josef Hoëné-Wronski um 1830 verwendet wurde.

Nicht nur Hoëné-Wronski hat den Messianismus geprägt; gerade bei den Philosophen, die im
19. Jahrhundert geboren wurden, wurde dieser Begriff weiterentwickelt und durch mehrere
Dichter in der Gesellschaft verankert. Gleichwohl unterschieden sich die Ausprägungen des
Messianismus in Polen oder, besser gesagt, bei den Polen. Denn tatsächlich entwickelten sich
die Grundzüge des polnischen Messianismus nur am Anfang auf dem Gebiet Polens.
Mickiewicz, der berühmteste polnische Dichter, war zum Beispiel nicht in Polen geboren,
sondern in Litauen und verbrachte einen bedeutenden Teil seines Lebens im Ausland,
vornehmlich in Paris, wie auch ein Großteil der polnischen Intelligenz\autocite{braechter}.

Die Einstellung zu Russland hat bei Vertretern des Messianismus dabei eine entscheidende
Rolle gespielt. Bei den frühen Philosophen wie Stanisław Staszic und Hoëné-Wronski wurde
Russland eine führende Rolle bei der Vereinigung der slawischen Völker zugemessen und
trotz der unterschiedlichen Konfession betrachtete man Russland als „ein Bollwerk göttlichen
Rechts“ (Kohn 1956, 40). Ähnlich den Ansichten der russischen Slawophilen steht der Inhalt
des Messianismus auch im Gegensatz zur westlichen Welt. Stazsic kritisierte den Westen auf
zweierlei Weise, zum einen nahm er Bezug auf die zunehmende Säkularisierung, also die
Trennung zwischen Staat und Kirche und zum anderen auf die Demoralisierung
durch die Eroberungsfeldzüge des Westens\autocite[39]{kohn}. Die Slawen dagegen
konnten die “absolute, von einem Slawen aufgestellte Philosophie realisieren und durch die
Einheit von Staat und Kirche, von Vernunft und Glauben, die sozialen Antinomien lösen“
\footnote{Staszic o.J. zit. n. \autocite[40]{kohn}}.

Im Verlauf der 1830er veränderte sich die polnische Sichtweise auf die anderen Slawen. Am
Anfang dieser Periode stand der Aufstand von polnischen Nationalisten, der von russischen
Truppen niedergeschlagen wurde. Die Zeit danach war mit einem Exodus von vielen
tragenden Persönlichkeiten des geistigen Lebens verbunden, so zum Beispiel mit den drei
wichtigsten polnischen Dichtern des 19. Jahrhunderts: Mickiewicz, Krasinski und Slowacki.
Hauptsächlich wanderten diese Dichter in westeuropäische Länder aus, zumeist mit dem Ziel
Frankreich und hier vor allem Paris. In der Zeit nach 1830 waren die Dichter und ein Großteil
der Bildungselite gegen Russland eingestellt\autocite[252]{falkovic}, es wuchs aber auch
die Abneigung gegen die anderen slawischen Völker, besonders gegen die Tschechen, weil
diese in den westlichen Liberalismus eingebunden waren und von ihnen auch der liberale
Austroslawismus ausging. In den Vorstellungen von Mickiewiecz und anderen Messianisten
dominierte dagegen ein Bild von den Slawen als Bauern ohne die westlichen Einflüsse, deren
gesellschaftliche Harmonie auf der Religion basierte\autocite[50]{kohn}. In den 1840ern
äußerte Mickiewicz den Anspruch des Messianismus, die slawischen Völker anzuführen und
begründete dies mit biblischen Motiven: „Jetzt werden Sie auch erkennen, warum die
polnische Nation näher bei der Wahrheit steht als irgendein anderes slawisches Volk, weil die
Offenbarung Christi immer der Maßstab für alle die sein wird, die ihm folgen, weil es bloß
einen Weg zur Wahrheit gibt: den Weg des Kreuzes.“\footnote{Mickiewicz 1914 zit. n. \autocite[54]{kohn}} 

Damit richtete sich der Messianismus gegen die übrigen slawischen Völker, wie es sich
dann auf der ersten panslawischen Zusammenkunft 1847 in Prag offenbarte. Dort dominierte
ein Konzept des Austroslawismus, das die „Umwandlung der Habsburgermonarchie in eine
slawisch-österrreichische Föderation vorsah“\autocite[103]{cetnarow}, die aber die
polnischen Gebiete unter russischer und preußischer Herrschaft nicht einschloss. Das Ziel der
Polen, ihren Staat wiederherzustellen, war auf dem austroslawisch dominierten Kongress kein
Thema, dessen sich die österreichischen Slawen annehmen wollten. Der Prager Kongress
stand sogar „in scharfen Gegensatz zu den polnischen Unabhängigkeitsbestrebungen [...]“\autocite[110]{cetnarow}
. Für die Polen sollte zuerst der polnische Staat wiederhergestellt
werden und dann eine Föderation der Slawen in Mitteleuropa geschaffen werden. Durch die
Größe Polens hatten aber die Tschechen Angst vor einer dominierenden Rolle Polens,
besonders in Bezug auf ein Gebiet, das historisch-rechtlich zum tschechischen Gebiet gehörte,
kulturell aber eher polnisch geprägt war, nämlich Schlesien und hier die Region um Teschen
(heutiges Cieszyn). Die Vertreter dieser Region argumentierten auf dem Kongress im Sinne
der polnischen Position und lösten deswegen „ein gewisses Mißverständnis in den polnisch-
tschechischen Beziehungen“\autocite[112]{cetnarow} aus. Eigentlich spiegelte sich auf dem
Kongress in Prag eine Hoffnung auf Verständigung der Slawen wider. Von der polnischen
Gesellschaft wurde der Kongress aber mit der Hoffnung auf die Wiederherstellung ihres
Staates und einer gewissen Hoheit gegenüber anderen Nationalitäten, wie den Ukrainern
verbunden. Gegen diese Ansichten richteten sich die übrigen Slawen. Bei den polnischen
Intellektuellen wurde der Kongress mit all seinen verschiedenen und zum Teil anti-polnischen
Positionen als negativ für das polnische Ziel einer Unabhängigkeit angesehen und in der
Folge wurde „das Gefühl der Solidarität mit den slawischen Völkern abgeschwächt und
feindselige Haltungen verstärkt“\autocite[114]{cetnarow}.

\subsection{Südslawischer Illyrismus}
Infolge der aufkommenden pannationalen Bewegungen in Europa am Anfang des 19.
Jahrhunderts haben sich auch bei den Südslawen Ideen zur übergreifenden Identität
entwickelt. Hier soll auf die illyrische Bewegung eingegangen werden, die besonders wirksam
bei den Kroaten war.

Die Südslawen als Ganzes hatten zu Beginn des 19. Jahrhundert einige Besonderheiten im
Vergleich zu den anderen Slawen. Zum einen waren sie von den übrigen slawischen Stämmen
getrennt. Im Norden siedelten die Deutsch-Österreicher und in einem Gürtel trennten
weiterhin die Ungarn und die Rumänen südslawische Stämme wie die Serben und Bulgaren
von den West- und Ostslawen.

Es steht fest, dass die erste Idee einer slawischen Einheit zuerst im 17. Jahrhundert von einem
Südslawen, dem Jesuiten Juraj Križanić, entwickelt wurde. Er stellte sich eine auf
Multikonfessionalität basierende Vereinigung von den verschiedenen Kirchen der Slawen vor,
so zum Beispiel der lutheranischen Richtung bei den Slowaken, der orthodoxen bei den
Russen und der katholischen Kirche bei den Slowenen und Kroaten.\autocite[232]{nicoll}
Seine Schriften blieben aber in der Genese der panslawischen Identität im 19. Jahrhundert
unberücksichtigt. Wichtiger waren für den Illyrismus die Ansichten von Ján Kollár. Denn er
hatte die südslawischen Völker sprachlich zusammengefasst und in seinen Schriften die
Verbindung zum antiken Volk der Illyrer hergestellt.\autocite[65]{shidak}

Eine politische Tendenz wurde bereits am Anfang des 19. Jahrhunderts innerhalb der
Napoleonischen Kriege deutlich, als die Franzosen eine sogenannte Illyrische Provinz
gründeten, wo sich die Slowenen, Kroaten und Serben in einem gemeinsamen halbstaatlichen
Gebilde wiederfanden. Die Reformen, die die französischen Verwalter im Hinblick auf die
Schaffung einer gemeinsamen Nationalität unternahmen, waren aber mit einigen
Schwierigkeiten verbunden: Der Begriff Illyrismus wurde zwar oft verwendet, aber sein
Inhalt wurde von den verschiedenen slawischen Völkern jeweils anders betrachtet. Die
Slowenen hielten sich für die wahren Illyrer und bestanden in der Diskussion auf ihren
eigenen Dialekt und die Kroaten waren wiederum für ihren eigenen Dialekt als übergeordnete
Sprache.\autocite[62]{kohn}

Zu einem wichtigen Impulsgeber der illyrischen Bewegung wurde Ludevit Gaj. Er war ein
kroatischer Schriftsteller und wurde bei seinen Aufenthalten in Böhmen von dem zuvor
erwähnten Kollár beeinflusst. Kollárs Ansichten, dass die südslawischen Stämme kulturell
eine Einheit bildeten, wurden von ihm in der Hochzeit des Illyrismus in den 1830ern in
einigen einflussreichen Zeitungen vertreten. Der Name Illyrer wurde für sein Konzept
gewählt, da man den Südslawen keinen der nationalen Namen (Serben oder Kroaten)
aufdrängen wollte.\autocite[75]{shidak}

In Kroatien fand der Illyrismus viele Anhänger, da dieser als Konzept die Kroaten in den
Mittelpunkt der Südslawen stellte und dem aufkommenden Nationalismus der Ungarn ein
Konzept entgegenstellte, in dem die Slawen als Einheit auftraten.
Innerhalb der anderen südslawischen Völker stieß der kroatisch geprägte Illyrismus aber auf
Widerstand. Denn er verdeckte die Besonderheiten der Slawen mit ihrer unteschiedlichen
Geschichte und Sprache.

Die Slowenen beispielsweise befanden sich nicht innerhalb der ungarisch dominierten Hälfte
Österreichs, sondern in der von den Deutsch-Österreichern beeinflussten Hälfte des Reiches,
womit der steigende ungarische Nationalismus die Slowenen nicht berührte. Außerdem war es
auch eine Zeit für die Slowenen, in der mehrere herausragende Dichter und Gelehrte
hervortraten, so zum Beispiel France Prešeren, Janez Bleiweis, Fran Levstik und Jernej
Kopitar, die sich an der Herausbildung eines slowenischen Nationalbewusstseins beteiligten.
Die Tendenz in der Bildungselite ging bei den Slowenen in Richtung einer liberaleren und
demokratischen Konzeption der slawischen Zusammenarbeit, nämlich zum Austroslawismus.
\autocite[64]{kohn} Und nicht zuletzt war für die Slowenen die Sprache der Kroaten und
Serben nicht nah genug an ihrer, sodass die „sprachliche Grundlage in der Ideologie des
Illyrismus auch in dieser Hinsicht als nicht genügend reale zeigte“\autocite[81]{shidak}.

Auch die anderen großen südslawischen Völker waren eher gegen die illyrische Konzeption
der Kroaten. Denn bei allen Völkern kamen in der ersten Hälfte des 19. Jahrhunderts
bedeutende Dichter auf, die das eigene Nationalitätsbewusstsein stärkten, so zum Beispiel bei
den Serben Branko Radičević oder Đura Jakšić.

Eine slawische Einheit scheiterte auch an den historischen Eigenheiten der Völker.
Beispielsweise wurde das Nationalbewusstsein der Serben in den Napoleonischen Kriegen
gestärkt: der serbische Kampf unter Đorđe Petrović und Miloš Obrenović gegen die
unterdrückenden Mächte wurde zu einer nationalen serbischen Legende. Dies trug dazu bei,
dass bei den Serben schon „zu Anfang des (19.)Jahrhunderts ein Staatskern entstand, der
schon durch sein Bestehen allein zum anziehenden Zentrum für das serbische Volk werden
musste“\autocite[77]{shidak}. Weiterhin gab es zwischen den einzelnen südslawischen Völkern
Konflikte um Einfluss auf kleinere Völker, wie auf die Bosnier, denen eine nationale Identität
fehlte und die wegen ihrer Multikonfessionalität zum Streitpunkt zwischen den orthodoxen
Serben und den katholischen Kroaten wurden. Ein anderes Beispiel für die Unstimmigkeiten
zwischen den Südslawen war das Gebiet Mazedoniens. Dieses wurde zugleich von den
Bulgaren und den Serben für sich beansprucht und beide Völker bauten es in ihre eigene
Geschichte ein.

Der Gedanke des Illyrismus, der eine Gleichberechtigung der slawischen Völker propagierte,
scheiterte also an der politischen Realität. Dies zeigte sich sehr deutlich in der zweiten Hälfte
des 19. Jahrhunderts, als auch die Serben und die Bulgaren übernationale Ambitionen
entwickelten. Die Bulgaren strebten zum Beispiel in der zweiten Hälfte des 19. Jahrhunderts
ein Großbulgarien an. Die politische Führungsriege der Serben wollte dagegen selbst das
Zentrum der südslawischen Völker werden und ein Großserbisches Reich errichten.\autocite[84]{shidak}

\newpage
\closuresection{Schlussbetrachtung}
\subsection*{Zusammenfassung}
Welchen Eindruck gewinnt man in der vorliegenden Arbeit vom Panslawismus? Man könnte
unter Berücksichtigung der vielfältigen Ausprägungen dieser Idee der folgenden
Charakterisierung eines Historikers Rechts geben: „A tendency among the peoples of the
slavic world to manifest their ethnic ties. Although there were earlier expressions, modern
Panslawism emerged in the nineteenth century in several unsystematic and partially
conflicting forms.“\autocite[232]{nicoll}

Man kann diese recht abstrakte Definition in der dargelegten Arbeit mit Inhalt versehen. Wir
können tatsächlich sehen, dass jede Ausprägung durch einzelne Leute geschah, die zwar
zumeist einen gemeinsamen Bezugspunkt hatten (nämlich Herder), die aber in der politischen
Phase in den 1830ern allesamt eigene Konzeptionen entwickelten, die manchmal einander
auch widersprachen oder miteinander konkurrierten. Außerdem stützten die Denker der
panslawischen Strömungen ihre Konzepte zumeist auf bestimmten Idealen oder religiösen
Argumenten. Das passierte besonders bei den Polen in ihrem Messianismus und beim
russischen Panslawismus. Es bestätigt sich dadurch die unsystematische Form.
Dass der Panslawismus als Ganzes gesehen auch Konflikte gefördert hat, sieht man ebenfalls
im Vergleich der Konzepte. Wenn man sich zum Beispiel den Austroslawismus und den
polnischen Messianismus anschaut, liegt die Unvereinbarkeit in dem Wunsch der
Austroslawisten Österreich zu erhalten und Gleichberechtigung unter den Slawen zu fördern,
während die polnischen Messianisten eine Wiederherstellung Polens mit dem österreichischen
Teil Galiziens anstrebten. Dieses Ziel war mit dem Erhalt Österreichs nicht vereinbar.
Außerdem beanspruchten die Polen die Führung in einer panslawischen Gemeinschaft, was
wiederum der Gleichberechtigung, die Palacký geäußert hat, widersprach. Zudem seien noch
die beschriebenen Streitigkeiten territorialer Art um Schlesien genannt.
Man kann eigentlich jede Bewegung einer anderen entgegenstellen, so auch die Polnische der
Illyrischen. Während die Kroaten in den Ungarn ihre Gegner sahen, identifizierten sich die
Polen mit dem Streben der Ungarn nach einem Nationalstaat. Schaut man sich beispielsweise
den russischen Panslawismus mit seinen slawophilen Vertretern an, steht er im Gegensatz zum
Austroslawismus. Denn Pogodin redet einer Universalmonarchie das Wort, während der
austroslawische Wortführer Palacký gerade auf der Ablehnung des russischen Zaren den
Austroslawismus formt.
\subsection*{Stellungnahme}
Die panslawischen Ausprägungen übersahen also die Realität, weil sie dem differenzierten
Charakter der slawischen Völker nicht genug Beachtung schenkten. Die slawischen Völker
waren im Begriff ihren eigenen Nationalmythos zu begründen, was man an der Vielzahl an
nationalen Dichtern in der 1. Hälfte des 19. Jahrhunderts wie Mickiewicz, Puškin und
Prešeren sieht.

Weiter denke ich, dass diese Arbeit Aufschluss im Hinblick darauf gibt, wieso der
Panslawismus niemals zur offiziellen Politik des Zarenreiches wurde. Neben einer
beschriebenen Tendenz, dass am Hof und in der Politik häufig Russen mit ausländischen
Wurzeln anzutreffen waren, spiegelt sich ein außenpolitischer Pragmatismus im Denken der
Herrscher Russlands in der 2. Hälfte des 19. Jahrhunderts wider. Der sowjetische Historiker Koz'menko\autocite[191]{kozmenko}
formulierte das in Bezug auf die Balkanstaaten so: „\textcyr{В Петербурге понимали [...] слабость
балканских стран, которые, идя вместе с Россией, нуждались в ее поддержке и мало чем
могли помочь ей.}“ Außerdem gab es durch die vielfältigen Ausprägungen des Panslawismus
in den slawischen Völkern eigentlich keine Perspektive für ein geeintes slawisches Reich
unter russischer Führung. Die in der russischen Gesellschaft vorherrschenden panslawischen
Ideen wie zum Beispiel bei Pogodin konnten bei den anderen Slawen keine Begeisterung
auslösen. Jede Tendenz von Russland aus die eigene panslawische Idee zu verwirklichen,
musste zu einer slawischen Gegenbewegung führen, wie man am Beispiel des
Austroslawismus und seiner Ablehnung der Hegemonie Russlands sieht.

Es ist aber auch allgemein fraglich, welche Gemeinsamkeiten die slawischen Völker
verbanden und welche überhaupt für einen Zusammenschluss geeignet waren. Die häufig
benutzten Begriffe der verwandten Sprache oder der Religion erscheinen in dieser Hausarbeit
ungeeignet, was man an der Ablehnung des Illyrismus durch die Serben, obwohl sie sich mit
den Kroaten praktisch die gleiche Sprache teilten, merkt. Nicht zuletzt war ein
Zusammenschluss von den beiden orthodoxen Völkern der Serben und Bulgaren ebenfalls
unvorstellbar.

Der Austroslawismus als aussichtsreichste Idee des Panslawismus basierte demgegenüber
nicht auf Religion oder Sprache, sondern auf der politischen Gemeinsamkeit, dass jedes
einzelne slawische Volk eine Minderheit war, alle gemeinsam aber eine Mehrheit. Als
einzelne Minderheiten fürchtete man die Übermacht der Deutschen und der Ungarn; als
Mehrheit wollte man einen demokratischen und föderativen Staat gemeinsam mit den nicht-slawischen Völkern aufbauen. Dass der Verwirklichung dieses Konzeptes nach der
Entwicklung Österreichs zur Doppelmonarchie 1867 keine Zukunft beschieden war, könnte
eine der Ursachen für den Zerfall Österreich-Ungarns nach dem 1. Weltkrieg sein.

\subsection*{Ausblick}
Die vorliegende Hausarbeit hat ein breites Spektrum der panslawischen Ideen umfasst.
Allerdings stellt die Hausarbeit nur einige Ausprägungen und Ansichten dar, auf denen auch
die Forschergemeinde Schwerpunkte gesetzt hat. Trotzdem gibt es weitere Ansichten und
Ideen, die in der Forschung nur eine kleine Rolle einnehmen, die aber gleichzeitig nicht
minder interessant wären. Es mangelt zum Beispiel an der Darstellung der Weißrussen und
der Ukrainer in der Periode des Panslawismus. Die einzige Ausnahme ist die Arbeit von
Botušans'kyj\autocite{botushans}, die allerdings auch nur die Ukrainer in Galizien darstellt, nicht die im
Russischen Reich. Ich denke aber, dass in Zukunft den Ukrainern deutlich mehr
Aufmerksamkeit seitens der Forschungsgemeinde zu Teil werden wird, als der ich bei der
Lektüre der Forschungsliteratur gewahr wurde.

Lohnenswert wäre es deshalb, weil es in der ersten Hälfte des 19. Jahrhunderts einen
Höhepunkt des ukrainischen Nationalbewusstseins durch den Dichter Taras Ševčenko und der
Kyrillo-Methodianischen Bruderschaft gab; in der westeuropäischen Forschungsliteratur
wurde dieser Teil der ukrainischen Geschichte noch nicht untersucht. In Bezug auf den
Zeitraum dieser Hausarbeit wären aber die Ansichten der ukrainischen Intellektuellen zur
panslawischen Idee sehr interessant. Es wäre wohl fahrlässig von mir von einem Desiderat in
der Forschung zu sprechen, weil meine Recherche sich größtenteils nur auf die deutsche,
russische und englische Literatur beschränkte; die ukrainische, polnische oder bulgarische
Literatur konnte ich wegen der begrenzten sprachlichen Möglichkeiten nicht nutzen. Ebenso
habe ich nur wenig über die Weißrussen gelesen. Es wäre ein spannendes Thema zu
untersuchen, inwiefern die Weißrussen vom Panslawismus beeinflusst wurden. Man müsste
das auch mit der Frage verknüpfen, wie eigentlich die weißrussische oder die ukrainische
Gesellschaft zu der Zeit aussah. Und welche Auswirkungen hatten die westeuropäischen
Strömungen auf die Weißrussen und die Ukrainer?

Es wäre generell ein interessantes Thema zu verfolgen, wie sich eine gebildete Schicht in den
einzelnen slawischen Völkern entwickelte, welche Faktoren dafür ausschlaggebend waren und
welche Ansichten diese Intellektuellen vertraten. In Bezug auf bestimmte Völker wie die
Tschechen und die Slowaken scheint die Forschung fortgeschritten zu sein, andere, kleinere
Völker wie die Sorben, Russinen und Mazedonier wurden nach meinem Eindruck bisher noch
nicht zum Gegenstand der Forschung gemacht.

\literature
\end{document}
