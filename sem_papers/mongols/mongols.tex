\documentclass[12pt,headsepline,a4paper]{scrartcl}
\usepackage[OT2,T1]{fontenc}
\usepackage[ngerman]{babel}
\usepackage[
backend = bibtex8,
bibencoding = utf8,
defernumbers=true,
natbib = true,
style = verbose-ibid, 
maxnames = 2,
minnames = 1,
backref = true,
backrefstyle = two, 
]{biblatex} 
\addbibresource{mongols.bib}
\usepackage[utf8]{inputenc}
\usepackage{scrpage2}  %<---
\usepackage{bibentry}
\usepackage[onehalfspacing]{setspace}
\usepackage[a4paper]{geometry}
\geometry{left=3cm,right=3cm,top=2cm, bottom=2cm}
\pagestyle{scrheadings}%<
\clearscrheadfoot
\chead{\leftmark}
\automark[section]{section} 
\rohead{\pagemark}
\title{}
\author{}

\begin{document}
\begin{titlepage}
\date{}

\title{\Large Ein Brief des Kaisers über die Mongolengefahr}
{\let\newpage\relax\maketitle}

\begin{center}
\vfill
Hausarbeit\\
vorgelegt von\\
Igor Fischer \\
\null
Hamburg 2012
\vfill
\end{center}

\begin{minipage}{0.5\textwidth}
\begin{flushleft} 
Universität Hamburg\\
Fachbereich Geschichte\\
Proseminar 53-193 „Die Invasion der Mongolen – Ihre Wahrnehmung in Europa anhand
ausgewählter Quellen aus dem 13. Jahrhundert“\\
Wintersemester 2011/12\\
Dozentin: Dr. Ingeborg Braisch\\
\end{flushleft}
\end{minipage}
\vline
\begin{minipage}{0.5\textwidth}
\end{minipage}

\pagenumbering{gobble}
\end{titlepage}

\tableofcontents
\thispagestyle{empty}
\pagenumbering{arabic}
\newpage

\section*{Einleitung}
\markboth{Einleitung}{Einleitung}
\addcontentsline{toc}{section}{Einleitung}
13. September 2011: Angela Merkel ist in der Mongolei und unterzeichnet
verschiedene Verträge mit lächelnden Gastgebern. Dies ist ein Bild, das ganz treffend
die Mongolei im 21. Jahrhundert beschreibt. Es ließen sich noch weitere Besuche von
westlichen Präsidenten wie George Bush oder Vladimir Putin in diesem Land aufzählen.
All das vermittelt das Bild eines Landes, welches eine positive Rolle in der Welt spielt.

Im 13. Jahrhundert war dieses Volk der Mongolen in einem sehr viel anderen Licht.
Man assoziierte mit ihm allerhand negative Aspekte und sah im Auftauchen der
Mongolen sogar das Ende der Welt.

Es ist vielleicht vermessen, die heutige Welt mit der des 13. Jahrhunderts zu
vergleichen, ohne die Zwischenschritte zu beachten, die die westliche Gesellschaft beim
Rezipieren anderer Kulturen und Gesellschaften gemacht hat. Dieser Versuch soll auch
unterbleiben. Schließlich gab es auch im 13. Jahrhundert sehr verschiedene Weisen,
auf die Fremden zu blicken. Innerhalb dieser Hausarbeit soll vor allem ein Blickwinkel
analysiert werden, der des Kaisers des Heiligen Römischen Reiches Friedrichs II. Seine
Bedeutung ist auch heutzutage sehr groß und er wurde bis ins 20. Jahrhundert zum
„genialen Staatsmann, Vorläufer der Moderne und deutschen Idealherrscher stilisiert“ \autocite[Rückseite]{rader2010} .

Für die Analyse seiner Sicht auf die Mongolen wird ein Brief verwendet, der 1241 an
Heinrich III. geschickt wurde, also zu einer Zeit, in der die Mongolen sehr nah an den
Grenzen des Reiches waren. Zur Struktur der Hausarbeit: Zuerst sollen in der
Quellenbeschreibung die Rahmenbedingungen des Briefes dargestellt werden, d. h. wer
schrieb den Brief an wen und welche Intention lässt sich anhand der ersten Sätze
erkennen. Danach soll der Inhalt vor allem im Hinblick auf die Argumentationsstruktur
dargestellt werden.
Die dann folgende Analyse verfolgt den Zweck, diese Einheit des
Briefes in mehrere Elemente oder Schwerpunkte zu zerlegen und zu schauen, wie man
die gegebenen Informationen kategorisieren kann.

Ausgehend von dieser Analyse soll innerhalb der Einordnung in den historischen
Kontext der Inhalt der Quelle mit den Ereignissen zur Zeit der Abfassung des Briefes
verbunden werden. Damit sollen die externen und internen Faktoren bei der Entstehung
der Quelle berücksichtigt werden. Es soll vor allem auch erarbeitet werden, welche
Funktion die Quelle hat. Abstrakt gesagt, steht also im Rahmen dieser Hausarbeit die
Untersuchung der Form und der Funktion der Quelle im Vordergrund.

\section{Quellenbeschreibung}
Für die Quelleninterpretation liegt ein Brief von Friedrich II. an Heinrich III. vor.
Der Brief trägt in der Übersetzung den Titel „Brief des Kaisers an den König von
England vom 3. Juli 1241“. Im Original ist der Brief auf Latein abgefasst worden, für
die Analyse wird die deutsche Übersetzung verwendet.

Der Autor bzw. der Auftraggeber des Briefes Friedrich II. war zur Zeit der Abfassung
Kaiser des Heiligen Römischen Reiches. Er wurde ab 1211 schrittweise König über
Deutschland und 1220 wurde ihm die Kaiserwürde vom Papst verliehen. In der im Titel
des Briefes genannten Zeit im Juli 1241 unternahm er einen Feldzug gegen den
Kirchenstaat.

Heinrich III. als Adressat war König von England seit 1216. Es war ein Schwager
von Friedrich II. Die Intention des Briefes wird später diskutiert, aber
oberflächlich nach dem Inhalt zu urteilen soll der Brief auf die Gefahr durch die
Mongolen hinweisen und ein gemeinsames Vorgehen gegen die Mongolen
anzusprechen. Die folgende Zusammenfassung wird genauer über den Inhalt und die
Argumentationsstruktur Aufschluss geben:

\section{Inhaltsangabe}
In seinem Brief an den König von England Heinrich III. widmet sich Friedrich II.
mehreren Aspekten des Vordringens der Mongolen. Zuerst geht er auf den Ursprung und
die Ziele der Mongolen ein. Er schreibt, dass sie in „einem Land am Ende der Welt im
Süden“ (Z. 8)\footnote{Im Folgenden wird auf die Quelle der Einfachheit halber mit Zeile und einer Zahl referiert, d.h.(Z.
Ziffer)} gelebt haben. Die Mongolen nennt er im ganzen Brief mit dem Namen
„Tartari“. Als Ziel der Mongolen gibt er an, dass sie danach streben, „die gesamte Erde mithilfe
seiner ungeheuren und unvergleichlichen Macht und Menge an Truppen allein zu
beherrschen“(Z. 19ff.).

Weiter beschreibt er die Kämpfe der Mongolen gegen die Kumanen und gegen die
Ruthenen, wobei er die militärische Überlegenheit und die Plötzlichkeit der Attacken
betont. Dann wird in dem Brief der Kampf der Mongolen gegen die Ungarn
thematisiert. Es wird in dem Brief darauf eingegangen, wieso die Ungarn verloren
hätten.

Im Brief werden für das Erzählte auch die Quellen genannt: Eine Quelle soll der
Bischof Stehan II. von Waitzen gewesen sein. Andere Quellen sind der Sohn Friedrichs
II. Konrad, der König von Böhmen und die Herzöge von Österreich und Bayern.

Nach der Darstellung des Vordringens der Mongolen, wird auf ihre Eigenschaften
eingegangen: Es wird beispielsweise ihr Aussehen beschrieben und sie werden
charakterisiert. Dazu kommt die Beschreibung ihrer militärischen Ausrüstung sowie
ihrer Fortbewegungsmittel, Pferde und Schläuche. Der vorliegende Auszug aus dem
Brief schließt mit dem Aufruf, die Mongolen gemeinsam zu bekämpfen.

\section{Quellenanalyse}
Die Quelle ist, wie in der Quellenbeschreibung angegeben, als ein Warnung gedacht
und soll die Zusammenarbeit zwischen den Königreichen fördern. Eine Warnung macht
im Grunde aber nur Sinn, wenn man dem Adressaten Aspekte vermittelt, die für ihn eine
Bedrohung sein könnten. Es soll daher im Folgenden vorgestellt werden, wie Friedrich
II. die Mongolen beschreibt.

\subsection{Darstellung der Mongolen}
Den Mongolen werden im Brief eher negative Eigenschaften beigemessen. Dies
scheint dazu zu dienen, die Angst vor den Mongolen
\footnote{Obwohl in der Quelle der Name „Tartaren“ verwendet wird, soll außerhalb der Quelle die
Bezeichnung „Mongolen“ verwendet werden.} 
zu schüren, d. h. sie möglichst
gewalttätig und brutal zu charakterisieren. Bereits am Anfang des Briefes werden die
Mongolen negativ dargestellt, wenn von einer „barbarischen Herkunft und
Lebensweise“(Z. 9) die Rede ist.

Die im weiteren benutzten Begriffe in Zusammenhang mit dem Vordringen der
Mongolen haben ebenfalls eine eher negative Konnotation, wie man am folgenden Satz
sieht: „Es hat nun also ein allgemeines Gemetzel gegeben, die Vernichtung von
Königreichen und die Zerstörung der fruchtbaren Erde, die dieses verruchte Volk
durchzog.“(Z. 15-17) Es sind nicht die einzigen Sätze im Brief, die mit den Mongolen
negative Elemente verbinden. Tod und Verwüstung werden zum Beispiel in Zeile 23ff.
mit den Mongolen assoziiert: „Nachdem sie nun alles getötet und erbeutet hatten, was
ihre Augen erblicken konnten und hinter sich eine Spur völliger Verwüstung gelassen
hatten, waren sie zu der volkreichen Siedlung der Kumanen gekommen.“ Das Vorgehen
der Mongolen gegen die Kumanen wird als sehr brutal beschrieben: „Und diejenigen,
die die Flucht nicht rettete, tötete das bluttriefende Schwert der Tartaren.“(Z. 28f.) Es
wird sehr oft versucht die Mongolen mit dem Tod und Barbarentum zu verbinden, wie
beispielsweise beim Kampf gegen die Ruthenen: „Ja und plötzlich stürmen die Tartaren
herbei, um zu morden und zu plündern“(Z. 34f.) In Bezug auf den Kampf gegen die
Ruthenen bekommen die Mongolen eine klare negative Wertung: „Unter dem
plötzlichen ungestümen Vordringen und den Sturmangriffen jenes barbarischen Volkes,
das heranrast, wie der Zorn Gottes und sein Blitz plötzlich geschleudert werden
[...]“(Z. 35f.).

Beim weiteren Vordringen gegen die Ungarn wird ein ähnlich negatives Bild von den
Mongolen gezeichnet: „Die Prälaten und Magnaten Ungarns wurden niedergemetzelt,
der König selbst floh nach Dalmatien und die Feinde [die Mongolen] verheerten alles
jenseits der Donau.“(Z. 51f.)

Das negative Mongolenbild wird noch direkter in Zusammenhang mit dem Charakter
der Mongolen beschrieben: „Denn dieses Volk ist wild und lebt außerhalb des Gesetzes
und kennt keine Menschlichkeit.“(Z. 64) Zuletzt wird von ihrem Äußeren ein negatives
Bild gezeichnet: „Sie haben breite Gesichter, blicken grimmig und stoßen ein
schreckliches Geschrei aus, das gut zu ihrer Gesinnung passt.“(Z. 68f.)

\subsection{Militärische Aspekte der Mongolendarstellung}
In dem vorhergehenden Teil wurden bereits das Vorgehen der Mongolen in
Grundzügen charakterisiert. Nun soll dargestellt werden, welche Eigenschaften ihnen
im militärischen Bereich gegeben werden.

Die militärischen Fähigkeiten der Mongolen werden sehr hoch eingeschätzt,
zumindest aber werden sie im Brief als überlegen gegenüber den Gegnern beschrieben
wie in Zeile 26ff., wenn geschrieben wird, dass den Mongolen „der Bogen eine nur
allzu vertraute Waffe ist, ebenso wie Wurfspieße und Pfeile, die sie ja ständig benutzen
und die stärker trainierte Arme haben als andere, diese Kumanen.“ Im Brief wird
außerdem häufig betont, dass die Mongolen sehr schnell und beweglich sind, was man
an bestimmter Lexik sehen kann wie in Zeile 35 an „plötzlich“ und in Zeile 48 an
„Wirbelwind“. Des Weiteren lässt sich erkennen, dass die Mongolen taktisch vielfältige
Vorgehensweisen kennen, wenn beispielsweise in Zeile 35 die Rede von einem
„Sturmangriff“ ist und in Zeile 48 davon, dass die Mongolen die Ungarn „umzingelten“.
Man könnte in diesem Sinne noch die Dreiteilung des mongolischen Heeres anfügen,
die in Zeile 58 bis 61 beschrieben wird. Diese Dreiteilung war dazu gedacht, die Ungarn
von mehreren Seiten anzugreifen und sie zu umzingeln.

Die erwähnte Ausrüstung des Bogens und der Wurfspieße ist nicht die einzige, derer
sich die Mongolen bedienen. So werden im Brief die Fortbewegungsmittel besonders
herausgestellt: „Die Tartaren, unvergleichliche Bogenschützen, tragen künstliche
Schläuche mit sich, mit denen sie reißende Flüsse und Seen durchschwimmen, ohne
Schaden zu nehmen.“(Z. 76ff.) Dies könnte sich darauf beziehen, dass auch die
Engländer in Gefahr sein könnten, da die Mongolen durchaus die Wasserstraße zum
englischen Festland überqueren könnten. Da ähnliche Briefe auch an andere Monarchen
geschickt wurden, u. a. an den französischen König\autocite[519]{heinisch1977} , lässt sich aus der Beschreibung der
Schläuche herauslesen, dass auch der Rhein als natürlich Grenze keinen Schutz für
Frankreich darstellt.

Ein weiteres wichtiges Fortbewegungsmittel der Mongolen wird in Zeile 78ff.
beschrieben: „Und wenn den Pferden, die sie mit sich führen, das Futter fehlt, dann sind
diese, wie man sagt, mit der Rinde von Bäumen und Blättern und Grasswurzeln
zufrieden und bleiben doch sehr schnell und sehr ausdauernd.“ Dies hätte
möglicherweise auch die Funktion, die Angst vor den Mongolen zu schüren, weil in
diesem Satz deutlich wird, dass das Vordringen der Mongolen sich nicht nur auf die
Steppengebiete beschränken kann, sondern durchaus auch die eher bewaldeten Gebiete
Süddeutschlands oder Englands betreffen kann.

Friedrich II. beschreibt die Organisation der Mongolen:
„Dennoch folgt es einem Herrn, den es gehorsam und ehrfurchtsvoll achtet und den
Herrn der Welt nennt. [...] Auf einen Wink ihres Führers stürzen sie sich in jede
beliebige Gefahr.“(Z. 65-68) Diese Einheit ist in einem Kontrast zu der im Brief in
Zeilen 81-90 geäußerten Uneinigkeit der europäischen Monarchen zu sehen.\autocite[80]{bezzola1974}

\subsection{Ziele der Mongolen}
Nachdem der Brief ausführlich in Bezug auf den Charakter und die das militärische
Vorgehen analysiert wurde, soll nun vorgestellt werden, welche Ziele der Verfasser den
Mongolen unterstellt. Dies ist, wie ich finde, umso wichtiger, da dies innerhalb des
diplomatischen historischen Kontextes durchaus eine Rolle spielen kann.

In diesem Aspekt ist der Brief nicht geradlinig geschrieben; es wird häufig von einer
Perspektive zu einer anderen gewechselt. So ist ein Ziel nach der Darstellung des Briefes, dass Gott die Mongolen (bzw.
Tataren) „bis in die Gegenwart erhalten [hat], um sein eigenes Volk zu züchtigen und zu
bessern [...]“(Z. 13f.) Eine andere Darstellung unterstellt Mongolen „[...] die
gesamte Erde mithilfe seiner ungeheuren und unvergleichlichen Macht und Menge an
Truppen allein zu beherrschen.“(Z. 19ff.) Dazu passt das Gesuch der Mongolen an den
ungarischen König, „[...] er solle sich selbst und sein Reich ihnen unterwerfen und so
im voraus ihre Gnade gewinnen “.(Z. 44f.)

Allgemein wird aber in der Beschreibung des Vordringens der Mongolen (Kap. 3.1)
klar, dass der Eindruck vermittelt werden soll, dass das einzige Ziel der Mongolen ist,
zu töten und Krieg zu führen. Darauf weist Lexik hin, die in dem Bereich
angesiedelt ist wie „morden“(Z. 34), „völlige Verwüstung“(Z. 39) usw. (genaue
Darstellung unter 4.1).

\subsection{Reaktion der Gegner}
Als letzten Punkt der Analyse lohnt es sich, die Reaktion der Gegner der Mongolen
anzuschauen, soweit diese im Brief dargestellt werden. Im Brief werden vor allem die
schlechte Vorbereitung und eine gewisse Hochmütigkeit der Ukrainer (Ruthenen)
gegenüber den Mongolen betont: „Obwohl sie so nahe waren, fühlten die Ruthenen,
[...], sich kaum veranlasst, Vorsicht walten zu lassen und sich zu verschanzen.“(Z. 30f.)

Ähnlich kritisch geht Friedrich II. mit den Ungarn um. Im Brief werden sie
„übermütig“ und „ahnungslos“ genannt, weil sie „ihre Feinde verachteten, während sie
trotz der Nähe der Feinde träge schlummerten und auf die natürlichen Bollwerke
vertrauten“(Z. 49f.).

Der letzte Punkt könnte in der Funktion als einer indirekten Warnung an den König
von England gerichtet sein, dass die Mongolen in der Lage sind, natürliche Barrieren zu
überwinden und damit unter Umständen den Kanal zwischen England und Frankreich
zu passieren oder auch gebirgige Landschaften zu überqueren.
\section{Einordnung in den historischen Kontext}
Nachdem die Quelle in ihrem Inhalt analysiert worden ist, soll sie in ihrer Form und
Funktion in den geschichtlichen Kontext eingebettet werden. Es soll gezeigt werden,
welche Ereignisse und Vorgänge in den Brief direkt oder indirekt einfließen und
inwiefern die vorliegende Quelle die Atmosphäre in Europa zur Zeit des Angriffes der
Mongolen widerspiegelt.
\subsection{Diplomatischer Kontext}
In gewisser Weise wird im Brief auf den diplomatischen Kontext referiert. Dies
passiert vor allem im letzten Absatz; da ist die Rede davon, dass bei Ausräumung der
Streitigkeiten die Mongolen als gemeinsame Feinde, „sich dann nicht so sehr freuen
[würden], dass unter den christlichen Fürsten derartige Zwietracht herrscht, dass sie
ihnen den Weg bereitet“(Z. 89f.). Die Beziehungen zwischen den Fürsten werden hier
als problematisch und strittig dargestellt. Zur Zeit der Abfassung des Briefes waren sie
dies tatsächlich, denn innerhalb der vorangegangen Jahre und Monate war es in Europa
zu bewegenden Ereignissen gekommen. Besonders zwischen dem Papst Gregor IX. und
Friedrich II. gab es wiederholt Konflikte, sodass der Papst zweimal den Bann über
Friedrich II. aussprach: Das erste Mal nach einem verzögerten Kreuzzug nach Jerusalem
1227. Dieser Bann wurde 1230 aufgehoben. 

Bedeutender für das Verständnis des
vorliegenden Briefes ist aber der Bann von 1239, der im Zuge einer kriegerischen
Politik
Friedrichs
II.
in
Italien
ausgesprochen
wurde.
Damit
wurde
„ein
Propagandakrieg von bisher unbekannten Dimensionen“\autocite[79]{beduerf2000} ausgelöst.

Der damit verbundene Streit zwischen dem Papst und Friedrich II. war in den
Vormonaten des Briefes auf einem Höhepunkt: Friedrich II. war mehrmals auf Rom
marschiert und belagerte papstfreundliche Städte wie Faenza und im Zeitraum des
Briefes war er nahe Rom. Der Streit zwischen den europäischen Königen spitzte sich
nicht zuletzt dadurch zu, weil Friedrich II. im Mai 1241 einen Schiffsverband mit hohen
christlichen Würdenträgern gewaltsam auflösen ließ und über 100 Bischöfe festsetzte\autocite[138]{rotter2000} .
Die öffentliche Meinung Europas war seit diesem Vorgang gegen ihn gerichtet.\autocite[69]{schaller1998}

In diesem Kontext bekommt der Brief zur Funktion einer Warnung vor den
Mongolen auch die Funktion, die oben beschriebenen Ereignisse in den Hintergrund
treten zu lassen und die Gunst der anderen Könige auf Grundlage einer Gefahr durch die
Mongolen zu gewinnen. Deswegen werden die Mongolen undiplomatisch und
kriegerisch dargestellt.

Im Übrigen sollte man auch auf das schauen, was im Brief nicht gesagt wird: Es
werden nämlich keine konkreten Schritte gegen die Mongolen vorgeschlagen. Die
Mongolen werden nur beschrieben und es ergeht ein Aufruf zur Einheit, aber darüber
hinaus schreibt Friedrich II. kein Wort über bevorstehende Militäroperationen oder
Absichten, gegen die Mongolen zu marschieren oder konkrete Verteidigungsmaßnahmen
zu treffen. Dies könnte man ebenfalls so deuten, als ob für Friedrich II. die zur Zeit des
Abfassens des Briefes stattfindende Belagerung Roms größere Priorität hatte als die
Mongolen. Dadurch wird nochmals bestätigt, dass der Brief als eine Beschwichtigung
angelegt und unter Umständen dazu gedacht war, die Herrscher Europas als Verbündete
gegen die Mongolen zu gewinnen, in Wirklichkeit aber seinen Feldzug gegen den Papst
fortzusetzen.

Auf jeden Fall könnte man den Brief auch dahingehend interpretieren, dass er
 die anderen Herrscher für den Kampf gegen die Mongolen animieren will –
ohne einen eigenen Beitrag, da dieser in der Quelle nicht erwähnt wird. Nach Rotter\autocite[138]{rotter2000}
 hatte Friedrich II. nicht die Absicht, gegen die Mongolen zu marschieren,
sondern wollte andere Herrscher zum Kampf gegen die Mongolen animieren.

\subsection{Der Brief innerhalb des Streites zwischen Friedrich II. und dem Papst}
Man könnte den Brief außer im diplomatischen Kontext auch im Vergleich der
allgemeinen zeitgenössischen Darstellungen sehen. Es offenbaren sich dabei durchaus
wichtige Informationen. Es ließe sich dadurch im Brief die Funktion der
Selbstdarstellung finden, denn in seiner Beschreibung der Mongolen spiegelt sich sein
weltlicher Charakter wider. In seinem Brief an Heinrich III. „wartet [er] mit Details über
die Lebensweise, das Aussehen und die Bewaffnung auf widerspricht den Stereotypen
von den entfesselten Scharen der Endzeit“\autocite[470]{rader2010}. Mit Blick auf die Zeilen 64 bis 80 der
Quelle lässt sich die These von Rader\autocite[470]{rader2010} belegen, dass es Friedrich II. mehr an
einer Beobachtung lag und nicht einer Vision. Damit ist bei ihm auch hier ein weltliches
Wissenschaftsverständnis zu finden.

Diesen Brief könnte man außerdem auch im Kontext des Propagandakrieges
zwischen dem Papst und Friedrich II. sehen. In der Zeit des 13. Jahrhunderts existierten
viele Theorien, die die Welt in einer Tendenz zur Endzeit begriffen. Im Streit mit
Friedrich II. nutzte der Papst diese Theorien, um Friedrich II als den „Vorläufer des
Antichrists“\autocite[69]{schaller1998} zu diffamieren und die öffentliche Meinung gegen ihn zu lenken. Die
Mongolen verstärkten bei manchen den Glauben, dass die Endzeit nahe sei. In der
päpstlichen Propaganda gegen Friedrich II. wurde dieser häufig mit den Mongolen
assoziiert und auch als Verbündeter der Mongolen bezeichnet.\autocite[469]{rader2010}

Für Friedrich galt es, diesen Vorwürfen entgegenzutreten und gewissermaßen die
Erwartungen an die Endzeit aufzulösen, die Mongolen als menschlich darzustellen und
seine Person von ihnen zu distanzieren. Ich denke, dass diese Absichten den Brief
durchziehen. Die in der damaligen Publizistik verbreitete Meinung, dass die Mongolen
aus dem Land von Gog und Magog stammten und ein jüdisches Weltreich errichten
wollen, stellt Friedrich II. ein eher geografisches Verständnis durch die Benutzung der
Begriffe Süden und Norden (Z. 8-10) vor. Außerdem distanziert er sich klar von den
Vorstellungen, dass die Mongolen aus dem Land Gog und Magog kämen, indem er
zugibt, dass es nicht bekannt sei, „woher es kommt und welchen Ursprung es hat“ (Z.
12). Damit widerspricht er indirekt den Vorwürfen, dass er ein Verbündeter der
Mongolen sei. Es werden auch keine Legenden genannt, die mit dem Namen „Tartari“
damals assoziiert wurden. Und es scheint, dass dieser Name ohne negative Konnotation
im Brief verwendet wird.

Weiterhin hat die Analyse ergeben, dass die Mongolen zwar negativ dargestellt
werden, es aber keine akribischen Versuche gemacht werden, die Mongolen übermenschlich
erscheinen zu lassen. Die Mongolen sollen natürlich als Feinde dargestellt werden und
in ihren Absichten werden sie im Brief auf das Kriegführen und Morden begrenzt. Dies
passiert sicherlich in der Absicht die Mongolen als undiplomatisch darzustellen, um
Bündnisabsichten anderer Herrscher mit ihnen zu untergraben. Doch in der
Beschreibung ihres militärischen Vorgehens und der Reaktion der Europäer werden die
Mongolen wiederum als nicht unbesiegbar beschrieben. Es wird zwar von keiner
Niederlage ihrerseits berichtet, aber ihre Überlegenheit basiert dem Brief nach auf
physischer Stärke (Z. 27). Ihre Ausrüstung wie Bogen oder Eisenplatten ist nichts, was
die Europäer nicht kennen würden, sodass der Brief auch hier die Mongolen eher real
beschreibt. Nicht zu vergessen ist die im Brief beschriebene Reaktion der Gegner der
Mongolen. Es erscheint dabei der Eindruck, dass ihre Niederlagen selbst verschuldet
waren, weil beispielsweise die Ruthenen oder die Ungarn die von den Mongolen
ausgehende Bedrohung nicht ernst genommen haben wie in Kap. 3.4 analysiert worden ist.
Zuletzt ist die Darstellung der guten Organisationsstruktur der Mongolen (3.3) eine
Anspielung auf die Uneinigkeit der europäischen Könige.

Zusammenfassend stellt der Brief den damals gängigen Prophezeiungen von der
Endzeit eine Sicht auf die Mongolen entgegen, die diese als Feinde, aber auch als
Menschen darstellt, die nicht durch übermenschliche Kräfte gesiegt haben, sondern
durch die Fehler der Gegner bzw. durch militärische Mittel.
\newpage
\section*{Schlussbetrachtung}
\addcontentsline{toc}{section}{Schlussbetrachtung}

In der
Quellenbeschreibung wurde die Quelle in ihren Formalien beschrieben und es wurde
deutlich, dass diese Korrespondenz zwischen sehr bedeutenden Männern ihrer Zeit
ablief. In der Inhaltsangabe wurde die Struktur des Textes wiedergegeben, so wie sie im
Textfluss zu finden ist. Man merkte dabei, dass Friedrich II. seinen Text zum Teil in
einer chronologischen Reihenfolge strukturiert hat – vom Ursprung der Mongolen über
die Siege über die Kumanen und Ruthenen bis zu ihrem Sieg gegen die Ungarn.

Die weitere Analyse hat aufgezeigt, dass der Text durchaus in einer anderen Art
wiedergegeben werden kann und dass man die Informationen auf eine andere Weise
kategorisieren kann als nach der rein chronologischen. Es wurde bei der Analyse erarbeitet,
dass die Quelle die Mongolen in vielfältigen Aspekten beschreibt, zum Beispiel in
ihrem Aussehen, ihrer Bewaffnung und ihren Zielen sowie der Reaktion der Gegner.
Soweit die zusammengefassten Erkenntnisse zur Form des Briefes.

Mit der Beachtung des historischen Kontextes konnte man auch die Funktion des
Briefes feststellen: Zum einen sollte der Brief die Aufmerksamkeit der europäischen
Könige und besonders Heinrich III. auf die Mongolen lenken und den Krieg Friedrichs
II. gegen den Papst durch die Nichterwähnung in den Hintergrund treten zu lassen. Man
konnte erkennen, dass der Brief zum Teil auch als Selbstdarstellung Friedrichs II. als
weltlichen und christlichen Herrscher angesehen werden kann. Dies war als eine Erwiderung auf die
Propaganda des Papstes gedacht, in welcher er als Antichrist bezeichnet wurde. Nicht
zuletzt galt es für Friedrich, durch einen Brief eine Gegenposition in Bezug auf die
Untergangsstimmung in der Gesellschaft zu schaffen. Die Mongolen sollten als sehr
stark, aber auch menschlich dargestellt werden. Durch die Darstellung der Einheit der
Mongolen, sollten die Christen indirekt zu einem geeinten Vorgehen gegen die
Mongolen aufgerufen werden – mit Friedrich II als dem Anführer.
\addcontentsline{toc}{section}{Literatur}
%\bibliography
\printbibliography
\end{document}
