%%This is a very basic article template.
%%There is just one section and two subsections.
\documentclass[a4paper]{scrartcl}
%\usepackage[]{graphicx}
%\usepackage{booktabs} 
\usepackage[T1]{fontenc}
\usepackage[utf8]{inputenc}
\usepackage[ngerman]{babel}
\usepackage{bibgerm} 
\usepackage{natbib}
%\usepackage{microtype} 
\usepackage[onehalfspacing]{setspace}
\usepackage[a4paper]{geometry}
\geometry{left=4.5cm,right=2cm,top=2cm, bottom=2cm}
\newcommand\autocite[1]{\cite[]{#1}}

 
\begin{document}
\section*{Vergleich der Silbensysteme des Russischen mit Hilfe eines Natural Language Processing-Ansatzes}

Meine Arbeit wird sich mit Silbensystemen des Russischen beschäftigen. Dabei soll ein qualitativer Ansatz mit einem quantitativen verbunden werden. Es soll der komplette Kreislauf eines Programms beschrieben werden: Theorie, Umsetzung, Test, Resultate. Die technische Umsetzung soll grö\ss eren Anteil an der Arbeit bekommen als die Analyse. Dies ist dem geschuldet, dass der technische Teil auch mehr Zeit (ca. 5 Wochen) in Anspruch nehmen wird.  \citep{bird2009}
   
\textbf{Beginnend} mit der Rolle der Silbe in der Sprache nach \citet{cholin2011}, sollen im ersten Kapitel die bisherigen Ansätze zur Segmentierung von Wörtern in Silben beschrieben werden. Hauptsächlich gibt es für das Russische drei Ansätze \cite{scherba1983} \& \cite{vino1953}, \cite{kasatkin2001} und \cite{bondarko1998}. Die größten Unterschiede liegen in der Aufteilung von Konsonantenclustern zwischen Vokalen. Dabei gibt es noch andere Ansätze wie \cite{deterding2001} und experimentelle Daten von \cite{cote2011}, die eher den Systemen der russischen Wissenschaftler widersprechen.

Trotz der reichen theoretischen Grundlage wurde aber noch nicht untersucht, wie sich die unterschiedlichen Ansätze in einem Corpus unterscheiden. Während es in anderen Sprachen relativ gute Ansätze \cite{padro2012} \& \cite{iaco2011} gibt, um quantitative Aspekte der Silbenphonologie zu berücksichtigen, mangelt es im Russischen daran. Auch der Russische National-Korpus wurde bisher nur graphemisch untersucht. Für die vorliegende Arbeit wird ein 1-Million Ausschnitt des Russischen National-Korpus\footnote{www.ruscorpora.ru} verwendet.

Innerhalb der Implementierung (\textbf{Kap. 2}) sollen die algorhythmischen Ansätze \cite{oakes1998} \& \autocite{jurafsky2008} zu einer Umsetzung des Programms besprochen werden, u.a. soll auch die Bedeutung eines solchen Programms im Englischen gezeigt werden anhand von SounDex \citep{bird2009}. Als nächstes sollen die Werkzeuge vorgestellt werden, mit denen der Korpus analysiert werden soll. Der Silbenkorpus wird wohl aus ca. 2-3 Millionen Silben bestehen und es ist daher auch angemessen über die Code-Profiling Prozeduren \cite{mckinney2012} zu sprechen, die eine relativ schnelle Bearbeitung ermöglichen könnten. Dies ist umso wichtiger, da das Programm Open-Source geführt werden soll\footnote{Als Entwicklungsplattform soll die Versionkontrolle Git mit der Repository https://github.com/ verwendet werden.} und von anderen weiterentwickelt werden kann\footnote{Als Lizenzvertrag habe ich die MIT-Licence oder die Creative Commons 3.0 im Blick.}.

In \textbf{Kapitel 3 und 4} sollen die Resultate des Silbenparsers gegenüberstellt und mit Hilfe von quantitativen Korpus-Ansätzen gezeigt werden, wie sich die Silben in den unterschiedlichen Ansätzen clustern \cite{lebart1998} und wie weit die Unterschiede zwischen den Systemen signifikant sind \cite{baayen2001}.
 
Als Ausblick (\textbf{Kap. 5}) sollen die Möglichkeiten der Verwendung des Programm besprochen werden. 
    
\bibliographystyle{agsm}
\bibliography{../syll_lit}

\end{document}
